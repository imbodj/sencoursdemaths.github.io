\documentclass[12pt, a4paper]{report}

% LuaLaTeX :

\RequirePackage{iftex}
\RequireLuaTeX

% Packages :

\usepackage[french]{babel}
%\usepackage[utf8]{inputenc}
%\usepackage[T1]{fontenc}
\usepackage[pdfencoding=auto, pdfauthor={Hugo Delaunay}, pdfsubject={Mathématiques}, pdfcreator={agreg.skyost.eu}]{hyperref}
\usepackage{amsmath}
\usepackage{amsthm}
%\usepackage{amssymb}
\renewcommand{\proofname}{Solution}
\usepackage{stmaryrd}
\usepackage{tikz}
\usepackage{tkz-euclide}
\usepackage{fontspec}
\defaultfontfeatures[Erewhon]{FontFace = {bx}{n}{Erewhon-Bold.otf}}
\usepackage{fourier-otf}
\usepackage[nobottomtitles*]{titlesec}
\usepackage{fancyhdr}
\usepackage{listings}
\usepackage{catchfilebetweentags}
\usepackage[french, capitalise, noabbrev]{cleveref}
\usepackage[fit, breakall]{truncate}
\usepackage[top=2.5cm, right=2cm, bottom=2.5cm, left=2cm]{geometry}
\usepackage{enumitem}
\usepackage{tablists} %Pour faire 1)  2) 3)
\usepackage{tocloft}
\usepackage{microtype}
%\usepackage{mdframed}
%\usepackage{thmtools}
\usepackage{xcolor}
\usepackage{tabularx}
\usepackage{xltabular}
\usepackage{aligned-overset}
\usepackage[subpreambles=true]{standalone}
\usepackage{environ}
\usepackage[normalem]{ulem}
\usepackage{multicol}
 \usepackage{variations}
\usepackage{array}% Pour faire des tableaux
\usepackage{etoolbox}
\usepackage{setspace}
\usepackage[bibstyle=reading, citestyle=draft]{biblatex}
\usepackage{xpatch}
\usepackage[many, breakable]{tcolorbox}
\usepackage[backgroundcolor=white, bordercolor=white, textsize=scriptsize]{todonotes}
\usepackage{luacode}
\usepackage{float}
\usepackage{needspace}


% Police :

\setmathfont{Erewhon Math}

% Tikz :

\usetikzlibrary{calc}
\usetikzlibrary{3d}

% Longueurs :

\setlength{\parindent}{0pt}
\setlength{\headheight}{15pt}
\setlength{\fboxsep}{0pt}
\titlespacing*{\chapter}{0pt}{-20pt}{10pt}
\setlength{\marginparwidth}{1.5cm}
\setstretch{1.1}

% Métadonnées :

\author{agreg.skyost.eu}
\date{\today}

% Titres :

\setcounter{secnumdepth}{3}

\renewcommand{\thechapter}{\Roman{chapter}}
\renewcommand{\thesubsection}{\Roman{subsection}}
\renewcommand{\thesubsubsection}{\arabic{subsubsection}}
\renewcommand{\theparagraph}{\alph{paragraph}}

\titleformat{\chapter}{\huge\bfseries}{\thechapter}{20pt}{\huge\bfseries}
\titleformat*{\section}{\LARGE\bfseries}
\titleformat{\subsection}{\Large\bfseries}{\thesubsection \, - \,}{0pt}{\Large\bfseries}
\titleformat{\subsubsection}{\large\bfseries}{\thesubsubsection. \,}{0pt}{\large\bfseries}
\titleformat{\paragraph}{\bfseries}{\theparagraph. \,}{0pt}{\bfseries}

\setcounter{secnumdepth}{4}

% Table des matières :

\renewcommand{\cftsecleader}{\cftdotfill{\cftdotsep}}
\addtolength{\cftsecnumwidth}{10pt}

% Redéfinition des commandes :

\renewcommand*\thesection{\arabic{section}}
\renewcommand{\ker}{\mathrm{Ker}}

% Nouvelles commandes :

\newcommand{\website}{http://sencoursdemaths.com}

\newcommand{\tr}[1]{\mathstrut ^t #1}
\newcommand{\im}{\mathrm{Im}}
\newcommand{\rang}{\operatorname{rang}}
\newcommand{\trace}{\operatorname{trace}}
\newcommand{\id}{\operatorname{id}}
\newcommand{\stab}{\operatorname{Stab}}
\newcommand{\paren}[1]{\left(#1\right)}
\newcommand{\accol}[1]{\left\{#1\right\}}
\newcommand{\croch}[1]{\left[ #1 \right]}
\newcommand{\Grdcroch}[1]{\Bigl[ #1 \Bigr]}
\newcommand{\grdcroch}[1]{\bigl[ #1 \bigr]}
\newcommand{\abs}[1]{\left\lvert #1 \right\rvert}
\newcommand{\limi}[3]{\displaystyle \lim_{#1\to #2}#3}
\newcommand{\pinf}{+\infty}
\newcommand{\minf}{-\infty}
%%%%%%%%%%%%%% ENSEMBLES %%%%%%%%%%%%%%%%%
\newcommand{\ensemblenombre}[1]{\mathbb{#1}}
\newcommand{\Nn}{\ensemblenombre{N}}
\newcommand{\Zz}{\ensemblenombre{Z}}
\newcommand{\Qq}{\ensemblenombre{Q}}
\newcommand{\Qqp}{\Qq^+}
\newcommand{\Rr}{\ensemblenombre{R}}
\newcommand{\Cc}{\ensemblenombre{C}}
\newcommand{\Nne}{\Nn^*}
\newcommand{\Zze}{\Zz^*}
\newcommand{\Zzn}{\Zz^-}
\newcommand{\Qqe}{\Qq^*}
\newcommand{\Rre}{\Rr^*}
\newcommand{\Rrp}{\Rr_+}
\newcommand{\Rrm}{\Rr_-}
\newcommand{\Rrep}{\Rr_+^*}
\newcommand{\Rrem}{\Rr_-^*}
\newcommand{\Cce}{\Cc^*}
%%%%%%%%%%%%%%  INTERVALLES %%%%%%%%%%%%%%%%%
\newcommand{\intff}[2]{\left[#1\;,\; #2\right]  }
\newcommand{\intof}[2]{\left]#1 \;, \;#2\right]  }
\newcommand{\intfo}[2]{\left[#1 \;,\; #2\right[  }
\newcommand{\intoo}[2]{\left]#1 \;,\; #2\right[  }



\providecommand{\newpar}{\\[\medskipamount]}

\newcommand{\annexessection}{%
  \newpage%
  \subsection*{Annexes}%
}

\providecommand{\lesson}[3]{%
  \title{#3}%
  \hypersetup{pdftitle={#2 : #3}}%
  \setcounter{section}{\numexpr #2 - 1}%
  \section{#3}%
  \fancyhead[R]{\truncate{0.73\textwidth}{#2 : #3}}%
}

\providecommand{\development}[3]{%
  \title{#3}%
  \hypersetup{pdftitle={#3}}%
  \section*{#3}%
  \fancyhead[R]{\truncate{0.73\textwidth}{#3}}%
}

\providecommand{\sheet}[3]{\development{#1}{#2}{#3}}

\providecommand{\ranking}[1]{%
  \title{Terminale #1}%
  \hypersetup{pdftitle={Terminale #1}}%
  \section*{Terminale #1}%
  \fancyhead[R]{\truncate{0.73\textwidth}{Terminale #1}}%
}

\providecommand{\summary}[1]{%
  \textit{#1}%
  \par%
  \medskip%
}

\tikzset{notestyleraw/.append style={inner sep=0pt, rounded corners=0pt, align=center}}

%\newcommand{\booklink}[1]{\website/bibliographie\##1}
\newcounter{reference}
\newcommand{\previousreference}{}
\providecommand{\reference}[2][]{%
  \needspace{20pt}%
  \notblank{#1}{
    \needspace{20pt}%
    \renewcommand{\previousreference}{#1}%
    \stepcounter{reference}%
    \label{reference-\previousreference-\thereference}%
  }{}%
  \todo[noline]{%
    \protect\vspace{20pt}%
    \protect\par%
    \protect\notblank{#1}{\cite{[\previousreference]}\\}{}%
    \protect\hyperref[reference-\previousreference-\thereference]{p. #2}%
  }%
}

\definecolor{devcolor}{HTML}{00695c}
\providecommand{\dev}[1]{%
  \reversemarginpar%
  \todo[noline]{
    \protect\vspace{20pt}%
    \protect\par%
    \bfseries\color{devcolor}\href{\website/developpements/#1}{[DEV]}
  }%
  \normalmarginpar%
}

% En-têtes :

\pagestyle{fancy}
\fancyhead[L]{\truncate{0.23\textwidth}{\thepage}}
\fancyfoot[C]{\scriptsize \href{\website}{\texttt{http://sencoursdemaths.com}}}

% Couleurs :

\definecolor{property}{HTML}{ffeb3b}
\definecolor{proposition}{HTML}{ffc107}
\definecolor{lemma}{HTML}{ff9800}
\definecolor{theorem}{HTML}{f44336}
\definecolor{corollary}{HTML}{e91e63}
\definecolor{definition}{HTML}{673ab7}
\definecolor{notation}{HTML}{9c27b0}
\definecolor{example}{HTML}{00bcd4}
\definecolor{cexample}{HTML}{795548}
\definecolor{application}{HTML}{009688}
\definecolor{remark}{HTML}{3f51b5}
\definecolor{algorithm}{HTML}{607d8b}
\definecolor{proof}{HTML}{e1f5fe}
\definecolor{exercice}{HTML}{e1f5fe}

% Théorèmes :

\theoremstyle{definition}
\newtheorem{theorem}{Théorème}

\newtheorem{property}[theorem]{Propriété}
\newtheorem{proposition}[theorem]{Proposition}
\newtheorem{lemma}[theorem]{Activité d'introduction}
\newtheorem{corollary}[theorem]{Conséquence}

\newtheorem{definition}[theorem]{Définition}
\newtheorem{notation}[theorem]{Notation}

\newtheorem{example}[theorem]{Exemple}
\newtheorem{cexample}[theorem]{Contre-exemple}
\newtheorem{application}[theorem]{Application}

\newtheorem{algorithm}[theorem]{Algorithme}
\newtheorem{exercice}[theorem]{Exercice}

\theoremstyle{remark}
\newtheorem{remark}[theorem]{Remarque}




\counterwithin*{theorem}{section}

\newcommand{\applystyletotheorem}[1]{
  \tcolorboxenvironment{#1}{
    enhanced,
    breakable,
    colback=#1!8!white,
    %right=0pt,
    %top=8pt,
    %bottom=8pt,
    boxrule=0pt,
    frame hidden,
    sharp corners,
    enhanced,borderline west={4pt}{0pt}{#1},
    %interior hidden,
    sharp corners,
    after=\par,
  }
}

\applystyletotheorem{property}
\applystyletotheorem{proposition}
\applystyletotheorem{lemma}
\applystyletotheorem{theorem}
\applystyletotheorem{corollary}
\applystyletotheorem{definition}
\applystyletotheorem{notation}
\applystyletotheorem{example}
\applystyletotheorem{cexample}
\applystyletotheorem{application}
\applystyletotheorem{remark}
%\applystyletotheorem{proof}
\applystyletotheorem{algorithm}
\applystyletotheorem{exercice}

% Environnements :

\NewEnviron{whitetabularx}[1]{%
  \renewcommand{\arraystretch}{2.5}
  \colorbox{white}{%
    \begin{tabularx}{\textwidth}{#1}%
      \BODY%
    \end{tabularx}%
  }%
}

% Maths :

\DeclareFontEncoding{FMS}{}{}
\DeclareFontSubstitution{FMS}{futm}{m}{n}
\DeclareFontEncoding{FMX}{}{}
\DeclareFontSubstitution{FMX}{futm}{m}{n}
\DeclareSymbolFont{fouriersymbols}{FMS}{futm}{m}{n}
\DeclareSymbolFont{fourierlargesymbols}{FMX}{futm}{m}{n}
\DeclareMathDelimiter{\VERT}{\mathord}{fouriersymbols}{152}{fourierlargesymbols}{147}

% Code :

\definecolor{greencode}{rgb}{0,0.6,0}
\definecolor{graycode}{rgb}{0.5,0.5,0.5}
\definecolor{mauvecode}{rgb}{0.58,0,0.82}
\definecolor{bluecode}{HTML}{1976d2}
\lstset{
  basicstyle=\footnotesize\ttfamily,
  breakatwhitespace=false,
  breaklines=true,
  %captionpos=b,
  commentstyle=\color{greencode},
  deletekeywords={...},
  escapeinside={\%*}{*)},
  extendedchars=true,
  frame=none,
  keepspaces=true,
  keywordstyle=\color{bluecode},
  language=Python,
  otherkeywords={*,...},
  numbers=left,
  numbersep=5pt,
  numberstyle=\tiny\color{graycode},
  rulecolor=\color{black},
  showspaces=false,
  showstringspaces=false,
  showtabs=false,
  stepnumber=2,
  stringstyle=\color{mauvecode},
  tabsize=2,
  %texcl=true,
  xleftmargin=10pt,
  %title=\lstname
}

\newcommand{\codedirectory}{}
\newcommand{\inputalgorithm}[1]{%
  \begin{algorithm}%
    \strut%
    \lstinputlisting{\codedirectory#1}%
  \end{algorithm}%
}





\begin{document}
  %<*content>
  \development{algebra}{theoreme-de-wantzel}{Théorème de Wantzel}

  \summary{Une application sympathique de la théorie des corps en géométrie. Les arguments sont assez simples et donnent lieu à de jolies applications.}

  \begin{notation}
    On note $\mathbb{E}$ l'ensemble des nombres constructibles. Tout au long du développement, on se permettra de confondre points et coordonnées.
  \end{notation}

  \reference[GOZ]{49}

  \begin{lemma}
    \label{theoreme-de-wantzel-1}
    $\mathbb{E}$ contient le corps $\mathbb{Q}$.
  \end{lemma}

  \begin{proof}
    Tout élément $z \in \mathbb{Z}$ est constructible. Soit $(p,q) \in \mathbb{Z}^* \times \mathbb{N}^*$. Les points $P = (p,0)$ et $Q = (0,q)$ sont constructibles. On considère la droite $(d)$, parallèle à $(PQ)$ passant par $(0,1)$. Cette droite est constructible, et son point d'intersection avec la droite passant par les points $(0,0)$ et $(1,0)$ est $\left( \frac{p}{q}, 0 \right)$ par le théorème de Thalès.
    \begin{center}
      \begin{tikzpicture}
        \coordinate (P) at (1.5,0);
        \coordinate (Q) at (0,2);
        \coordinate (C) at (0,0);
        \coordinate (D) at (0,1);
        \coordinate (E) at (0.75,0);
        \draw (-2.5,0) -- (2.5,0);
        \draw (0,-1.5) -- (0,2.5);
        \draw[teal] ($(P)!1.4!(Q)$) -- ($(Q)!1.4!(P)$);
        \draw[teal] ($(D)!2!(E)$) -- ($(E)!2.4!(D)$);
        \node at (P) [above right] {$P$};
        \node at (P) {$\bullet$};
        \node at (Q) [above right] {$Q$};
        \node at (Q) {$\bullet$};
        \node at (C) [below left] {$(0,0)$};
        \node at (C) {$\bullet$};
        \node at (D) [below left] {$(0,1)$};
        \node at (D) {$\bullet$};
        \node at (E) [below] {\color{teal}$\left( \frac{p}{q}, 0 \right)$};
        \node at (E) {\color{teal}$\bullet$};
      \end{tikzpicture}
    \end{center}
    Donc $\frac{p}{q} \in \mathbb{E}$. Comme $0 \in \mathbb{E}$, on a bien $\mathbb{Q} \subseteq \mathbb{E}$.
  \end{proof}

  \begin{lemma}
    \label{theoreme-de-wantzel-2}
    $\mathbb{E}$ est un sous-corps de $\mathbb{R}$ stable par racine carrée.
  \end{lemma}

  \begin{proof}
    Soient $u, v \in \mathbb{E}$. Commençons par montrer que $\mathbb{E}$ est un sous-corps de $\mathbb{R}$.
    \begin{itemize}
      \item Le point $(u, 0)$ est constructible donc son symétrique $(-u, 0)$ l'est aussi. Donc $-u \in \mathbb{E}$.
      \begin{center}
        \begin{tikzpicture}
          \draw (-2.5,0) -- (2.5,0);
          \draw (0,-1.5) -- (0,1.5);
          \node (A) at (1,0) {$\bullet$};
          \node at (A) [above right] {$(u,0)$};
          \node (B) at (-1,0) {\color{teal}$\bullet$};
          \node at (B) [above left] {\color{teal}$(-u,0)$};
          \draw[teal] (0,0) circle (1);
        \end{tikzpicture}
      \end{center}
      \item La droite passant par les points $(0, u)$ et $(-u, 0)$ et la droite passant par les points $(v, 0)$ et $(v, v)$ ont pour point d'intersection $(v, u + v)$ (par le théorème de Thalès). Donc $u + v \in \mathbb{E}$.
      \begin{center}
        \begin{tikzpicture}
          \coordinate (A) at (0,1.5);
          \coordinate (B) at (-1.5,0);
          \coordinate (C) at (1,0);
          \coordinate (D) at (1,1);
          \coordinate (E) at (1,2.5);
          \draw[teal] ($(B)!2!(A)$) -- ($(A)!1.5!(B)$);
          \draw[teal] ($(C)!3!(D)$) -- ($(D)!1.8!(C)$);
          \draw[cyan, dashed] (D) -- (0,0);
          \draw (-2.5,0) -- (2.5,0);
          \draw (0,-1.5) -- (0,2.5);
          \node at (A) [above left] {$(0,u)$};
          \node at (A) {$\bullet$};
          \node at (B) [above left] {$(-u,0)$};
          \node at (B) {$\bullet$};
          \node at (C) [below right] {$(v,0)$};
          \node at (C) {$\bullet$};
          \node at (D) [right] {$(v,v)$};
          \node at (D) {$\bullet$};
          \node at (E) [right] {\color{teal}$(v,u+v)$};
          \node at (E) {\color{teal}$\bullet$};
        \end{tikzpicture}
      \end{center}
      \item D'après ce qui précède, $v + 1$ et $v + 1 - u$ appartiennent à $\mathbb{E}$. La droite passant par les points $(v + 1 - u, v + 1)$ et $(u, v)$ et la droite passant par les points $(0,0)$ et $(1,0)$ ont pour point d'intersection $(uv, 0)$ (par le théorème de Thalès). Donc $uv \in \mathbb{E}$.
      \begin{center}
        \begin{tikzpicture}
          \coordinate (A) at (1.5,3);
          \coordinate (B) at (2,2);
          \coordinate (C) at (0,0);
          \coordinate (D) at (1,0);
          \coordinate (E) at (3,0);
          \draw (-1.5,0) -- (2.5,0);
          \draw (0,-1.5) -- (0,3.5);
          \draw[teal] ($(A)!3.5!(B)$) -- ($(B)!1.5!(A)$);
          \draw[teal] ($(C)!3.5!(D)$) -- ($(D)!1.5!(C)$);
          \draw[cyan, dashed] (C) -- (A);
          \draw[cyan, dashed] (D) -- (B);
          \node at (A) [above right] {$(v+1-u,v+1)$};
          \node at (A) {$\bullet$};
          \node at (B) [above right] {$(v,v)$};
          \node at (B) {$\bullet$};
          \node at (C) [below left] {$(0,0)$};
          \node at (C) {$\bullet$};
          \node at (D) [below] {$(1,0)$};
          \node at (D) {$\bullet$};
          \node at (E) [below right] {\color{teal}$(uv,0)$};
          \node at (E) {\color{teal}$\bullet$};
        \end{tikzpicture}
      \end{center}
      \item On suppose $u \neq 0$. La droite passant par les points $(1,0)$ et $(u,-1)$ et la droite passant par les points $(0,0)$ et $(1,1)$ ont pour point d'intersection $(u^{-1}, u^{-1})$ (par le théorème de Thalès). Donc $u^{-1} \in \mathbb{E}$.
      \begin{center}
        \begin{tikzpicture}
          \coordinate (A) at (1,0);
          \coordinate (B) at (0.5,-1);
          \coordinate (C) at (0,0);
          \coordinate (D) at (1,1);
          \coordinate (E) at (2,2);
          \draw (-1.5,0) -- (2.5,0);
          \draw (0,-1.5) -- (0,3.5);
          \draw[teal] ($(A)!1.5!(B)$) -- ($(B)!3.5!(A)$);
          \draw[teal] ($(C)!2.5!(D)$) -- ($(D)!2.5!(C)$);
          \draw[cyan, dashed] (-1,-1) -- (B);
          \node at (A) [below right] {$(1,0)$};
          \node at (A) {$\bullet$};
          \node at (B) [below right] {$(u,-1)$};
          \node at (B) {$\bullet$};
          \node at (C) [above left] {$(0,0)$};
          \node at (C) {$\bullet$};
          \node at (D) [above] {$(1,1)$};
          \node at (D) {$\bullet$};
          \node at (E) [right] {\color{teal}$(u^{-1},u^{-1})$};
          \node at (E) {\color{teal}$\bullet$};
        \end{tikzpicture}
      \end{center}
    \end{itemize}
    Ainsi, $\mathbb{E}$ est un sous-corps de $\mathbb{R}$, qui contient $\mathbb{Q}$ par le \cref{theoreme-de-wantzel-1}. Maintenant, soit $x \in \mathbb{E}$ avec $x > 0$. Comme $\mathbb{E}$ est un sous-corps de $\mathbb{R}$, on a $\frac{x+1}{2} \in \mathbb{E}$. Le cercle de centre $\left( \frac{x+1}{2}, 0 \right)$ passant par $(0,0)$ et la droite passant par les points $(x,0)$ et $(x,x)$ ont pour point d'intersection $(x,\sqrt{x})$ et $(x,-\sqrt{x})$ par le théorème de Pythagore. Donc $\sqrt{x} \in \mathbb{E}$.
    \begin{center}
      \begin{tikzpicture}
        \draw[teal] (1.5,0) circle (1.5);
        \coordinate (A) at (2,0);
        \coordinate (B) at (2,2);
        \coordinate (C) at (0,0);
        \coordinate (D) at (1.5,0);
        \coordinate (E) at (2,1.41421356);
        \coordinate (F) at (2,-1.41421356);
        \draw (-2.5,0) -- (2.5,0);
        \draw (0,-2.5) -- (0,2.5);
        \draw[teal] ($(A)!1.15!(B)$) -- ($(B)!2.15!(A)$);
        \draw[cyan, dashed] (C) -- (E);
        \draw[cyan, dashed] (C) -- (F);
        \draw[red] (A) -- ($(A)+(0,0.3)$) -- ($(A)+(-0.3,0.3)$) -- ($(A)+(-0.3,0)$) -- cycle;
        \node at (A) {$\bullet$};
        \node at (A) [above right] {$(x,0)$};
        \node at (B) {$\bullet$};
        \node at (B) [above right] {$(x,x)$};
        \node at (C) {$\bullet$};
        \node at (C) [below left] {$(0,0)$};
        \node at (D) {$\bullet$};
        \node at (D) [below] {$\left( \frac{x+1}{2}, 0 \right)$};
        \node at (E) {\color{teal}$\bullet$};
        \node at (E) [above right] {\color{teal}$(x,\sqrt{x})$};
        \node at (F) {\color{teal}$\bullet$};
        \node at (F) [below right] {\color{teal}$(x,-\sqrt{x})$};
      \end{tikzpicture}
    \end{center}
  \end{proof}

  \begin{theorem}[Wantzel]
    Soit $\alpha \in \mathbb{R}$. Alors, $\alpha \in \mathbb{E}$ si et seulement s'il existe une suite finie $(L_0, \dots, L_p)$ de sous-corps de $\mathbb{R}$ vérifiant :
    \begin{enumerate}[label=(\roman*)]
      \item $L_0 = \mathbb{Q}$.
      \item $\forall i \in \llbracket 0, p-1 \rrbracket$, $L_{i+1}$ est une extension quadratique (de degré $2$) de $L_i$.
      \item $\alpha \in L_p$.
    \end{enumerate}
  \end{theorem}

  \reference[ULM18]{103}

  \begin{proof}
    On suppose $\alpha$ constructible. Alors, il existe un point $M$ tel que $\alpha$ est l'abscisse de $M$. $M$ s'obtient à l'aide d'un nombre fini de constructions de points $M_1, \dots, M_m$. Pour tout $i \in \llbracket 1, m \rrbracket$, on note $(x_i, y_i)$ les coordonnées de $M_i$. De ce fait, on a une tour d'extension
    \[ \underbrace{K_0}_{= \mathbb{Q}} \subseteq K_1 \subseteq \dots \subseteq K_m \]
    avec $\alpha \in K_m$ et pour tout $0 \in \llbracket 1, m-1 \rrbracket$, $K_{i+1} = K_i(x_i, y_i)$. Soit $i \in \llbracket 1, m-1 \rrbracket$. Montrons que $[K_{i+1}:K_i] \leq 2$. On a différents cas possibles :
    \begin{itemize}
      \item \uline{$M_i$ est l'intersection de deux droites} passant par des nombres constructibles de $K_i$. Alors, les coordonnées $(x_i, y_i)$ de $M_i$ sont solution d'un système d'équations de la forme
      \[
        \begin{cases}
          ax + by = c \\
          a'x + b'y = c'
        \end{cases}
      \]
      avec $a, b, c, a', b', c' \in K_i$ par construction. Donc, $x_i, y_i \in K_i$ et ainsi, $[K_{i+1}:K_i] = 1$.
      \item \uline{$M_i$ est l'intersection d'une droite et d'un cercle} passant par des points dont les coordonnées sont des nombres constructibles de $K_i$ et de rayon un nombre constructible de $K_i$. Alors, les coordonnées $(x_i, y_i)$ de $M_i$ sont solution d'un système d'équations de la forme
      \[
        \begin{cases}
          ax + by = c \\
          (x-a')^2 + (y-b')^2 = c'
        \end{cases}
      \]
      avec $a, b, c, a', b', c' \in K_i$ par construction. Raisonnons selon la nullité de $a$.
      \begin{itemize}
        \item \uline{Si $a \neq 0$}, la première équation donne
        \[ x = -\frac{by+c}{a} \]
        et en réinjectant dans la deuxième équation, on obtient que $y_i$ est racine d'un polynôme de degré $2$. Ainsi, $[K_i(y_i) : K_i] \leq 2$. Puisque $x_i = -\frac{by_i+c}{a} \in K_i(y_i)$, on a bien $[K_{i+1}:K_i] \leq 2$.
        \item \uline{Si $a = 0$}, alors $y_i = \frac{c}{b} \in K_i$ (on ne peut pas avoir $b = 0$ dans ce cas). Or, cette fois-ci c'est $x_i$ qui est racine d'un polynôme de degré $2$. On peut conclure de la même manière que ci-dessus.
      \end{itemize}
      \item \uline{$M_i$ est l'intersection de deux cercles} passant par des points dont les coordonnées sont des nombres constructibles de $K_i$ et de rayon un nombre constructible de $K_i$. Alors, les coordonnées $(x_i, y_i)$ de $M_i$ sont solution d'un système d'équations de la forme
      \[
      \begin{cases}
        (x-a)^2 + (y-b)^2 = c \\
        (x-a')^2 + (y-b')^2 = c'
      \end{cases}
      \]
      avec $a, b, c, a', b', c' \in K_i$ par construction. On soustrait la deuxième équation à la première, pour obtenir le système équivalent :
      \[
        \begin{cases}
          -2(a-a')x -2(b-b')y = c-c' - (a^2-a'^2) - (b^2-b'^2) \\
          (x-a')^2 + (y-b')^2 = c'
        \end{cases}
      \]
      ce qui nous ramène au cas précédent.
    \end{itemize}
    Il suffit alors d'extraire de la suite $(K_0, \dots, K_m)$ une sous-suite $(L_0, \dots, L_p)$ strictement croissante (au sens de l'inclusion) en ne conservant dans la suite initiale que les corps extension quadratique du précédent (avec $L_0 = K_0$ et $L_p = K_n$). On obtient une suite de sous-corps de $\mathbb{R}$ (par le \cref{theoreme-de-wantzel-2}) qui remplit les trois conditions annoncées.
    \newpar
    Réciproquement, supposons l'existence d'une suite $(L_0, \dots, L_p)$ de sous-corps de $\mathbb{R}$ répondant aux trois conditions de l'énoncé. Montrons par récurrence que
    \[ \forall j \in \llbracket 0, p \rrbracket, \, L_j \subseteq \mathbb{E} \]
    \begin{itemize}
      \item \uline{Initialisation :} $L_0 = \mathbb{Q}$ : cela résulte du \cref{theoreme-de-wantzel-1}.
      \item \uline{Hérédité :} Supposons $L_j \subseteq \mathbb{E}$ pour $j \in \llbracket 0, p-1 \rrbracket$. Soit $x \in L_{j+1}$. Comme, par hypothèse,
      \[ [L_{j+1}:L_j] = 2 \]
      la famille $(1,x,x^2)$ est $L_j$-liée :
      \[ \exists a, b, c \in L_j \text{ non tous nuls tels que } ax^2 + bx + c = 0 \]
      \begin{itemize}
        \item \uline{Si $a = 0$}, alors, $x = -\frac{c}{b} \in L_j$. Donc $x \in \mathbb{E}$.
        \item \uline{Si $a \neq 0$}, alors, $x = \frac{1}{2a}(-b \pm \sqrt{b^2 - 4ac})$. Donc, comme $\mathbb{E}$ est un sous-corps de $\mathbb{R}$ stable par racine carrée (cf. \cref{theoreme-de-wantzel-2}), $x \in \mathbb{E}$.
      \end{itemize}
      Ainsi, $L_{j+1} \subseteq \mathbb{E}$. En conclusion, $L_p \subseteq \mathbb{E}$, donc $\alpha$ est constructible.
    \end{itemize}
  \end{proof}

  La réciproque et la conclusion du sens direct du théorème sont mieux rédigées dans \cite{[GOZ]}, à mon avis.

  \reference[GOZ]{52}

  \begin{corollary}
    \label{theoreme-de-wantzel-3}
    Si $\alpha \in \mathbb{R}$ est constructible, il existe $e \in \mathbb{N}$ tel $[\mathbb{Q}(\alpha):\mathbb{Q}] = 2^e$.
  \end{corollary}

  \begin{proof}
    Soit $\alpha \in \mathbb{E}$. D'après le théorème précédent, il existe une suite finie $(L_0, \dots, L_p)$ de sous-corps de $\mathbb{R}$ vérifiant :
    \begin{enumerate}[label=(\roman*)]
      \item $L_0 = \mathbb{Q}$.
      \item $\forall i \in \llbracket 0, p-1 \rrbracket$, $L_{i+1}$ est une extension quadratique (de degré $2$) de $L_i$.
      \item $\alpha \in L_p$.
    \end{enumerate}
    Par le théorème de la base télescopique,
    \[ [L_p : \mathbb{Q}] = 2^p \]
    et par ce même théorème,
    \[ [L_p : \mathbb{Q}] = [L_p : \mathbb{Q}(\alpha)] [\mathbb{Q}(\alpha) : \mathbb{Q}] \]
    et en particulier, $[\mathbb{Q}(\alpha) : \mathbb{Q}]$ est un diviseur de $2^p$ : ce qu'on voulait.
  \end{proof}

  \begin{application}[Duplication du cube]
    Soit un cube de volume $\mathcal{V}$ dont on suppose son arête $a$ constructible. Il est impossible de dessiner, à la règle et au compas, l'arête d'un cube de volume $2\mathcal{V}$.
  \end{application}

  \begin{proof}
    On a $\mathcal{V} = a^3$ et donc $ 2\mathcal{V} = 2a^3$. L'arête d'un cube est la racine cubique de son volume. Il faut donc construire le nombre
    \[ \sqrt[3]{2a^3} = a\sqrt[3]{2} \]
    Comme $a$ est constructible, ceci revient à construire le nombre
    \[ \alpha = \sqrt[3]{2} \]
    Le polynôme $P = X^3 - 2$ est irréductible sur $\mathbb{Q}$ (par le critère d'Eisenstein) et annule $\alpha$ : c'est son polynôme minimal sur $\mathbb{Q}$. On a ainsi
    \[ [\mathbb{Q}(\alpha) : \mathbb{Q}] = 3 \]
    donc $\alpha$ n'est pas constructible par le \cref{theoreme-de-wantzel-3}.
  \end{proof}
  %</content>
\end{document}
