\documentclass[12pt, a4paper]{report}

% LuaLaTeX :

\RequirePackage{iftex}
\RequireLuaTeX

% Packages :

\usepackage[french]{babel}
%\usepackage[utf8]{inputenc}
%\usepackage[T1]{fontenc}
\usepackage[pdfencoding=auto, pdfauthor={Hugo Delaunay}, pdfsubject={Mathématiques}, pdfcreator={agreg.skyost.eu}]{hyperref}
\usepackage{amsmath}
\usepackage{amsthm}
%\usepackage{amssymb}
\renewcommand{\proofname}{Solution}
\usepackage{stmaryrd}
\usepackage{tikz}
\usepackage{tkz-euclide}
\usepackage{fontspec}
\defaultfontfeatures[Erewhon]{FontFace = {bx}{n}{Erewhon-Bold.otf}}
\usepackage{fourier-otf}
\usepackage[nobottomtitles*]{titlesec}
\usepackage{fancyhdr}
\usepackage{listings}
\usepackage{catchfilebetweentags}
\usepackage[french, capitalise, noabbrev]{cleveref}
\usepackage[fit, breakall]{truncate}
\usepackage[top=2.5cm, right=2cm, bottom=2.5cm, left=2cm]{geometry}
\usepackage{enumitem}
\usepackage{tablists} %Pour faire 1)  2) 3)
\usepackage{tocloft}
\usepackage{microtype}
%\usepackage{mdframed}
%\usepackage{thmtools}
\usepackage{xcolor}
\usepackage{tabularx}
\usepackage{xltabular}
\usepackage{aligned-overset}
\usepackage[subpreambles=true]{standalone}
\usepackage{environ}
\usepackage[normalem]{ulem}
\usepackage{multicol}
 \usepackage{variations}
\usepackage{array}% Pour faire des tableaux
\usepackage{etoolbox}
\usepackage{setspace}
\usepackage[bibstyle=reading, citestyle=draft]{biblatex}
\usepackage{xpatch}
\usepackage[many, breakable]{tcolorbox}
\usepackage[backgroundcolor=white, bordercolor=white, textsize=scriptsize]{todonotes}
\usepackage{luacode}
\usepackage{float}
\usepackage{needspace}


% Police :

\setmathfont{Erewhon Math}

% Tikz :

\usetikzlibrary{calc}
\usetikzlibrary{3d}

% Longueurs :

\setlength{\parindent}{0pt}
\setlength{\headheight}{15pt}
\setlength{\fboxsep}{0pt}
\titlespacing*{\chapter}{0pt}{-20pt}{10pt}
\setlength{\marginparwidth}{1.5cm}
\setstretch{1.1}

% Métadonnées :

\author{agreg.skyost.eu}
\date{\today}

% Titres :

\setcounter{secnumdepth}{3}

\renewcommand{\thechapter}{\Roman{chapter}}
\renewcommand{\thesubsection}{\Roman{subsection}}
\renewcommand{\thesubsubsection}{\arabic{subsubsection}}
\renewcommand{\theparagraph}{\alph{paragraph}}

\titleformat{\chapter}{\huge\bfseries}{\thechapter}{20pt}{\huge\bfseries}
\titleformat*{\section}{\LARGE\bfseries}
\titleformat{\subsection}{\Large\bfseries}{\thesubsection \, - \,}{0pt}{\Large\bfseries}
\titleformat{\subsubsection}{\large\bfseries}{\thesubsubsection. \,}{0pt}{\large\bfseries}
\titleformat{\paragraph}{\bfseries}{\theparagraph. \,}{0pt}{\bfseries}

\setcounter{secnumdepth}{4}

% Table des matières :

\renewcommand{\cftsecleader}{\cftdotfill{\cftdotsep}}
\addtolength{\cftsecnumwidth}{10pt}

% Redéfinition des commandes :

\renewcommand*\thesection{\arabic{section}}
\renewcommand{\ker}{\mathrm{Ker}}

% Nouvelles commandes :

\newcommand{\website}{http://sencoursdemaths.com}

\newcommand{\tr}[1]{\mathstrut ^t #1}
\newcommand{\im}{\mathrm{Im}}
\newcommand{\rang}{\operatorname{rang}}
\newcommand{\trace}{\operatorname{trace}}
\newcommand{\id}{\operatorname{id}}
\newcommand{\stab}{\operatorname{Stab}}
\newcommand{\paren}[1]{\left(#1\right)}
\newcommand{\accol}[1]{\left\{#1\right\}}
\newcommand{\croch}[1]{\left[ #1 \right]}
\newcommand{\Grdcroch}[1]{\Bigl[ #1 \Bigr]}
\newcommand{\grdcroch}[1]{\bigl[ #1 \bigr]}
\newcommand{\abs}[1]{\left\lvert #1 \right\rvert}
\newcommand{\limi}[3]{\displaystyle \lim_{#1\to #2}#3}
\newcommand{\pinf}{+\infty}
\newcommand{\minf}{-\infty}
%%%%%%%%%%%%%% ENSEMBLES %%%%%%%%%%%%%%%%%
\newcommand{\ensemblenombre}[1]{\mathbb{#1}}
\newcommand{\Nn}{\ensemblenombre{N}}
\newcommand{\Zz}{\ensemblenombre{Z}}
\newcommand{\Qq}{\ensemblenombre{Q}}
\newcommand{\Qqp}{\Qq^+}
\newcommand{\Rr}{\ensemblenombre{R}}
\newcommand{\Cc}{\ensemblenombre{C}}
\newcommand{\Nne}{\Nn^*}
\newcommand{\Zze}{\Zz^*}
\newcommand{\Zzn}{\Zz^-}
\newcommand{\Qqe}{\Qq^*}
\newcommand{\Rre}{\Rr^*}
\newcommand{\Rrp}{\Rr_+}
\newcommand{\Rrm}{\Rr_-}
\newcommand{\Rrep}{\Rr_+^*}
\newcommand{\Rrem}{\Rr_-^*}
\newcommand{\Cce}{\Cc^*}
%%%%%%%%%%%%%%  INTERVALLES %%%%%%%%%%%%%%%%%
\newcommand{\intff}[2]{\left[#1\;,\; #2\right]  }
\newcommand{\intof}[2]{\left]#1 \;, \;#2\right]  }
\newcommand{\intfo}[2]{\left[#1 \;,\; #2\right[  }
\newcommand{\intoo}[2]{\left]#1 \;,\; #2\right[  }



\providecommand{\newpar}{\\[\medskipamount]}

\newcommand{\annexessection}{%
  \newpage%
  \subsection*{Annexes}%
}

\providecommand{\lesson}[3]{%
  \title{#3}%
  \hypersetup{pdftitle={#2 : #3}}%
  \setcounter{section}{\numexpr #2 - 1}%
  \section{#3}%
  \fancyhead[R]{\truncate{0.73\textwidth}{#2 : #3}}%
}

\providecommand{\development}[3]{%
  \title{#3}%
  \hypersetup{pdftitle={#3}}%
  \section*{#3}%
  \fancyhead[R]{\truncate{0.73\textwidth}{#3}}%
}

\providecommand{\sheet}[3]{\development{#1}{#2}{#3}}

\providecommand{\ranking}[1]{%
  \title{Terminale #1}%
  \hypersetup{pdftitle={Terminale #1}}%
  \section*{Terminale #1}%
  \fancyhead[R]{\truncate{0.73\textwidth}{Terminale #1}}%
}

\providecommand{\summary}[1]{%
  \textit{#1}%
  \par%
  \medskip%
}

\tikzset{notestyleraw/.append style={inner sep=0pt, rounded corners=0pt, align=center}}

%\newcommand{\booklink}[1]{\website/bibliographie\##1}
\newcounter{reference}
\newcommand{\previousreference}{}
\providecommand{\reference}[2][]{%
  \needspace{20pt}%
  \notblank{#1}{
    \needspace{20pt}%
    \renewcommand{\previousreference}{#1}%
    \stepcounter{reference}%
    \label{reference-\previousreference-\thereference}%
  }{}%
  \todo[noline]{%
    \protect\vspace{20pt}%
    \protect\par%
    \protect\notblank{#1}{\cite{[\previousreference]}\\}{}%
    \protect\hyperref[reference-\previousreference-\thereference]{p. #2}%
  }%
}

\definecolor{devcolor}{HTML}{00695c}
\providecommand{\dev}[1]{%
  \reversemarginpar%
  \todo[noline]{
    \protect\vspace{20pt}%
    \protect\par%
    \bfseries\color{devcolor}\href{\website/developpements/#1}{[DEV]}
  }%
  \normalmarginpar%
}

% En-têtes :

\pagestyle{fancy}
\fancyhead[L]{\truncate{0.23\textwidth}{\thepage}}
\fancyfoot[C]{\scriptsize \href{\website}{\texttt{http://sencoursdemaths.com}}}

% Couleurs :

\definecolor{property}{HTML}{ffeb3b}
\definecolor{proposition}{HTML}{ffc107}
\definecolor{lemma}{HTML}{ff9800}
\definecolor{theorem}{HTML}{f44336}
\definecolor{corollary}{HTML}{e91e63}
\definecolor{definition}{HTML}{673ab7}
\definecolor{notation}{HTML}{9c27b0}
\definecolor{example}{HTML}{00bcd4}
\definecolor{cexample}{HTML}{795548}
\definecolor{application}{HTML}{009688}
\definecolor{remark}{HTML}{3f51b5}
\definecolor{algorithm}{HTML}{607d8b}
\definecolor{proof}{HTML}{e1f5fe}
\definecolor{exercice}{HTML}{e1f5fe}

% Théorèmes :

\theoremstyle{definition}
\newtheorem{theorem}{Théorème}

\newtheorem{property}[theorem]{Propriété}
\newtheorem{proposition}[theorem]{Proposition}
\newtheorem{lemma}[theorem]{Activité d'introduction}
\newtheorem{corollary}[theorem]{Conséquence}

\newtheorem{definition}[theorem]{Définition}
\newtheorem{notation}[theorem]{Notation}

\newtheorem{example}[theorem]{Exemple}
\newtheorem{cexample}[theorem]{Contre-exemple}
\newtheorem{application}[theorem]{Application}

\newtheorem{algorithm}[theorem]{Algorithme}
\newtheorem{exercice}[theorem]{Exercice}

\theoremstyle{remark}
\newtheorem{remark}[theorem]{Remarque}




\counterwithin*{theorem}{section}

\newcommand{\applystyletotheorem}[1]{
  \tcolorboxenvironment{#1}{
    enhanced,
    breakable,
    colback=#1!8!white,
    %right=0pt,
    %top=8pt,
    %bottom=8pt,
    boxrule=0pt,
    frame hidden,
    sharp corners,
    enhanced,borderline west={4pt}{0pt}{#1},
    %interior hidden,
    sharp corners,
    after=\par,
  }
}

\applystyletotheorem{property}
\applystyletotheorem{proposition}
\applystyletotheorem{lemma}
\applystyletotheorem{theorem}
\applystyletotheorem{corollary}
\applystyletotheorem{definition}
\applystyletotheorem{notation}
\applystyletotheorem{example}
\applystyletotheorem{cexample}
\applystyletotheorem{application}
\applystyletotheorem{remark}
%\applystyletotheorem{proof}
\applystyletotheorem{algorithm}
\applystyletotheorem{exercice}

% Environnements :

\NewEnviron{whitetabularx}[1]{%
  \renewcommand{\arraystretch}{2.5}
  \colorbox{white}{%
    \begin{tabularx}{\textwidth}{#1}%
      \BODY%
    \end{tabularx}%
  }%
}

% Maths :

\DeclareFontEncoding{FMS}{}{}
\DeclareFontSubstitution{FMS}{futm}{m}{n}
\DeclareFontEncoding{FMX}{}{}
\DeclareFontSubstitution{FMX}{futm}{m}{n}
\DeclareSymbolFont{fouriersymbols}{FMS}{futm}{m}{n}
\DeclareSymbolFont{fourierlargesymbols}{FMX}{futm}{m}{n}
\DeclareMathDelimiter{\VERT}{\mathord}{fouriersymbols}{152}{fourierlargesymbols}{147}

% Code :

\definecolor{greencode}{rgb}{0,0.6,0}
\definecolor{graycode}{rgb}{0.5,0.5,0.5}
\definecolor{mauvecode}{rgb}{0.58,0,0.82}
\definecolor{bluecode}{HTML}{1976d2}
\lstset{
  basicstyle=\footnotesize\ttfamily,
  breakatwhitespace=false,
  breaklines=true,
  %captionpos=b,
  commentstyle=\color{greencode},
  deletekeywords={...},
  escapeinside={\%*}{*)},
  extendedchars=true,
  frame=none,
  keepspaces=true,
  keywordstyle=\color{bluecode},
  language=Python,
  otherkeywords={*,...},
  numbers=left,
  numbersep=5pt,
  numberstyle=\tiny\color{graycode},
  rulecolor=\color{black},
  showspaces=false,
  showstringspaces=false,
  showtabs=false,
  stepnumber=2,
  stringstyle=\color{mauvecode},
  tabsize=2,
  %texcl=true,
  xleftmargin=10pt,
  %title=\lstname
}

\newcommand{\codedirectory}{}
\newcommand{\inputalgorithm}[1]{%
  \begin{algorithm}%
    \strut%
    \lstinputlisting{\codedirectory#1}%
  \end{algorithm}%
}




% Bibliographie :

%\addbibresource{\bibliographypath}%
\defbibheading{bibliography}[\bibname]{\section*{#1}}
\renewbibmacro*{entryhead:full}{\printfield{labeltitle}}%
\DeclareFieldFormat{url}{\newline\footnotesize\url{#1}}%

\AtEndDocument{%
  \newpage%
  \pagestyle{empty}%
  \printbibliography%
}


\begin{document}
  %<*content>
  \development{algebra}{surjectivite-de-l-exponentielle}{\texorpdfstring{$\exp : \mathcal{M}_n(\mathbb{R}) \rightarrow \mathrm{GL}_n(\mathbb{R})$}{exp : Mn(R) -> GLn(R)} est surjective}

  \summary{Dans ce développement, on démontre que l'exponentielle de matrices est surjective en utilisant des théorèmes d'analyse.}

  \reference[I-P]{396}

  \begin{lemma}
    \label{surjectivite-de-l-exponentielle-1}
    Soit $M \in \mathrm{GL}_n(\mathbb{C})$. Alors $M^{-1} \in \mathbb{C}[X]$ (ie. $M^{-1}$ est un polynôme en $M$).
  \end{lemma}

  \begin{proof}
    D'après le théorème de Cayley-Hamilton, $\chi_M(M) = 0$. Or, en notant $\chi_M = \sum_{k=0}^n a_k X^k$, on a $a_0 = (-1)^n \det(M)$, d'où
    \[ 0 = M^n + \dots + a_1 M + (-1)^n \det(M) I_n \]
    En notant $Q = X^{n-1} + a_{n-1}X^{n-2} + \dots + a_2 X + a_1$, on en déduit que $(-1)^{n+1} \det(M) I_n = Q(M)M$. D'où
    \[ M^{-1} = \frac{(-1)^{n+1}}{\det(M)} Q(M) \in \mathbb{C}[M] \]
    ce qu'il fallait démontrer.
  \end{proof}

  \begin{lemma}
    \label{surjectivite-de-l-exponentielle-2}
    Soit $M \in \mathcal{M}_n(\mathbb{C})$. Alors, $\exp(M) \in \mathrm{GL}_n(\mathbb{C})$.
  \end{lemma}

  \begin{proof}
    $M$ et $-M$ commutent, donc
    \[ \exp(M)\exp(-M) = \exp(M-M) = I_n = \exp(-M)\exp(M) \]
    Ainsi $\exp(M)$ est inversible, d'inverse $\exp(-M)$.
  \end{proof}
  
    \begin{notation}
    Soit $C \in \mathcal{M}_n(\mathbb{C})$. On note $\mathbb{C}[C]^* = \mathbb{C}[C] \, \cap \, \mathrm{GL}_n(\mathbb{C})$.
  \end{notation}
  
  \begin{lemma}
    \label{surjectivite-de-l-exponentielle-3}
    Soit $C \in \mathcal{M}_n(\mathbb{C})$. $\mathbb{C}[C]^*$ est un sous-groupe de $\mathrm{GL}_n(\mathbb{C})$.
  \end{lemma}
  
  \begin{proof}
    \begin{itemize}
      \item $I_n \in \mathbb{C}[C]$ et $I_n \in \mathrm{GL}_n(\mathbb{C})$, donc $I_n \in \mathbb{C}[C]^*$.
      \item Soit $M \in \mathbb{C}[C]^*$. Comme $M \in \mathrm{GL}_n(\mathbb{C})$, $M^{-1}$ existe, est inversible, et, par le \cref{surjectivite-de-l-exponentielle-1}, $M^{-1} \in \mathbb{C}[C]$.
      \item Enfin, $\mathbb{C}[C]^*$ est clairement stable par multiplication.
    \end{itemize}
  \end{proof}

  \begin{lemma}
    \label{surjectivite-de-l-exponentielle-4}
    $\exp$ est différentiable en $0$ et,
    \[ \mathrm{d}\exp_0 = I_n \]
  \end{lemma}

  \begin{proof}
    Soit $H \in \mathcal{M}_n(\mathbb{C})$.
    \begin{align*}
      \exp(0+H) - \exp(H) &= \sum_{k=0}^{+\infty} \frac{H^k}{k!} \\
      &= I_n + H + \sum_{k=2}^{+\infty} \frac{H^k}{k!} \\
    \end{align*}
    Soit $\Vert . \Vert$ une norme d'algèbre sur $\mathcal{M}_n(\mathbb{C})$. On a :
    \begin{align*}
      \left\Vert \sum_{k=2}^{+\infty} \frac{H^k}{k!} \right\Vert &\leq \sum_{k=2}^{+\infty} \left\Vert \frac{H^k}{k!} \right\Vert \\
      &\leq \sum_{k=2}^{+\infty} \frac{\Vert H \Vert^k}{k!} \\
      &= e^{\Vert H \Vert} - \Vert H \Vert - 1
    \end{align*}
    En effectuant un développement limité de l'exponentielle réelle à l'origine, on obtient bien $\left\Vert \sum_{k=2}^{+\infty} \frac{H^k}{k!} \right\Vert = o(\Vert H \Vert)$.
  \end{proof}

  \begin{theorem}
    \label{surjectivite-de-l-exponentielle-5}
    $\exp : \mathcal{M}_n(\mathbb{C}) \rightarrow \mathrm{GL}_n(\mathbb{C})$ est surjective.
  \end{theorem}

  \begin{proof}
    Fixons $C \in \mathcal{M}_n(\mathbb{C})$ pour le reste de la démonstration. Comme $\mathbb{C}[C]$ est un sous-espace vectoriel de l'espace $\mathcal{M}_n(\mathbb{C})$, il est de dimension finie et est donc fermé. En particulier, $\exp(C) \in \mathbb{C}[C]$. Le \cref{surjectivite-de-l-exponentielle-2} combiné au \cref{surjectivite-de-l-exponentielle-3}, nous dit que $\exp : \mathbb{C}[C] \rightarrow \mathbb{C}[C]^*$ est bien définie. Il s'agit de plus d'un morphisme de groupes. En effet, $\forall A, B \in \mathbb{C}[C]$, on a $AB=BA$, d'où $\exp(A)\exp(B) = \exp(A+B) = \exp(B)\exp(A)$.
    \newpar
    Montrons que $\mathbb{C}[C]^*$ est un ouvert connexe de $\mathbb{C}[C]$. Notons qu'il s'agit bien d'un ouvert de $\mathbb{C}[C]$, car c'est l'intersection de $\mathbb{C}[C]$ avec $\mathrm{GL}_n(\mathbb{C})$ qui est ouvert dans $\mathcal{M}_n(\mathbb{C})$. Ensuite, soient $A, B \in \mathbb{C}[C]^*$. On pose
    \[ P = \det((1-X)A+XB) \]
    $P$ ne s'annule ni en $0$, ni en $1$ par inversibilité de $A$ et $B$. $P$ a un nombre fini de racines car n'est pas nul : on peut trouver une fonction continue $\gamma : [0,1] \rightarrow \mathbb{C}$ qui évite ces racines. Ainsi,
    \[ \forall t \in [0,1], \, (1-\gamma(t))A + \gamma(t)B \in \mathbb{C}[C]^* \]
    donc $\mathbb{C}[C]^*$ est connexe par arcs, donc est en particulier connexe.
    \newpar
    Il s'agit maintenant de montrer que $\exp(\mathbb{C}[C])$ est un ouvert-fermé de $\mathbb{C}[C]^*$. Commençons par montrer qu'il est ouvert en montrant qu'il contient un voisinage de chacun de ses points. Par le théorème d'inversion locale appliqué à $\exp : \mathbb{C}[C] \rightarrow \mathbb{C}[C]$ (qui est bien $\mathcal{C}^1$ sur l'espace de Banach $\mathbb{C}[C]$ et, par le \cref{surjectivite-de-l-exponentielle-4}, $\det(\mathrm{d}\exp_0) \neq 0$) : il existe $U$ un voisinage de $0$ dans $\mathbb{C}(C)$ et un ouvert $V$ de $\mathbb{C}(C)$ contenant $\exp(0) = I_n$ tels que $\exp : U \rightarrow V$ soit un difféomorphisme de classe $\mathcal{C}^1$. Soit $A \in \mathbb{C}[C]$. Posons
    \[
      f_A :
      \begin{array}{ccc}
        \mathbb{C}[C] &\rightarrow& \mathbb{C}[C] \\
        M &\mapsto& \exp(A)^{-1}M
      \end{array}
    \]
    et montrons que $\exp(A)V = f^{-1}(V)$.
    Pour tout $B \in V$, $f_A(\exp(A)B) = \exp(A)^{-1}(\exp(A)B) = B \in V$, donc $\exp(A)V \subseteq f^{-1}(V)$.
    \newpar
    Soit $B \in f^{-1}(V)$, alors $f_A(B) \in V$. Or, $f_A(B) = \exp(A)^{-1}B$, donc $B = \exp(A)f_A(B) \in \exp(A)V$. On en déduit que $\exp(A)V = f^{-1}(V)$ et que $\exp(A)V$ est un ouvert par continuité de $f$.
    \newpar
    Comme $V$ contient $I_n$, $\exp(A)V$ est un voisinage de $\exp(A)$. Or, $\exp(A)V$ est inclus dans $\mathbb{C}[C]$ car pour tout $B \in V$, il existe $M \in \mathbb{C}[C]$ tel que $\exp(M)=B$. Ainsi,
    \[ \exp(A)B = \exp(A)\exp(M) = \exp(A+M) \in \exp(\mathbb{C}[C]) \]
    On en déduit que $\exp(\mathbb{C}[C])$ est un ouvert.
    \newpar
    Posons maintenant $O = \mathbb{C}[C]^* \setminus \exp(\mathbb{C}[C])$ et montrons que
    \[ O = \bigcup_{A \in O} A\exp(\mathbb{C}[C]) \tag{$*$} \]
    Soient $A \in O$ et $B \in \exp(\mathbb{C}[C])$. Alors $AB \in \mathbb{C}[C]^*$. Supposons par l'absurde que $AB \in \exp(\mathbb{C}[C])$. Il existe donc $M \in \exp(\mathbb{C}[C])$ tel que $AB = M$ et $A=MB^{-1}$. Comme $\exp(\mathbb{C}[C])$ est un groupe multiplicatif, alors $A \in \exp(\mathbb{C}[C])$ : absurde. On conclut que
    \[ \bigcup_{A \in O} A\exp(\mathbb{C}[C]) \subseteq O \]
    Réciproquement, supposons que $M \in O$. Comme $I_n \in \exp(\mathbb{C}[C])$, alors $M \in M\exp(\mathbb{C}[C])$. On en déduit $(*)$, ainsi que la fermeture de $\exp(\mathbb{C}[M])$ par passage au complémentaire.
    \newpar
    $\exp(\mathbb{C}[M])$ est un ouvert fermé non vide (car contient $I_n$) de $\mathbb{C}[M]^*$, alors $\exp(\mathbb{C}[M]) = \mathbb{C}[M]^*$. Pour conclure, si $M \in \mathrm{GL}_n(\mathbb{C})$, alors $M \in \mathbb{C}[M]$ et donc $M \in \mathbb{C}[M]^*$. Ainsi, $M \in \exp(\mathbb{C}[M])$, et $\exp$ est bien surjective.
  \end{proof}

  \begin{application}
    $\exp(\mathcal{M}_n(\mathbb{R})) = \mathrm{GL}_n(\mathbb{R})^2$, où $\mathrm{GL}_n(\mathbb{R})^2$ désigne les carrés de $\mathrm{GL}_n(\mathbb{R})$.
  \end{application}

  \begin{proof}
    Soit $M \in \mathcal{M}_n(\mathbb{R})$. Alors,
    \[ \exp(M) = \exp \left( \frac{M}{2} \right)^2 \]
    d'où $\exp(\mathcal{M}_n(\mathbb{R})) \subseteq \mathrm{GL}_n(\mathbb{R})^2$. Réciproquement, soit $A \in \mathrm{GL}_n(\mathbb{R})^2$. Posons $B = A^2$. D'après le \cref{surjectivite-de-l-exponentielle-5},
    \[ \exists P \in \mathbb{C}[X] \text{ telle que } A = \exp(P(A)) \]
    Comme $A$ est une matrice réelle, alors en passant au conjugué, on obtient $A = \exp(\overline{P}(A))$. Ainsi,
    \[ B = A^2 = \exp((P + \overline{P})(A)) \in \exp(\mathcal{M}_n(\mathbb{R})) \]
    d'où $\mathrm{GL}_n(\mathbb{R})^2 \subseteq \exp(\mathcal{M}_n(\mathbb{R}))$.
  \end{proof}
  %</content>
\end{document}
