\documentclass[12pt, a4paper]{report}

% LuaLaTeX :

\RequirePackage{iftex}
\RequireLuaTeX

% Packages :

\usepackage[french]{babel}
%\usepackage[utf8]{inputenc}
%\usepackage[T1]{fontenc}
\usepackage[pdfencoding=auto, pdfauthor={Hugo Delaunay}, pdfsubject={Mathématiques}, pdfcreator={agreg.skyost.eu}]{hyperref}
\usepackage{amsmath}
\usepackage{amsthm}
%\usepackage{amssymb}
\renewcommand{\proofname}{Solution}
\usepackage{stmaryrd}
\usepackage{tikz}
\usepackage{tkz-euclide}
\usepackage{fontspec}
\defaultfontfeatures[Erewhon]{FontFace = {bx}{n}{Erewhon-Bold.otf}}
\usepackage{fourier-otf}
\usepackage[nobottomtitles*]{titlesec}
\usepackage{fancyhdr}
\usepackage{listings}
\usepackage{catchfilebetweentags}
\usepackage[french, capitalise, noabbrev]{cleveref}
\usepackage[fit, breakall]{truncate}
\usepackage[top=2.5cm, right=2cm, bottom=2.5cm, left=2cm]{geometry}
\usepackage{enumitem}
\usepackage{tablists} %Pour faire 1)  2) 3)
\usepackage{tocloft}
\usepackage{microtype}
%\usepackage{mdframed}
%\usepackage{thmtools}
\usepackage{xcolor}
\usepackage{tabularx}
\usepackage{xltabular}
\usepackage{aligned-overset}
\usepackage[subpreambles=true]{standalone}
\usepackage{environ}
\usepackage[normalem]{ulem}
\usepackage{multicol}
 \usepackage{variations}
\usepackage{array}% Pour faire des tableaux
\usepackage{etoolbox}
\usepackage{setspace}
\usepackage[bibstyle=reading, citestyle=draft]{biblatex}
\usepackage{xpatch}
\usepackage[many, breakable]{tcolorbox}
\usepackage[backgroundcolor=white, bordercolor=white, textsize=scriptsize]{todonotes}
\usepackage{luacode}
\usepackage{float}
\usepackage{needspace}


% Police :

\setmathfont{Erewhon Math}

% Tikz :

\usetikzlibrary{calc}
\usetikzlibrary{3d}

% Longueurs :

\setlength{\parindent}{0pt}
\setlength{\headheight}{15pt}
\setlength{\fboxsep}{0pt}
\titlespacing*{\chapter}{0pt}{-20pt}{10pt}
\setlength{\marginparwidth}{1.5cm}
\setstretch{1.1}

% Métadonnées :

\author{agreg.skyost.eu}
\date{\today}

% Titres :

\setcounter{secnumdepth}{3}

\renewcommand{\thechapter}{\Roman{chapter}}
\renewcommand{\thesubsection}{\Roman{subsection}}
\renewcommand{\thesubsubsection}{\arabic{subsubsection}}
\renewcommand{\theparagraph}{\alph{paragraph}}

\titleformat{\chapter}{\huge\bfseries}{\thechapter}{20pt}{\huge\bfseries}
\titleformat*{\section}{\LARGE\bfseries}
\titleformat{\subsection}{\Large\bfseries}{\thesubsection \, - \,}{0pt}{\Large\bfseries}
\titleformat{\subsubsection}{\large\bfseries}{\thesubsubsection. \,}{0pt}{\large\bfseries}
\titleformat{\paragraph}{\bfseries}{\theparagraph. \,}{0pt}{\bfseries}

\setcounter{secnumdepth}{4}

% Table des matières :

\renewcommand{\cftsecleader}{\cftdotfill{\cftdotsep}}
\addtolength{\cftsecnumwidth}{10pt}

% Redéfinition des commandes :

\renewcommand*\thesection{\arabic{section}}
\renewcommand{\ker}{\mathrm{Ker}}

% Nouvelles commandes :

\newcommand{\website}{http://sencoursdemaths.com}

\newcommand{\tr}[1]{\mathstrut ^t #1}
\newcommand{\im}{\mathrm{Im}}
\newcommand{\rang}{\operatorname{rang}}
\newcommand{\trace}{\operatorname{trace}}
\newcommand{\id}{\operatorname{id}}
\newcommand{\stab}{\operatorname{Stab}}
\newcommand{\paren}[1]{\left(#1\right)}
\newcommand{\accol}[1]{\left\{#1\right\}}
\newcommand{\croch}[1]{\left[ #1 \right]}
\newcommand{\Grdcroch}[1]{\Bigl[ #1 \Bigr]}
\newcommand{\grdcroch}[1]{\bigl[ #1 \bigr]}
\newcommand{\abs}[1]{\left\lvert #1 \right\rvert}
\newcommand{\limi}[3]{\displaystyle \lim_{#1\to #2}#3}
\newcommand{\pinf}{+\infty}
\newcommand{\minf}{-\infty}
%%%%%%%%%%%%%% ENSEMBLES %%%%%%%%%%%%%%%%%
\newcommand{\ensemblenombre}[1]{\mathbb{#1}}
\newcommand{\Nn}{\ensemblenombre{N}}
\newcommand{\Zz}{\ensemblenombre{Z}}
\newcommand{\Qq}{\ensemblenombre{Q}}
\newcommand{\Qqp}{\Qq^+}
\newcommand{\Rr}{\ensemblenombre{R}}
\newcommand{\Cc}{\ensemblenombre{C}}
\newcommand{\Nne}{\Nn^*}
\newcommand{\Zze}{\Zz^*}
\newcommand{\Zzn}{\Zz^-}
\newcommand{\Qqe}{\Qq^*}
\newcommand{\Rre}{\Rr^*}
\newcommand{\Rrp}{\Rr_+}
\newcommand{\Rrm}{\Rr_-}
\newcommand{\Rrep}{\Rr_+^*}
\newcommand{\Rrem}{\Rr_-^*}
\newcommand{\Cce}{\Cc^*}
%%%%%%%%%%%%%%  INTERVALLES %%%%%%%%%%%%%%%%%
\newcommand{\intff}[2]{\left[#1\;,\; #2\right]  }
\newcommand{\intof}[2]{\left]#1 \;, \;#2\right]  }
\newcommand{\intfo}[2]{\left[#1 \;,\; #2\right[  }
\newcommand{\intoo}[2]{\left]#1 \;,\; #2\right[  }



\providecommand{\newpar}{\\[\medskipamount]}

\newcommand{\annexessection}{%
  \newpage%
  \subsection*{Annexes}%
}

\providecommand{\lesson}[3]{%
  \title{#3}%
  \hypersetup{pdftitle={#2 : #3}}%
  \setcounter{section}{\numexpr #2 - 1}%
  \section{#3}%
  \fancyhead[R]{\truncate{0.73\textwidth}{#2 : #3}}%
}

\providecommand{\development}[3]{%
  \title{#3}%
  \hypersetup{pdftitle={#3}}%
  \section*{#3}%
  \fancyhead[R]{\truncate{0.73\textwidth}{#3}}%
}

\providecommand{\sheet}[3]{\development{#1}{#2}{#3}}

\providecommand{\ranking}[1]{%
  \title{Terminale #1}%
  \hypersetup{pdftitle={Terminale #1}}%
  \section*{Terminale #1}%
  \fancyhead[R]{\truncate{0.73\textwidth}{Terminale #1}}%
}

\providecommand{\summary}[1]{%
  \textit{#1}%
  \par%
  \medskip%
}

\tikzset{notestyleraw/.append style={inner sep=0pt, rounded corners=0pt, align=center}}

%\newcommand{\booklink}[1]{\website/bibliographie\##1}
\newcounter{reference}
\newcommand{\previousreference}{}
\providecommand{\reference}[2][]{%
  \needspace{20pt}%
  \notblank{#1}{
    \needspace{20pt}%
    \renewcommand{\previousreference}{#1}%
    \stepcounter{reference}%
    \label{reference-\previousreference-\thereference}%
  }{}%
  \todo[noline]{%
    \protect\vspace{20pt}%
    \protect\par%
    \protect\notblank{#1}{\cite{[\previousreference]}\\}{}%
    \protect\hyperref[reference-\previousreference-\thereference]{p. #2}%
  }%
}

\definecolor{devcolor}{HTML}{00695c}
\providecommand{\dev}[1]{%
  \reversemarginpar%
  \todo[noline]{
    \protect\vspace{20pt}%
    \protect\par%
    \bfseries\color{devcolor}\href{\website/developpements/#1}{[DEV]}
  }%
  \normalmarginpar%
}

% En-têtes :

\pagestyle{fancy}
\fancyhead[L]{\truncate{0.23\textwidth}{\thepage}}
\fancyfoot[C]{\scriptsize \href{\website}{\texttt{http://sencoursdemaths.com}}}

% Couleurs :

\definecolor{property}{HTML}{ffeb3b}
\definecolor{proposition}{HTML}{ffc107}
\definecolor{lemma}{HTML}{ff9800}
\definecolor{theorem}{HTML}{f44336}
\definecolor{corollary}{HTML}{e91e63}
\definecolor{definition}{HTML}{673ab7}
\definecolor{notation}{HTML}{9c27b0}
\definecolor{example}{HTML}{00bcd4}
\definecolor{cexample}{HTML}{795548}
\definecolor{application}{HTML}{009688}
\definecolor{remark}{HTML}{3f51b5}
\definecolor{algorithm}{HTML}{607d8b}
\definecolor{proof}{HTML}{e1f5fe}
\definecolor{exercice}{HTML}{e1f5fe}

% Théorèmes :

\theoremstyle{definition}
\newtheorem{theorem}{Théorème}

\newtheorem{property}[theorem]{Propriété}
\newtheorem{proposition}[theorem]{Proposition}
\newtheorem{lemma}[theorem]{Activité d'introduction}
\newtheorem{corollary}[theorem]{Conséquence}

\newtheorem{definition}[theorem]{Définition}
\newtheorem{notation}[theorem]{Notation}

\newtheorem{example}[theorem]{Exemple}
\newtheorem{cexample}[theorem]{Contre-exemple}
\newtheorem{application}[theorem]{Application}

\newtheorem{algorithm}[theorem]{Algorithme}
\newtheorem{exercice}[theorem]{Exercice}

\theoremstyle{remark}
\newtheorem{remark}[theorem]{Remarque}




\counterwithin*{theorem}{section}

\newcommand{\applystyletotheorem}[1]{
  \tcolorboxenvironment{#1}{
    enhanced,
    breakable,
    colback=#1!8!white,
    %right=0pt,
    %top=8pt,
    %bottom=8pt,
    boxrule=0pt,
    frame hidden,
    sharp corners,
    enhanced,borderline west={4pt}{0pt}{#1},
    %interior hidden,
    sharp corners,
    after=\par,
  }
}

\applystyletotheorem{property}
\applystyletotheorem{proposition}
\applystyletotheorem{lemma}
\applystyletotheorem{theorem}
\applystyletotheorem{corollary}
\applystyletotheorem{definition}
\applystyletotheorem{notation}
\applystyletotheorem{example}
\applystyletotheorem{cexample}
\applystyletotheorem{application}
\applystyletotheorem{remark}
%\applystyletotheorem{proof}
\applystyletotheorem{algorithm}
\applystyletotheorem{exercice}

% Environnements :

\NewEnviron{whitetabularx}[1]{%
  \renewcommand{\arraystretch}{2.5}
  \colorbox{white}{%
    \begin{tabularx}{\textwidth}{#1}%
      \BODY%
    \end{tabularx}%
  }%
}

% Maths :

\DeclareFontEncoding{FMS}{}{}
\DeclareFontSubstitution{FMS}{futm}{m}{n}
\DeclareFontEncoding{FMX}{}{}
\DeclareFontSubstitution{FMX}{futm}{m}{n}
\DeclareSymbolFont{fouriersymbols}{FMS}{futm}{m}{n}
\DeclareSymbolFont{fourierlargesymbols}{FMX}{futm}{m}{n}
\DeclareMathDelimiter{\VERT}{\mathord}{fouriersymbols}{152}{fourierlargesymbols}{147}

% Code :

\definecolor{greencode}{rgb}{0,0.6,0}
\definecolor{graycode}{rgb}{0.5,0.5,0.5}
\definecolor{mauvecode}{rgb}{0.58,0,0.82}
\definecolor{bluecode}{HTML}{1976d2}
\lstset{
  basicstyle=\footnotesize\ttfamily,
  breakatwhitespace=false,
  breaklines=true,
  %captionpos=b,
  commentstyle=\color{greencode},
  deletekeywords={...},
  escapeinside={\%*}{*)},
  extendedchars=true,
  frame=none,
  keepspaces=true,
  keywordstyle=\color{bluecode},
  language=Python,
  otherkeywords={*,...},
  numbers=left,
  numbersep=5pt,
  numberstyle=\tiny\color{graycode},
  rulecolor=\color{black},
  showspaces=false,
  showstringspaces=false,
  showtabs=false,
  stepnumber=2,
  stringstyle=\color{mauvecode},
  tabsize=2,
  %texcl=true,
  xleftmargin=10pt,
  %title=\lstname
}

\newcommand{\codedirectory}{}
\newcommand{\inputalgorithm}[1]{%
  \begin{algorithm}%
    \strut%
    \lstinputlisting{\codedirectory#1}%
  \end{algorithm}%
}





\begin{document}
  %<*content>
  \sheet{algebra}{chiffrement-rsa}{Chiffrement RSA}

  \summary{On commence par récapituler toute l'arithmétique des entiers, connue depuis la L1, et sans faire appel à la théorie des groupes. On détaille ensuite un exemple pratique de chiffrement RSA, et on explique les mathématiques se trouvant derrière à l'aide de la première partie.}

  \subsection{Arithmétique dans \texorpdfstring{$\mathbb{Z}$}{Z}}

  \subsubsection{Divisibilité dans \texorpdfstring{$\mathbb{Z}$}{Z}}

  \begin{definition}
    Soient $a$ et $b$ deux entiers relatifs. On dit que $b$ \textbf{divise} $a$ (ou que $a$ est un \textbf{multiple} de $b$) s'il existe $k \in \mathbb{Z}$ tel que $a = kb$. On note ceci par $b \mid a$.
  \end{definition}

  \begin{example}
    \begin{itemize}
      \item $6 = 2 \times 3$ donc $2$ et $3$ sont des diviseurs de $6$.
      Les diviseurs dans $\mathbb{N}$ de $6$ sont : $1$, $2$, $3$ et $6$.
      \item $-52 = (-4) \times 13$ donc $-4$, $4$, $-13$ et $13$ sont des diviseurs de $-52$. Les diviseurs dans $\mathbb{Z}$ de $-52$ sont : $-52$, $-26$, $-13$, $-4$, $-2$, $-1$, $1$, $2$, $4$, $13$, $26$ et $52$.
    \end{itemize}
  \end{example}

  \begin{proposition}
    \begin{enumerate}[label=(\roman*)]
      \item Tout entier relatif $b$ divise $0$ (car $0 = 0 \times b$).
      \item $1$ divise tout entier relatif $a$ (car $a = a \times 1$).
      \item Si $c \mid a$ et $c \mid b$ alors $c \mid (au + bv)$ pour tout $u$, $v \in \mathbb{Z}$.
    \end{enumerate}
  \end{proposition}

  \begin{proof}
    Montrons le dernier point : il existe $k$, $k'$ tels que $a = kc$ et $b = k'c$. Donc $ua + vb = ukc + vk'c = (uk + vk')c$. D'où $c \mid (au + bv)$.
  \end{proof}

  \begin{definition}
    Un nombre entier $p \geq 2$ est dit \textbf{premier} si ses seuls diviseurs positifs sont $1$ et lui-même.
  \end{definition}

  \begin{example}
    $2$, $3$, $5$, $7$, $11$ et $13$ sont des nombres premiers et il en existe une infinité.
  \end{example}

  \subsubsection{Division euclidienne}

  \begin{theorem}[Division euclidienne dans $\mathbb{Z}$]
    Soient $a \in \mathbb{N}$ et $b \in \mathbb{N}^*$. On appelle \textbf{division euclidienne} de $a$ par $b$, l'opération qui à $(a, b)$, associe le couple d'entiers $(q, r)$ tel que $a = bq + r$ où $0 \leq r < b$. Un tel couple existe forcément et est unique.
  \end{theorem}

  \begin{proof}
    Si $a < b$, alors il suffit de prendre $q = 0$ et $r = a$. Nous supposerons donc dans la suite $a \geq b$.
    \begin{itemize}
      \item \uline{Existence :} On note $S$ l'ensemble des entiers naturels $s$ qui s'écrivent $s = a - tb$ où $t \in \mathbb{N}$. Cet ensemble est non vide (car il contient $a$) et comme c'est un sous-ensemble de $\mathbb{N}$, il admet un plus petit élément $r = a - qb$. On a forcément $r < b$ (sinon $a-(q+1)b$ serait dans $S$ et serait plus petit que $r$). Donc $0 \leq a - qb < b$ et ce couple $(q, r)$ vérifie les conditions données par le théorème.
      \item \uline{Unicité :} On suppose qu'il existe un deuxième couple $(q', r')$ vérifiant les conditions du théorème. On a $a = bq + r = bq' + r'$, donc $b(q-q') = r-r'$. Comme $0 \leq r < b$ alors $-b < -r \leq 0$. De plus $0 \leq r' < b$, donc en additionnant les inégalités on a $-b < r' - r < b$. Comme $b \mid r - r'$ on a $r - r' = 0$ (i.e. $r = r'$) et donc $q = q'$. D'où $(q', r') = (q, r)$.
    \end{itemize}
  \end{proof}

  \begin{definition}
    En reprenant les notations du théorème, $a$ s'appelle le \textbf{dividende}, $b$ le \textbf{diviseur}, $q$ le \textbf{quotient} et $r$ le \textbf{reste} de la division euclidienne.
  \end{definition}

  \begin{remark}
    Il est possible d'étendre le principe de la division euclidienne aux entiers relatifs. La condition pour le reste $r$ devient alors $0 \leq r < |b|$.
  \end{remark}

  \begin{example}
    On souhaite effectuer la division euclidienne de $314$ par $7$. Détaillons étape par étape :
    \begin{itemize}
      \item On cherche combien de fois $7$ est contenu dans $31$ (cela ne sert à rien de commencer par $3$ car $3 < 7$). On a $4 \times 7 = 28$ et $5 \times 7 = 35$ donc, on écrit $4$ sous le diviseur et le reste $31 - 28 = 3$. Puis, on abaisse le chiffre des unités qui est $4$.
      \item On recommence : combien de fois $7$ est-il contenu dans $34$ ? Comme $4 \times 7 = 28$ et $5 \times 7 = 35$, $7$ est contenu $4$ fois dans $34$ et il reste $34 - 28 = 6$.
      \item Comme $6 < 7$, la division euclidienne est terminée : on a $314 = 7 \times 44 + 6$.
    \end{itemize}
  \end{example}

  Donnons, pour finir, une propriété qui nous sera utile dans la sous-section suivante.

  \begin{proposition}
    Soit $n \in \mathbb{N}^*$. Deux entiers relatifs $a$ et $b$ ont le même reste dans la division euclidienne par $n$ si et seulement si $a-b$ est un multiple de $n$.
  \end{proposition}

  \begin{proof}
    Supposons que $a$ et $b$ ont le même reste dans la division euclidienne par $n$ i.e. $a = qn + r$ et $b = q'n + r$. Alors par différence, $a - b = (q-q')n$ donc $a-b$ est un multiple de $n$.
    \newpar
    Réciproquement, si $a-b$ est un multiple de $n$ alors il existe $k$ tel que $a-b = kn$. En effectuant la division euclidienne de $a$ par $n$, on a $a = qn + r$, d'où $nq + r -b = kn$. Ainsi, $b = (q-k)n + r$ avec $0 \leq r < q-k$, ce que l'on voulait.
  \end{proof}

  Voici également un énoncé qui sera utilisé par la suite. C'est un corollaire du théorème de théorème de Bézout.

  \begin{proposition}[Corollaire du théorème de Bézout]
    \label{chiffrement-rsa-1}
    Soient $a$ et $b$ deux entiers naturels non nuls premiers entre eux. Alors, il existe $u$, $v \in \mathbb{Z}$ tels que $au + bv = 1$.
  \end{proposition}

  \begin{proof}
    On note par $S$ l'ensemble des entiers naturels strictement positifs $s$ qui s'écrivent $s = na + mb$ où $n$, $m \in \mathbb{Z}$. Cet ensemble est non-vide (car il contient $a$) et comme c'est un sous-ensemble de $\mathbb{N}$, il admet un plus petit élément $d = au + bv > 0$.
    \begin{itemize}
      \item On a $1 \mid d$ (car $1 \mid a$ et $1 \mid b$) donc $1 \leq d$.
      \item Faisons la division euclidienne de $a$ par $d$ : on a $a = dq + r \iff r = a - dq = a  - (au + bv)q = a(1 - uq) + b(-vq)$ donc $r = 0$ (car sinon on aurait $r \in S$ mais $r < d$ et $d$ est le plus petit élément de $S$). Donc $d$ divise $a$ et par le même raisonnement, $d$ divise $b$. Donc $d$ divise leur plus grand diviseur commun positif qui est $1$. Donc $d \leq 1$.
    \end{itemize}
    D'où finalement $d = 1$.
  \end{proof}

  \begin{remark}
    Il est possible de trouver de tels entiers $u$ et $v$ en effectuant la division euclidienne de $a$ par $b$, puis de $b$ par le reste de la division précédente, etc\dots et en remontant. Il s'agit de la remontée de \textbf{l'algorithme d'Euclide}.
  \end{remark}

  \begin{corollary}
    \label{chiffrement-rsa-2}
    Soient $a$ et $b$ deux entiers naturels non nuls premiers entre eux. Soit $c \in \mathbb{Z}$ tel que $a \mid c$ et $b \mid c$. Alors $ab \mid c$.
  \end{corollary}

  \begin{proof}
    On écrit $c = ka$ et $c = k'b$. De plus, par le \cref{chiffrement-rsa-1}, il existe $u$ et $v$ tels que $au + bv = 1$. En multipliant l'égalité par $c$, on obtient $c = auc + bvc = au(k'b) + bv(ka) = ab(k'u+kv)$. D'où le résultat.
  \end{proof}

  \begin{lemma}[Euclide]
    \label{chiffrement-rsa-4}
    Soit $p$ un nombre premier et $a$ et $b$ deux entiers. Si $p \mid ab$ alors $p \mid a$ ou $p \mid b$.
  \end{lemma}

  \begin{proof}
    Soit $p$ un nombre premier tel que $p \mid ab$. Supposons que $p$ ne divise pas $a$. Alors comme $p$ est premier, ses seuls diviseurs sont $1$ et $p$. Comme $a$ n'est pas divisible par $p$, le plus grand diviseur commun positif à $a$ et $p$ est $1$. Donc par le \cref{chiffrement-rsa-1} il existe $u$, $v \in \mathbb{Z}$ tels que $au + pv = 1$. En multipliant par $b$ on obtient $\underbrace{abu}_{\text{Multiple de } p} + \underbrace{pbv}_{\text{Multiple de } p} = b$. Ainsi $p \mid b$.
  \end{proof}

  \subsubsection{Congruences dans $\mathbb{Z}$}

  Dans toute cette sous-section, on fixe un entier naturel $n \geq 2$.

  \begin{definition}
    On dit que deux entiers relatifs $a$ et $b$ sont \textbf{congrus} modulo $n$ si $a$ et $b$ ont le même reste dans la division euclidienne par $n$. On note alors $a \equiv b \mod n$.
  \end{definition}

  \begin{remark}
    \label{chiffrement-rsa-3}
    On remarque que $a$ est un multiple de $n$ si et seulement si $a \equiv 0 \mod n$.
  \end{remark}

  On signale que la congruence est une \textbf{relation d'équivalence}.

  \begin{proposition}
    Pour tout $a$, $b$, $c \in \mathbb{Z}$ :
    \begin{enumerate}[label=(\roman*)]
      \item $a \equiv a \mod n$ (\textbf{réflexivité})
      \item Si $a \equiv b \mod n$, alors $b \equiv a \mod n$ (\textbf{symétrie})
      \item Si $a \equiv b \mod n$, et si $b \equiv c \mod n$, alors $a \equiv c \mod n$ (\textbf{transitivité})
    \end{enumerate}
  \end{proposition}

  De plus, le congruence est compatible avec les opérations usuelles sur les entiers relatifs.

  \begin{theorem}
    Soient $a$, $b$, $c$ et $d \in \mathbb{Z}$ tels que $a \equiv b \mod n$ et $c \equiv d \mod n$. Alors on a la compatibilité avec :
    \begin{enumerate}[label=(\roman*)]
      \item L'\textbf{addition} : $a + c \equiv b + d \mod n$.
      \item La \textbf{multiplication} : $ac \equiv bd \mod n$.
      \item Les \textbf{puissances} : pour tout $k \in \mathbb{N}$, $a^k \equiv b^k \mod n$.
    \end{enumerate}
  \end{theorem}

  \begin{proof}
    \begin{enumerate}[label=(\roman*)]
      \item Comme $a \equiv b \mod n$, et $c \equiv d \mod n$, alors $(a-b)$ et $(c-d)$ sont des multiples de $n$. Donc il existe deux entiers relatifs $k$ et $k'$ tels que $a-b = kn$ et $c-d = k'n$. En additionnant ces deux égalités on trouve que $(a+c) - (b+d) = (k+k')n$. Donc par la \cref{chiffrement-rsa-3}, $a + c \equiv b + d \mod n$.
      \item Comme précédemment, on a $a-b = kn$ et $c-d = k'n$. En multipliant les deux égalités on trouve que $ac = (b+kn)(d+k'n) = bd + (k'b + kd + kk'n)n$. Donc par la \cref{chiffrement-rsa-3}, $ac \equiv bd \mod n$.
      \item On utilise la compatibilité avec la multiplication : $a \equiv b \mod n$ et $a \equiv b \mod n$ donc $a^2 \equiv b^2 \mod n$. De même, on a $a^3 \equiv b^3 \mod n$. Il suffit de répéter l'opération $k$ fois et on a $a^k \equiv b^k \mod n$.
    \end{enumerate}
  \end{proof}

  \begin{example}
    Comme $7 \equiv 3 \mod 4$, et $5 \equiv 1 \mod 4$, on a $35 = 5 \times 7 \equiv 1 \times 5 \mod 4$.
  \end{example}

  Nous utiliserons le résultat suivant dans la sous-section suivante.

  \begin{theorem}[Petit théorème de Fermat]
    \label{chiffrement-rsa-5}
    Soit $p$ un nombre premier et $a$ un entier quelconque. Alors $a^p \equiv a \mod p$.
  \end{theorem}

  \begin{proof}
    Soit $p$ un nombre premier et soit $a$ tel que $p$ ne divise pas $a$. Notons :
    \begin{itemize}
      \item $N = a \times 2a \times 3a \times \dots \times (p-1)a$.
      \item $r_k$ le reste de la division euclidienne de $ka$ par $p$ pour tout $k \in \mathbb{N}$ tel que $1 \leq k \leq p-1$.
      \item $(p-1)! = 1 \times 2 \times \text { ... } \times p-1$.
    \end{itemize}
    Montrons que $N = (p-1)! a^{p-1}$. Il suffit en fait de réordonner les facteurs de $N$ :
    \[ N = a \times 2a \times \dots \times (p-1)a = 1 \times 2 \times 3 \times \dots \times (p-1) \times a \times a \times \dots \times a = (p-1)!a^{p-1} \tag{$*$} \]
    De plus, $r_k$ est le reste de la division euclidienne de $ka$ par $p$ donc $ka \equiv r_k \mod p$. Par $(*)$, $N = a \times 2a \times \dots \times (p-1)a \equiv r_1 r_2 \dots r_{p-1} \mod p$. $0 \leq k \leq p-1 < p$, donc $k$ ne peut pas être divisible par $p$. Ainsi, par le \cref{chiffrement-rsa-4}, $ka$ n'est pas divisible par $p$ et donc $r_k \neq 0$.
    \newpar
    Soit $i$ tel que $1 \leq i \leq p-1$ et $r_i = r_k$. Montrons que $(i-k)a$ est divisible par $p$. Comme $r_i$ et $r_k$ sont respectivement le reste de la division euclidienne de $ia$ et $ka$ par $p$, on a $r_i - r_k = 0 \equiv (i-k)a \mod p$ donc $(i-k)a$ est divisible par $p$. Comme $p$ ne divise pas $a$, par le \cref{chiffrement-rsa-4}, on a $i-j$ divise $p$. Et comme $-p < i-j < p$, on en déduit que $i=j$.
    \newpar
    Pour tout $k$, on a $1 \leq r_k \leq p-1$. De plus, par ce qui précède, on a $p-1$ $r_k$ qui sont tous différents les uns des autres. Donc $\{r_1, r_2, \dots, r_{p-1} \} = \{1, 2, \dots, p-1 \}$  Ainsi, $r_1 r_2 \dots r_{p-1} = (p-1)!$.
    \newpar
    Enfin, on a $N = (p-1)!a^{p-1} = a \times 2a \times \dots \times (p-1)a \equiv r_1 r_2 \dots r_{p-1} \mod p$ et $r_1 r_2 \dots r_{p-1} \equiv (p-1)! \mod p$, donc, on a $N \equiv (p-1)! \mod p$. De par la définition des congruences modulo $p$, on a $N - (p-1)!$ divisible par $p$ et $N - (p-1)! = (p-1)!(a^{p-1} - 1)$. Comme $p$ divise $(p-1)!(a^{p-1} - 1)$ mais que pour tout $k \in \mathbb{N}$ tel que $1 \leq k \leq p-1$, $p$ ne divise pas $k$, alors, en appliquant le \cref{chiffrement-rsa-4} à chaque facteur de $(p-1)!$, il en résulte que $p$ divise $a^{p-1} - 1$.
    \newpar
    Pour conclure, comme $a^{p-1} - 1 \equiv 0 \mod p$ alors $a^{p-1} \equiv 1 \mod p$ et donc $a^p \equiv a \mod p$. Nous venons de montrer que pour tout $a$ \uline{non divisible par $p$}, on a $a^p \equiv a \mod p$. Soit maintenant $b$ un entier divisible par $p$. Alors $b \equiv 0 \mod p$ et donc $b^p \equiv 0^p \equiv 0 \mod p$. D'où $b^p \equiv b \mod p$.
  \end{proof}

  \begin{example}
    Cherchons le reste de la division euclidienne de $2019^5$ par $5$.
    \newpar
    Posons $a = 2019$ et $p = 5$. En faisant la division euclidienne de $a$ par $p$, on a $a = 403p + 4$ donc $a \equiv 4 \mod p$. Donc $a^p \equiv a \equiv 4 \mod p$.
  \end{example}

  \subsection{RSA par un exemple}

  \subsubsection{Calcul de le clé publique et de la clé privée}

  Alice souhaite envoyer un code à Bob et uniquement à lui à travers un groupe de messagerie. Pour éviter de communiquer son code à tous les autres membres du groupe de messagerie, ils cherchent donc un moyen pour que seul Bob soit capable de le lire et de le déchiffrer : ils vont employer la méthode de chiffrement \textbf{RSA}.
  \newpar
  Ce chiffrement se base sur le choix de deux nombres premiers $p$ et $q$ distincts. On pose alors $n = pq$ et $N = (p-1)(q-1)$, puis on choisit $e < N$ premier avec $N$.

  \begin{definition}
    Le couple $(n, e)$ est la \textbf{clé publique} de ce chiffrement et est utilisée pour chiffrer des nombres, des caractères, des mots, ...
  \end{definition}

  Pour déchiffrer un nombre, on détermine deux entiers $u$ et $v$ vérifiant $ue + vN = 1$ (ils existent par le \cref{chiffrement-rsa-1}). On pose $d$ comme le reste de la division euclidienne de $u$ par $N$.

  \begin{definition}
    Le triplet $(p, q, d)$ est la \textbf{clé privée} du chiffrement.
  \end{definition}

  Dans notre exemple, c'est Bob qui va générer la clé publique et la clé privée. Ainsi, il choisit $p = 5$, $q = 7$ (donc $n = 35$ et $N = 24$) et $e = 5$. Comme $5e - N = 1$, alors $d = u = 5$.
  \newpar
  Il communique sa clé publique (qui est $(35, 5)$) dans la conversation et garde bien précieusement sa clé privée (qui est $(5, 7, 5)$).

  \subsubsection{Chiffrement du code}

  Le code d'Alice est $2743$, elle le décompose en chacun de ses chiffres : $a_0 = 2$, $a_1 = 7$, $a_3 = 4$ et $a_4 = 3$. Pour tout $k$, elle calcule ensuite $b_k$ qui est le reste de la division euclidienne de $a_k^e$ par $n$. Ainsi, elle obtient :

  \begin{center}
    \begin{whitetabularx}{|X|X|X|X|X|}
      \hline
      $a_k$ & $2$ & $7$ & $4$ & $3$ \\
      \hline
      $b_k$ & $32$ & $7$ & $9$ & $33$ \\
      \hline
    \end{whitetabularx}
  \end{center}

  Par conséquent, elle écrit dans la conversation :
  \[32 \; 7 \; 9 \; 33\]

  À titre d'exemple, pour calculer $b_1$, il s'agit de calculer le reste de la division euclidienne de $a_1^5 = 7^5$ par $n = 35$. On peut, par exemple, procéder comme ceci :

  \begin{itemize}
    \item $7^2 = 49 \equiv 14 \mod 35$
    \item $7^3 = 7^2 \times 7 \equiv 28 \equiv -7 \mod 35$
    \item $7^5 = 7^3 \times 7^2 \equiv -98 \equiv 7 \mod 35$
  \end{itemize}

  \subsubsection{Déchiffrement du code}

  Bob a donc reçu, une suite de quatre nombres $(b_k)$ avec $b_0 = 32$, $b_1 = 7$, $b_2 = 9$ et $b_3 = 33$. Pour déchiffrer la suite $(b_k)$ en une suite $(a_k)$, il s'agit alors pour tout $k$, de calculer le reste de la division euclidienne de $b_k^d$ par $n$. Ce reste est $a_k$. Ainsi, dans le cadre de notre exemple, cela nous donne :

  \begin{center}
    \begin{whitetabularx}{|X|X|X|X|X|}
      \hline
      $b_k$ & $32$ & $7$ & $9$ & $33$ \\
      \hline
      $a_k$ & $2$ & $7$ & $4$ & $3$ \\
      \hline
    \end{whitetabularx}
  \end{center}

  Le code déchiffré est donc :
  \[2 \; 7 \; 4 \; 3\]

  Ce qui correspond bien au code qu'Alice a voulu transmettre. De plus, seul Bob connaît le nombre $d$ qui permet de déchiffrer le code. Leur objectif est donc atteint.
  \newpar
  L'algorithme RSA est dit ``asymétrique'' car pour chiffrer un message, il suffit de connaître la clé publique $(n, e)$. Cependant, pour déchiffrer un message, il faut connaître $n$ et $d$. Or, $d$ se calcule à partir de $e$ et $n$ en trouvant les nombres premiers $p$ et $q$ qui divisent $n$. Donc finalement, pour déchiffrer un message, il faut connaître la clé privée $(p, q, d)$.
  \newpar
  Dans cet exemple, les nombres $p$ et $q$ ont été choisis petits de manière à simplifier les calculs, mais si on souhaitait mettre en place cet algorithme de chiffrement dans un cadre plus sécuritaire, il faudrait ainsi choisir des nombres $p$ et $q$ beaucoup plus grands.
  \newpar
  Le chiffrement demande donc de pouvoir vérifier que de très grands nombres sont des nombres premiers, pour pouvoir trouver $p$ et $q$, mais aussi que le produit de ces deux très grands nombres, ne soit pas factorisable pratiquement. En effet les algorithmes efficaces connus qui permettent de vérifier qu'un nombre n'est pas premier ne fournissent pas de factorisation.

  \begin{center}
    \begin{tikzpicture}
      \node at (0,0) {Alice};
      \node[draw] at (3,0) [align=center] {Texte\\en clair};
      \node[draw] at (6,0) [align=center] {Texte\\chiffré};
      \node[draw] at (9,0) [align=center] {Texte\\en clair};
      \node at (12,0) {Bob};
      \draw[->] (0.7,0) -- node [below] {\tiny Rédaction} (2,0);
      \draw[->] (4,0) -- node [below] {\tiny Chiffrement} node [above] {\tiny Clé publique} (5,0);
      \draw[->] (7,0) -- node [below] {\tiny Déchiffrement} node [above] {\tiny Clé privée} (8,0);
      \draw[->] (10,0) -- node [below] {\tiny Lecture} (11.3,0);
    \end{tikzpicture}
  \end{center}

  \subsection{Explication mathématique}

  Soient $p$ et $q \geq 2$ des nombres premiers distincts.

  \begin{notation}
    On note $n = pq$ et $N = \varphi(n) = (p-1)(q-1)$.
  \end{notation}

  Prouvons tout d'abord l'existence de la clé publique ainsi que l'existence de la clé privée.

  \begin{proposition}
    Soit $e$ un entier premier avec $N$ soit $1$. Alors il existe $d < N$ tel que $de \equiv 1 \mod N$.
  \end{proposition}

  \begin{proof}
    En effet, par le \cref{chiffrement-rsa-1}, il existe $u$ et $v \in \mathbb{Z}$ tels que $ue + vN = 1$. Donc, on a $ue = 1 - vN$ et ainsi $ue \equiv 1 - vN \equiv 1 \mod N$. Il suffit alors de poser $d$ le reste de la division euclidienne de $u$ par $N$.
  \end{proof}

  Montrons enfin que ce chiffrement est valide.

  \begin{proposition}
    Soit $M$ un entier naturel strictement inférieur à $n$ que nous souhaitons (dé-)chiffrer. On pose $C$ le reste de la division euclidienne de $M^e$ par $n$. Alors, $M \equiv C^d \mod n$.
  \end{proposition}

  \begin{proof}
    On a $C^d \equiv (M^e)^d \equiv M^{ed} \mod n$ et $ed \equiv 1 \mod N$ donc il existe $k$ tel que $ed = Nk + 1$. Comme $p$ et $q$ sont deux nombres premiers, alors par le \cref{chiffrement-rsa-5}, on a $M^{p-1} \equiv 1 \mod p$ et $M^{q-1} \equiv 1 \mod q$. Donc $M^{ed} = M^{Nk + 1} = M(M^{p-1})^{k(q-1)} \equiv M \mod p$. En faisant le même raisonnement, on a $M^{ed} \equiv M \mod q$. Ainsi, $M^{ed}-M$ est un multiple de $p$ et de $q$ (deux premiers distincts), donc aussi de $n$ par le \cref{chiffrement-rsa-2}. On conclue que $M \equiv M^{ed} \equiv C^d \mod n$.
  \end{proof}
  %</content>
\end{document}
