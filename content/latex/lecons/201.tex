\documentclass[12pt, a4paper]{report}

% LuaLaTeX :

\RequirePackage{iftex}
\RequireLuaTeX

% Packages :

\usepackage[french]{babel}
%\usepackage[utf8]{inputenc}
%\usepackage[T1]{fontenc}
\usepackage[pdfencoding=auto, pdfauthor={Hugo Delaunay}, pdfsubject={Mathématiques}, pdfcreator={agreg.skyost.eu}]{hyperref}
\usepackage{amsmath}
\usepackage{amsthm}
%\usepackage{amssymb}
\renewcommand{\proofname}{Solution}
\usepackage{stmaryrd}
\usepackage{tikz}
\usepackage{tkz-euclide}
\usepackage{fontspec}
\defaultfontfeatures[Erewhon]{FontFace = {bx}{n}{Erewhon-Bold.otf}}
\usepackage{fourier-otf}
\usepackage[nobottomtitles*]{titlesec}
\usepackage{fancyhdr}
\usepackage{listings}
\usepackage{catchfilebetweentags}
\usepackage[french, capitalise, noabbrev]{cleveref}
\usepackage[fit, breakall]{truncate}
\usepackage[top=2.5cm, right=2cm, bottom=2.5cm, left=2cm]{geometry}
\usepackage{enumitem}
\usepackage{tablists} %Pour faire 1)  2) 3)
\usepackage{tocloft}
\usepackage{microtype}
%\usepackage{mdframed}
%\usepackage{thmtools}
\usepackage{xcolor}
\usepackage{tabularx}
\usepackage{xltabular}
\usepackage{aligned-overset}
\usepackage[subpreambles=true]{standalone}
\usepackage{environ}
\usepackage[normalem]{ulem}
\usepackage{multicol}
 \usepackage{variations}
\usepackage{array}% Pour faire des tableaux
\usepackage{etoolbox}
\usepackage{setspace}
\usepackage[bibstyle=reading, citestyle=draft]{biblatex}
\usepackage{xpatch}
\usepackage[many, breakable]{tcolorbox}
\usepackage[backgroundcolor=white, bordercolor=white, textsize=scriptsize]{todonotes}
\usepackage{luacode}
\usepackage{float}
\usepackage{needspace}


% Police :

\setmathfont{Erewhon Math}

% Tikz :

\usetikzlibrary{calc}
\usetikzlibrary{3d}

% Longueurs :

\setlength{\parindent}{0pt}
\setlength{\headheight}{15pt}
\setlength{\fboxsep}{0pt}
\titlespacing*{\chapter}{0pt}{-20pt}{10pt}
\setlength{\marginparwidth}{1.5cm}
\setstretch{1.1}

% Métadonnées :

\author{agreg.skyost.eu}
\date{\today}

% Titres :

\setcounter{secnumdepth}{3}

\renewcommand{\thechapter}{\Roman{chapter}}
\renewcommand{\thesubsection}{\Roman{subsection}}
\renewcommand{\thesubsubsection}{\arabic{subsubsection}}
\renewcommand{\theparagraph}{\alph{paragraph}}

\titleformat{\chapter}{\huge\bfseries}{\thechapter}{20pt}{\huge\bfseries}
\titleformat*{\section}{\LARGE\bfseries}
\titleformat{\subsection}{\Large\bfseries}{\thesubsection \, - \,}{0pt}{\Large\bfseries}
\titleformat{\subsubsection}{\large\bfseries}{\thesubsubsection. \,}{0pt}{\large\bfseries}
\titleformat{\paragraph}{\bfseries}{\theparagraph. \,}{0pt}{\bfseries}

\setcounter{secnumdepth}{4}

% Table des matières :

\renewcommand{\cftsecleader}{\cftdotfill{\cftdotsep}}
\addtolength{\cftsecnumwidth}{10pt}

% Redéfinition des commandes :

\renewcommand*\thesection{\arabic{section}}
\renewcommand{\ker}{\mathrm{Ker}}

% Nouvelles commandes :

\newcommand{\website}{http://sencoursdemaths.com}

\newcommand{\tr}[1]{\mathstrut ^t #1}
\newcommand{\im}{\mathrm{Im}}
\newcommand{\rang}{\operatorname{rang}}
\newcommand{\trace}{\operatorname{trace}}
\newcommand{\id}{\operatorname{id}}
\newcommand{\stab}{\operatorname{Stab}}
\newcommand{\paren}[1]{\left(#1\right)}
\newcommand{\accol}[1]{\left\{#1\right\}}
\newcommand{\croch}[1]{\left[ #1 \right]}
\newcommand{\Grdcroch}[1]{\Bigl[ #1 \Bigr]}
\newcommand{\grdcroch}[1]{\bigl[ #1 \bigr]}
\newcommand{\abs}[1]{\left\lvert #1 \right\rvert}
\newcommand{\limi}[3]{\displaystyle \lim_{#1\to #2}#3}
\newcommand{\pinf}{+\infty}
\newcommand{\minf}{-\infty}
%%%%%%%%%%%%%% ENSEMBLES %%%%%%%%%%%%%%%%%
\newcommand{\ensemblenombre}[1]{\mathbb{#1}}
\newcommand{\Nn}{\ensemblenombre{N}}
\newcommand{\Zz}{\ensemblenombre{Z}}
\newcommand{\Qq}{\ensemblenombre{Q}}
\newcommand{\Qqp}{\Qq^+}
\newcommand{\Rr}{\ensemblenombre{R}}
\newcommand{\Cc}{\ensemblenombre{C}}
\newcommand{\Nne}{\Nn^*}
\newcommand{\Zze}{\Zz^*}
\newcommand{\Zzn}{\Zz^-}
\newcommand{\Qqe}{\Qq^*}
\newcommand{\Rre}{\Rr^*}
\newcommand{\Rrp}{\Rr_+}
\newcommand{\Rrm}{\Rr_-}
\newcommand{\Rrep}{\Rr_+^*}
\newcommand{\Rrem}{\Rr_-^*}
\newcommand{\Cce}{\Cc^*}
%%%%%%%%%%%%%%  INTERVALLES %%%%%%%%%%%%%%%%%
\newcommand{\intff}[2]{\left[#1\;,\; #2\right]  }
\newcommand{\intof}[2]{\left]#1 \;, \;#2\right]  }
\newcommand{\intfo}[2]{\left[#1 \;,\; #2\right[  }
\newcommand{\intoo}[2]{\left]#1 \;,\; #2\right[  }



\providecommand{\newpar}{\\[\medskipamount]}

\newcommand{\annexessection}{%
  \newpage%
  \subsection*{Annexes}%
}

\providecommand{\lesson}[3]{%
  \title{#3}%
  \hypersetup{pdftitle={#2 : #3}}%
  \setcounter{section}{\numexpr #2 - 1}%
  \section{#3}%
  \fancyhead[R]{\truncate{0.73\textwidth}{#2 : #3}}%
}

\providecommand{\development}[3]{%
  \title{#3}%
  \hypersetup{pdftitle={#3}}%
  \section*{#3}%
  \fancyhead[R]{\truncate{0.73\textwidth}{#3}}%
}

\providecommand{\sheet}[3]{\development{#1}{#2}{#3}}

\providecommand{\ranking}[1]{%
  \title{Terminale #1}%
  \hypersetup{pdftitle={Terminale #1}}%
  \section*{Terminale #1}%
  \fancyhead[R]{\truncate{0.73\textwidth}{Terminale #1}}%
}

\providecommand{\summary}[1]{%
  \textit{#1}%
  \par%
  \medskip%
}

\tikzset{notestyleraw/.append style={inner sep=0pt, rounded corners=0pt, align=center}}

%\newcommand{\booklink}[1]{\website/bibliographie\##1}
\newcounter{reference}
\newcommand{\previousreference}{}
\providecommand{\reference}[2][]{%
  \needspace{20pt}%
  \notblank{#1}{
    \needspace{20pt}%
    \renewcommand{\previousreference}{#1}%
    \stepcounter{reference}%
    \label{reference-\previousreference-\thereference}%
  }{}%
  \todo[noline]{%
    \protect\vspace{20pt}%
    \protect\par%
    \protect\notblank{#1}{\cite{[\previousreference]}\\}{}%
    \protect\hyperref[reference-\previousreference-\thereference]{p. #2}%
  }%
}

\definecolor{devcolor}{HTML}{00695c}
\providecommand{\dev}[1]{%
  \reversemarginpar%
  \todo[noline]{
    \protect\vspace{20pt}%
    \protect\par%
    \bfseries\color{devcolor}\href{\website/developpements/#1}{[DEV]}
  }%
  \normalmarginpar%
}

% En-têtes :

\pagestyle{fancy}
\fancyhead[L]{\truncate{0.23\textwidth}{\thepage}}
\fancyfoot[C]{\scriptsize \href{\website}{\texttt{http://sencoursdemaths.com}}}

% Couleurs :

\definecolor{property}{HTML}{ffeb3b}
\definecolor{proposition}{HTML}{ffc107}
\definecolor{lemma}{HTML}{ff9800}
\definecolor{theorem}{HTML}{f44336}
\definecolor{corollary}{HTML}{e91e63}
\definecolor{definition}{HTML}{673ab7}
\definecolor{notation}{HTML}{9c27b0}
\definecolor{example}{HTML}{00bcd4}
\definecolor{cexample}{HTML}{795548}
\definecolor{application}{HTML}{009688}
\definecolor{remark}{HTML}{3f51b5}
\definecolor{algorithm}{HTML}{607d8b}
\definecolor{proof}{HTML}{e1f5fe}
\definecolor{exercice}{HTML}{e1f5fe}

% Théorèmes :

\theoremstyle{definition}
\newtheorem{theorem}{Théorème}

\newtheorem{property}[theorem]{Propriété}
\newtheorem{proposition}[theorem]{Proposition}
\newtheorem{lemma}[theorem]{Activité d'introduction}
\newtheorem{corollary}[theorem]{Conséquence}

\newtheorem{definition}[theorem]{Définition}
\newtheorem{notation}[theorem]{Notation}

\newtheorem{example}[theorem]{Exemple}
\newtheorem{cexample}[theorem]{Contre-exemple}
\newtheorem{application}[theorem]{Application}

\newtheorem{algorithm}[theorem]{Algorithme}
\newtheorem{exercice}[theorem]{Exercice}

\theoremstyle{remark}
\newtheorem{remark}[theorem]{Remarque}




\counterwithin*{theorem}{section}

\newcommand{\applystyletotheorem}[1]{
  \tcolorboxenvironment{#1}{
    enhanced,
    breakable,
    colback=#1!8!white,
    %right=0pt,
    %top=8pt,
    %bottom=8pt,
    boxrule=0pt,
    frame hidden,
    sharp corners,
    enhanced,borderline west={4pt}{0pt}{#1},
    %interior hidden,
    sharp corners,
    after=\par,
  }
}

\applystyletotheorem{property}
\applystyletotheorem{proposition}
\applystyletotheorem{lemma}
\applystyletotheorem{theorem}
\applystyletotheorem{corollary}
\applystyletotheorem{definition}
\applystyletotheorem{notation}
\applystyletotheorem{example}
\applystyletotheorem{cexample}
\applystyletotheorem{application}
\applystyletotheorem{remark}
%\applystyletotheorem{proof}
\applystyletotheorem{algorithm}
\applystyletotheorem{exercice}

% Environnements :

\NewEnviron{whitetabularx}[1]{%
  \renewcommand{\arraystretch}{2.5}
  \colorbox{white}{%
    \begin{tabularx}{\textwidth}{#1}%
      \BODY%
    \end{tabularx}%
  }%
}

% Maths :

\DeclareFontEncoding{FMS}{}{}
\DeclareFontSubstitution{FMS}{futm}{m}{n}
\DeclareFontEncoding{FMX}{}{}
\DeclareFontSubstitution{FMX}{futm}{m}{n}
\DeclareSymbolFont{fouriersymbols}{FMS}{futm}{m}{n}
\DeclareSymbolFont{fourierlargesymbols}{FMX}{futm}{m}{n}
\DeclareMathDelimiter{\VERT}{\mathord}{fouriersymbols}{152}{fourierlargesymbols}{147}

% Code :

\definecolor{greencode}{rgb}{0,0.6,0}
\definecolor{graycode}{rgb}{0.5,0.5,0.5}
\definecolor{mauvecode}{rgb}{0.58,0,0.82}
\definecolor{bluecode}{HTML}{1976d2}
\lstset{
  basicstyle=\footnotesize\ttfamily,
  breakatwhitespace=false,
  breaklines=true,
  %captionpos=b,
  commentstyle=\color{greencode},
  deletekeywords={...},
  escapeinside={\%*}{*)},
  extendedchars=true,
  frame=none,
  keepspaces=true,
  keywordstyle=\color{bluecode},
  language=Python,
  otherkeywords={*,...},
  numbers=left,
  numbersep=5pt,
  numberstyle=\tiny\color{graycode},
  rulecolor=\color{black},
  showspaces=false,
  showstringspaces=false,
  showtabs=false,
  stepnumber=2,
  stringstyle=\color{mauvecode},
  tabsize=2,
  %texcl=true,
  xleftmargin=10pt,
  %title=\lstname
}

\newcommand{\codedirectory}{}
\newcommand{\inputalgorithm}[1]{%
  \begin{algorithm}%
    \strut%
    \lstinputlisting{\codedirectory#1}%
  \end{algorithm}%
}




% Bibliographie :

%\addbibresource{\bibliographypath}%
\defbibheading{bibliography}[\bibname]{\section*{#1}}
\renewbibmacro*{entryhead:full}{\printfield{labeltitle}}%
\DeclareFieldFormat{url}{\newline\footnotesize\url{#1}}%

\AtEndDocument{%
  \newpage%
  \pagestyle{empty}%
  \printbibliography%
}


\begin{document}
  %<*content>
  \lesson{analysis}{201}{Espaces de fonctions. Exemples et applications.}

  \subsection{Espaces de fonctions continues sur un compact}

  \subsubsection{Continuité et compacité}

  \reference[DAN]{55}

  \begin{proposition}
    Soient $(E, d)$ et $(E', d')$ deux espaces métriques. On suppose $E$ compact. Si $f : E \rightarrow E'$ est continue, alors $f(E)$ est compact.
  \end{proposition}

  \begin{cexample}
    Cela ne marche pas si $f$ n'est pas continue. Par exemple, $\arcsin([-1, 1]) = \mathbb{R}$.
  \end{cexample}

  \begin{proposition}
    Sous les mêmes hypothèses et en supposant $f$ bijective, $f^{-1}$ est continue (ie. $f$ est un homéomorphisme).
  \end{proposition}

  \begin{theorem}[Des bornes]
    Une application continue sur un compact est bornée et atteint ses bornes.
  \end{theorem}

  \begin{theorem}[Heine]
    Une application continue sur un compact y est uniformément continue.
  \end{theorem}

  \begin{corollary}
    Toute fonction périodique continue sur $\mathbb{R}$ y est uniformément continue.
  \end{corollary}

  \subsubsection{Convergences simple et uniforme}

  \reference[GOU20]{231}

  \begin{definition}
    Soient $(f_n)$ et $f$ respectivement une suite de fonctions et une fonction définies sur un ensemble $X$ à valeurs dans un espace métrique $(E, d)$. On dit que :
    \begin{itemize}
      \item $(f_n)$ \textbf{converge simplement} vers $f$ si
      \[ \forall x \in X, \, \forall \epsilon > 0, \, \exists N \in \mathbb{N} \text{ tel que } \forall n \geq N, \, d(f_n(x), f(x)) < \epsilon \]
      \item $(f_n)$ \textbf{converge uniformément} vers $f$ si
      \[ \forall \epsilon > 0, \, \exists N \in \mathbb{N} \text{ tel que } \forall n \geq N, \, \forall x \in X, \, d(f_n(x), f(x)) < \epsilon \]
    \end{itemize}
  \end{definition}

  \begin{proposition}
    La convergence uniforme entraîne la convergence simple.
  \end{proposition}

  \begin{cexample}
    La réciproque est fausse. Il suffit en effet de considérer la suite $(f_n)$ définie pour tout $n \in \mathbb{N}$ et pour tout $x \in [0,1]$ par $f_n(x) = x^n$ converge simplement sur $[0,1]$ mais pas uniformément.
  \end{cexample}

  \begin{theorem}[Critère de Cauchy uniforme]
    Soit $(f_n)$ une suite de fonctions définies sur un ensemble $X$ à valeurs dans un espace métrique $(E, d)$. Alors $(f_n)$ converge uniformément si
    \[ \forall \epsilon > 0, \, \exists N \in \mathbb{N} \text{ tel que } \forall p > q \geq N, \forall x \in X, \, d(f_p(x), f_q(x)) < \epsilon \]
  \end{theorem}

  \reference{237}

  \begin{corollary}
    Une limite uniforme sur $\mathbb{R}$ de fonctions polynômiales est une fonction polynômiale.
  \end{corollary}

  \reference{232}

  \begin{notation}
    \begin{itemize}
      \item Pour toute fonction $g$ bornée sur un ensemble $X$ et à valeurs dans un espace vectoriel normé $(E, \Vert . \Vert)$, on note
      \[ \Vert g \Vert_\infty = \sup_{x \in X} \Vert g(x) \Vert \]
      \item On note $\mathcal{B}(X,E)$ l'ensemble des applications bornées de $X$ dans $E$.
    \end{itemize}
  \end{notation}

  \begin{proposition}
    En reprenant les notations précédentes, une suite de fonctions $(f_n)$ de $\mathcal{B}(X,E)$ converge uniformément vers $f \in \mathcal{B}(X,E)$ si $\Vert f_n - f \Vert_\infty \longrightarrow_{n \rightarrow +\infty} 0$.
  \end{proposition}

  \begin{proposition}
    Si $E$ est de Banach, alors $(\mathcal{B}(X,E), \Vert . \Vert_\infty)$ est de Banach.
  \end{proposition}

  \reference{238}

  \begin{theorem}[Théorèmes de Dini]
    \begin{enumerate}[label=(\roman*)]
      \item Soit $(f_n)$ une suite \textit{croissante} de fonctions réelles \textit{continues} définies sur un segment $I$ de $\mathbb{R}$. Si $(f_n)$ converge simplement vers une fonction \textit{continue} sur $I$, alors la convergence est uniforme.
      \item Soit $(f_n)$ une suite de \textit{fonctions croissantes} réelles \textit{continues} définies sur un segment $I$ de $\mathbb{R}$. Si $(f_n)$ converge simplement vers une fonction \textit{continue} sur $I$, alors la convergence est uniforme.
    \end{enumerate}
  \end{theorem}

  \subsubsection{Densité}

  \reference{304}
  \dev{theoreme-de-weierstrass-par-la-convolution}

  \begin{theorem}[Weierstrass]
    Toute fonction continue $f : [a,b] \rightarrow \mathbb{R}$ (avec $a, b \in \mathbb{R}$ tels que $a \leq b$) est limite uniforme de fonctions polynômiales sur $[a, b]$.
  \end{theorem}

  On a une version plus générale de ce théorème.

  \reference[LI]{46}

  \begin{theorem}[Stone-Weierstrass]
    Soit $K$ un espace compact et $\mathcal{A}$ une sous-algèbre de l'algèbre de Banach réelle $\mathcal{C}(K, \mathbb{R})$. On suppose de plus que :
    \begin{enumerate}[label=(\roman*)]
      \item $\mathcal{A}$ sépare les points de $K$ (ie. $\forall x \in K, \exists f \in A \text{ telle que } f(x) \neq f(y)$).
      \item $\mathcal{A}$ contient les constantes.
    \end{enumerate}
    Alors $\mathcal{A}$ est dense dans $\mathcal{C}(K, \mathbb{R})$.
  \end{theorem}

  \begin{remark}
    Il existe aussi une version ``complexe'' de ce théorème, où il faut supposer de plus que $\mathcal{A}$ est stable par conjugaison.
  \end{remark}

  \begin{example}
    La suite de polynômes réels $(r_n)$ définie par récurrence par
    \[ r_0 = 0 \text{ et } \forall n \in \mathbb{N}, r_{n+1} : t \mapsto r_n(t) + \frac{1}{2} (t - r_n(t)^2) \]
    converge vers $\sqrt{.}$ sur $[0,1]$.
  \end{example}

  \subsection{Espaces \texorpdfstring{$L_p$}{Lp}}

  \reference[G-K]{209}

  Soit $(X, \mathcal{A}, \mu)$ un espace mesuré. Les résultats qui vont suivre sont, par extension, également valable pour les fonctions à valeurs dans $\mathbb{C}$.

  \subsubsection{Espaces \texorpdfstring{$\mathcal{L}_p$}{ℒp}}

  \begin{definition}
    \begin{itemize}
      \item Pour $p \in [1, +\infty[$, on note $\mathcal{L}_p(X, \mathcal{A}, \mu))$ (où $\mathcal{L}_p$ en l'absence d'ambiguïté) l'ensemble des applications $f$ mesurables de $(X, \mathcal{A}, \mu)$ dans $(\mathbb{R}, \mathcal{B}(R))$ telles que
      \[ \int_X \vert f(x) \vert^p \, \mathrm{d}\mu(x) < +\infty \]
      on note alors $\Vert f \Vert_p = \left(\int_X \vert f(x) \vert^p \, \mathrm{d}\mu(x)\right)^{\frac{1}{p}}$.
      \item On note de même $\mathcal{L}_\infty$ l'ensemble des applications mesurables de $(X, \mathcal{A}, \mu)$ dans $(\mathbb{R}, \mathcal{B}(R))$ de sup-essentiel borné. On note alors $\Vert f \Vert_\infty$ pour $f \in \mathcal{L}_\infty$.
    \end{itemize}
  \end{definition}

  \reference[B-P]{163}

  \begin{example}
    Si $\mu$ est la mesure de comptage sur $(\mathcal{P}(\mathbb{N}), \mathbb{N})$, alors
    \[ \mathcal{L}_p = \ell_p = \left\{ (u_n) \in \mathbb{R}^n \mid \sum_{n \geq 0} \vert u_n \vert^p < +\infty \right\} \]
  \end{example}

  \begin{proposition}
    $\mathcal{L}_p$ est un sous-espace vectoriel de l'espace vectoriel des fonctions de $X$ dans $\mathbb{R}$.
  \end{proposition}

  \reference[G-K]{209}

  \begin{theorem}[Inégalité de Hölder]
    Soient $p, q \in ]1, +\infty[$ tels que $\frac{1}{p} + \frac{1}{q} = 1$, $f \in \mathcal{L}_p$ et $g \in \mathcal{L}_q$. Alors $fg \in \mathcal{L}_1$ et
    \[ \Vert fg \Vert_1 \leq \Vert f \Vert_p \Vert g \Vert_q \]
  \end{theorem}

  \begin{remark}
    C'est encore vrai pour $q = +\infty$ en convenant que $\frac{1}{+\infty} = 0$.
  \end{remark}

  \begin{application}
    On considère la fonction $\Gamma$ d'Euler. Alors,
    \[ \forall \theta \in ]0,1[, \forall x, y > 0, \, \, \Gamma(\Theta x + (1 - \Theta)y) \leq \Gamma(x)^{\theta} \Gamma(y)^{1 - \theta} \]
    et en particulier, $\Gamma$ est log-convexe sur $\mathbb{R}^+_*$.
  \end{application}

  \begin{theorem}[Inégalité de Minkowski]
    \[ \forall f, g \in \mathcal{L}_p, \, \Vert f + g \Vert_p \leq \Vert f \Vert_p + \Vert g \Vert_p \]
  \end{theorem}

  L'application $\Vert . \Vert_p$ définit donc une semi-norme sur $\mathcal{L}_p$ pour $p \in [1, +\infty]$. L'idée dans la sous-section suivante sera de construire un espace dans lequel l'axiome de séparation n'est pas pris en défaut.

  \subsubsection{Construction des espaces \texorpdfstring{$L_p$}{Lp}}

  \begin{definition}
    On définit pour tout $p \in [1, +\infty]$,
    \[ L_p = \mathcal{L}_p / V \]
    où $V = \{ v \in \mathcal{L}_p \mid v = 0 \text{ pp.} \}$.
  \end{definition}

  \begin{proposition}
    Dans un espace de mesure finie,
    \[ 1 \leq p < q \leq +\infty \implies L_q \subseteq L_p \]
  \end{proposition}

  \begin{cexample}
    La fonction $\mathbb{1}$ est dans $L_\infty(\mathbb{R}, \mathcal{B}(\mathbb{R}), \lambda)$ mais dans aucun $L_p(\mathbb{R}, \mathcal{B}(\mathbb{R}), \lambda)$ pour tout $p \in [1, +\infty[$.
  \end{cexample}

  \begin{theorem}
    Pour tout $p \in [1, +\infty]$, $(L_p), \Vert . \Vert_p$ est un espace vectoriel normé.
  \end{theorem}

  \begin{theorem}[Riesz-Fischer]
    Pour tout $p \in [1, +\infty]$, $L_p$ est complet pour la norme $\Vert . \Vert_p$.
  \end{theorem}

  \subsubsection{Convolution et régularisation dans \texorpdfstring{$L_1$}{L₁}}

  \reference[AMR08]{75}

  \begin{definition}
    Soient $f$ et $g$ deux fonctions de $\mathbb{R}^d$ dans $\mathbb{R}$. On dit que \textbf{la convolée} (ou \textbf{le produit de convolution}) de $f$ et $g$ en $x \in \mathbb{R}$ \textbf{existe} si la fonction
    \[
    \begin{array}{ccc}
      \mathbb{R} &\rightarrow& \mathbb{C} \\
      t &\mapsto& f(x-t)g(t)
    \end{array}
    \]
    est intégrable sur $\mathbb{R}^d$ pour la mesure de Lebesgue. On pose alors :
    \[ (f * g)(x) = \int_{\mathbb{R}^d} f(x-t)g(t) \, \mathrm{d}t \]
  \end{definition}

  \begin{example}
    Soient $a < b \in \mathbb{R}^+_*$. Alors $\mathbb{1}_{[-a, a]} * \mathbb{1}_{[-b,b]}$ existe pour tout $x \in \mathbb{R}$ et
    \[ \left( \mathbb{1}_{[-a, a]} * \mathbb{1}_{[-b,b]} \right)(x) =
    \begin{cases}
      2a &\text{si } 0 \leq \vert x \vert \leq b-a \\
      b+a-\vert x \vert &\text{si } b-a \leq \vert x \vert \leq b+a \\
      0 &\text{sinon}
    \end{cases}
    \]
  \end{example}

  \begin{proposition}
    Dans $L_1(\mathbb{R}^d)$, dès qu'il a un sens, le produit de convolution de deux fonctions est commutatif, bilinéaire et associatif.
  \end{proposition}

  \begin{theorem}[Convolution dans $L_1(\mathbb{R}^d)$]
    Soient $f, g \in L_1(\mathbb{R}^d)$. Alors :
    \begin{enumerate}[label=(\roman*)]
      \item pp. en $x \in \mathbb{R}^d$, $t \mapsto f(x-t)g(t)$ est intégrable sur $\mathbb{R}^d$.
      \item $f * g$ est intégrable sur $\mathbb{R}^d$.
      \item $\Vert f * g \Vert_1 \leq \Vert f \Vert_1 \Vert g \Vert_1$.
      \item L'espace vectoriel normé $(L_1(\mathbb{R}^d), \Vert . \Vert_1)$ muni de $*$ est une algèbre de Banach commutative.
    \end{enumerate}
  \end{theorem}

  \reference{114}

  \begin{proposition}
    L'algèbre $(L_1(\mathbb{R}^d), +, *, \cdot)$ n'a pas d'élément unité.
  \end{proposition}

  \begin{application}
    \[ f * f = f \iff f = 0 \]
  \end{application}

  \reference[B-P]{306}

  \begin{definition}
    On appelle \textbf{approximation de l'identité} toute suite $(\rho_n)$ de fonctions mesurables de $L_1(\mathbb{R}^d)$ telles que
    \begin{enumerate}[label=(\roman*)]
      \item $\forall n \in \mathbb{N}, \, \int_{\mathbb{R}^d} \rho_n \, \mathrm{d}\lambda_d = 1$.
      \item $\sup_{n \geq 1} \Vert \rho_n \Vert < +\infty$.
      \item $\forall \epsilon > 0, \, \lim_{n \rightarrow +\infty} \int_{\mathbb{R} \setminus B(0, \epsilon)} \rho_n(x) \, \mathrm{d}x = 0$.
    \end{enumerate}
  \end{definition}

  \reference[GOU20]{304}

  \begin{example}
    $\forall n \in \mathbb{N}$, on note :
    \[ a_n = \int_{-1}^1 (1-t^2)^n \, \mathrm{d}t \text{ et } p_n : t \mapsto \frac{(1-t^2)^n}{a_n} \mathbb{1}_{[-1, 1]}(t) \]
    Alors, $(p_n)$ est une approximation positive de l'identité.
  \end{example}

  \reference[AMR08]{96}

  \begin{application}
    \begin{enumerate}[label=(\roman*)]
      \item $\mathcal{C}^\infty_K(\mathbb{R}^d)$ est dense dans $\mathcal{C}_K(\mathbb{R}^d)$ pour $\Vert . \Vert_\infty$.
      \item $\mathcal{C}^\infty_K(\mathbb{R}^d)$ est dense dans $L_p(\mathbb{R}^d)$ pour $\Vert . \Vert_p$ avec $p \in [1, +\infty[$.
    \end{enumerate}
  \end{application}

  \subsection{Espace \texorpdfstring{$L_2$}{L₂}}

  \subsubsection{Propriétés hilbertiennes}

  \reference[BMP]{92}

  \begin{definition}
    On considère la forme bilinéaire suivante sur $L_2$ :
    \[ \langle ., . \rangle : (f, g) \mapsto \int_X f \overline{g} \, \mathrm{d}\mu \]
    C'est un produit scalaire hermitien, ce qui confère à $(L_2, \langle ., . \rangle)$ une structure d'espace de Hilbert.
  \end{definition}

  On peut donc énoncer quelques propriétés dont hérite $L_2$.

  \reference{98}

  \begin{theorem}
    Pour tout sous-espace vectoriel fermé $F$ de $L_2$,
    \[ L_2 = F \oplus F^\perp \]
  \end{theorem}

  \begin{corollary}
    Un sous-espace vectoriel $F$ de $L_2$ est dense dans $L_2$ si et seulement si $F^\perp = \{ 0 \}$.
  \end{corollary}

  \begin{theorem}
    Soit $(e_n)_{n \in I}$ une famille orthonormée dénombrable de $L_2$. Les propriétés suivantes sont équivalentes :
    \begin{enumerate}[label=(\roman*)]
      \item La famille orthonormée $(e_n)_{n \in I}$ est une base hilbertienne de $H$.
      \item $\forall f \in L_2, \, f = \sum_{n=0}^{+\infty} \langle f, e_n \rangle e_n$.
      \item $\forall f \in L_2, \, \Vert f \Vert_2 = \sum_{n=0}^{+\infty} \vert \langle f, e_n \rangle \vert^2$.
    \end{enumerate}
  \end{theorem}

  \subsubsection{Polynômes orthogonaux}

  \reference{110}

  Soit $I$ un intervalle de $\mathbb{R}$. On pose $\forall n \in \mathbb{N}$, $g_n : x \mapsto x^n$.

  \begin{definition}
    On appelle \textbf{fonction poids} une fonction $\rho : I \rightarrow \mathbb{R}$ mesurable, positive et telle que $\forall n \in \mathbb{N}, \rho g_n \in L_1(I)$.
  \end{definition}

  Soit $\rho : I \rightarrow \mathbb{R}$ une fonction poids.

  \begin{notation}
    On note $L_2(I, \rho)$ l'espace des fonctions de carré intégrable pour la mesure de densité $\rho$ par rapport à la mesure de Lebesgue.
  \end{notation}

  \begin{proposition}
    Muni de
    \[ \langle ., . \rangle : (f,g) \mapsto \int_I f(x) \overline{g(x)} \rho(x) \, \mathrm{d}x \]
    $L_2(I, \rho)$ est un espace de Hilbert.
  \end{proposition}

  \begin{theorem}
    Il existe une unique famille $(P_n)$ de polynômes unitaires orthogonaux deux-à-deux telle que $\deg(P_n) = n$ pour tout entier $n$. C'est la famille de \textbf{polynômes orthogonaux} associée à $\rho$ sur $I$.
  \end{theorem}

  \begin{example}[Polynômes de Hermite]
    Si $\forall x \in I, \, \rho(x) = e^{-x^2}$, alors
    \[ \forall n \in \mathbb{N}, \, \forall x \in I, \, P_n(x) = \frac{(-1)^n}{2^n} e^{x^2} \frac{\partial}{\partial x^n} \left( e^{-x^2} \right) \]
  \end{example}

  \reference{140}

  \begin{lemma}
    On suppose que $\forall n \in \mathbb{N}$, $g_n \in L_1(I, \rho)$ et on considère $(P_n)$ la famille des polynômes orthogonaux associée à $\rho$ sur $I$. Alors $\forall n \in \mathbb{N}$, $g_n \in L_2(I, \rho)$. En particulier, l'algorithme de Gram-Schmidt a bien du sens et $(P_n)$ est bien définie.
  \end{lemma}

  \begin{application}
    On considère $(P_n)$ la famille des polynômes orthogonaux associée à $\rho$ sur $I$ et on suppose qu'il existe $a > 0$ tel que
    \[ \int_I e^{a \vert x \vert} \rho(x) \, \mathrm{d}x < +\infty \]
    alors $(P_n)$ est une base hilbertienne de $L_2(I, \rho)$ pour la norme $\Vert . \Vert_2$.
  \end{application}

  \begin{cexample}
    On considère, sur $I = \mathbb{R}^+_*$, la fonction poids $\rho : x \mapsto x^{-\ln(x)}$. Alors, la famille des $g_n$ n'est pas totale. La famille des polynômes orthogonaux associée à ce poids particulier n'est donc pas totale non plus : ce n'est pas une base hilbertienne.
  \end{cexample}

  \subsection{Dualité}

  \reference[GOU21]{132}

  \begin{definition}
    On appelle \textbf{forme linéaire} d'un espace vectoriel $E$ sur un corps $\mathbb{K}$ toute application linéaire de $E$ dans $\mathbb{K}$ et on note $E^*$ appelé \textbf{dual} de $E$ l'ensemble des formes linéaires de $E$.
    \newpar
    On note $E'$ le \textbf{dual topologique} de $E$, qui est le sous-espace de $E^*$ constitué des formes linéaires continues.
  \end{definition}

  \reference[BMP]{103}

  \begin{theorem}[de représentation de Riesz]
    L'application
    \[
    \Phi :
    \begin{array}{ccc}
      H &\rightarrow& H' \\
      y &\mapsto& (x \mapsto \langle x, y \rangle)
    \end{array}
    \]
    est une isométrie linéaire bijective de $H$ sur son dual topologique $H'$.
  \end{theorem}

  \begin{example}
    Dans le cas $L_2(X, \mu)$,
    \[ \forall \varphi \in L_2', \, \exists! g \in H, \text{ telle que } \forall f \in L_2, \, \varphi(f) = \int_X f(g) \overline{g(x)} \, \mathrm{d}\mu(x) \]
  \end{example}

  \reference[Z-Q]{222}
  \dev{dual-de-lp}

  \begin{theorem}[Dual de $L_p$]
    On se place dans un espace mesuré de mesure finie. On note $\forall p \in ]1,2[$. L'application
    \[
    \varphi :
    \begin{array}{ll}
      L_q &\rightarrow (L_p)' \\
      g &\mapsto \left( \varphi_g : f \mapsto \int_X f g \, \mathrm{d}\mu \right)
    \end{array}
    \qquad \text{ où } \frac{1}{p} + \frac{1}{q} = 1
    \]
    est une isométrie linéaire surjective. C'est donc un isomorphisme isométrique.
  \end{theorem}

  \reference[LI]{140}

  \begin{remark}
    Plus généralement, si l'on identifie $g$ et $\varphi_g$ :
    \begin{itemize}
      \item $L_q$ est le dual topologique de $L_p$ pour $p \in ]1, +\infty[$.
      \item $L_\infty$ est le dual topologique de $L_1$ si $\mu$ est $\sigma$-finie.
    \end{itemize}
  \end{remark}
  %</content>
\end{document}
