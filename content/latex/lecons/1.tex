\documentclass[12pt, a4paper]{report}

% LuaLaTeX :

\RequirePackage{iftex}
\RequireLuaTeX

% Packages :

\usepackage[french]{babel}
%\usepackage[utf8]{inputenc}
%\usepackage[T1]{fontenc}
\usepackage[pdfencoding=auto, pdfauthor={Hugo Delaunay}, pdfsubject={Mathématiques}, pdfcreator={agreg.skyost.eu}]{hyperref}
\usepackage{amsmath}
\usepackage{amsthm}
%\usepackage{amssymb}
\renewcommand{\proofname}{Solution}
\usepackage{stmaryrd}
\usepackage{tikz}
\usepackage{tkz-euclide}
\usepackage{fontspec}
\defaultfontfeatures[Erewhon]{FontFace = {bx}{n}{Erewhon-Bold.otf}}
\usepackage{fourier-otf}
\usepackage[nobottomtitles*]{titlesec}
\usepackage{fancyhdr}
\usepackage{listings}
\usepackage{catchfilebetweentags}
\usepackage[french, capitalise, noabbrev]{cleveref}
\usepackage[fit, breakall]{truncate}
\usepackage[top=2.5cm, right=2cm, bottom=2.5cm, left=2cm]{geometry}
\usepackage{enumitem}
\usepackage{tablists} %Pour faire 1)  2) 3)
\usepackage{tocloft}
\usepackage{microtype}
%\usepackage{mdframed}
%\usepackage{thmtools}
\usepackage{xcolor}
\usepackage{tabularx}
\usepackage{xltabular}
\usepackage{aligned-overset}
\usepackage[subpreambles=true]{standalone}
\usepackage{environ}
\usepackage[normalem]{ulem}
\usepackage{multicol}
 \usepackage{variations}
\usepackage{array}% Pour faire des tableaux
\usepackage{etoolbox}
\usepackage{setspace}
\usepackage[bibstyle=reading, citestyle=draft]{biblatex}
\usepackage{xpatch}
\usepackage[many, breakable]{tcolorbox}
\usepackage[backgroundcolor=white, bordercolor=white, textsize=scriptsize]{todonotes}
\usepackage{luacode}
\usepackage{float}
\usepackage{needspace}


% Police :

\setmathfont{Erewhon Math}

% Tikz :

\usetikzlibrary{calc}
\usetikzlibrary{3d}

% Longueurs :

\setlength{\parindent}{0pt}
\setlength{\headheight}{15pt}
\setlength{\fboxsep}{0pt}
\titlespacing*{\chapter}{0pt}{-20pt}{10pt}
\setlength{\marginparwidth}{1.5cm}
\setstretch{1.1}

% Métadonnées :

\author{agreg.skyost.eu}
\date{\today}

% Titres :

\setcounter{secnumdepth}{3}

\renewcommand{\thechapter}{\Roman{chapter}}
\renewcommand{\thesubsection}{\Roman{subsection}}
\renewcommand{\thesubsubsection}{\arabic{subsubsection}}
\renewcommand{\theparagraph}{\alph{paragraph}}

\titleformat{\chapter}{\huge\bfseries}{\thechapter}{20pt}{\huge\bfseries}
\titleformat*{\section}{\LARGE\bfseries}
\titleformat{\subsection}{\Large\bfseries}{\thesubsection \, - \,}{0pt}{\Large\bfseries}
\titleformat{\subsubsection}{\large\bfseries}{\thesubsubsection. \,}{0pt}{\large\bfseries}
\titleformat{\paragraph}{\bfseries}{\theparagraph. \,}{0pt}{\bfseries}

\setcounter{secnumdepth}{4}

% Table des matières :

\renewcommand{\cftsecleader}{\cftdotfill{\cftdotsep}}
\addtolength{\cftsecnumwidth}{10pt}

% Redéfinition des commandes :

\renewcommand*\thesection{\arabic{section}}
\renewcommand{\ker}{\mathrm{Ker}}

% Nouvelles commandes :

\newcommand{\website}{http://sencoursdemaths.com}

\newcommand{\tr}[1]{\mathstrut ^t #1}
\newcommand{\im}{\mathrm{Im}}
\newcommand{\rang}{\operatorname{rang}}
\newcommand{\trace}{\operatorname{trace}}
\newcommand{\id}{\operatorname{id}}
\newcommand{\stab}{\operatorname{Stab}}
\newcommand{\paren}[1]{\left(#1\right)}
\newcommand{\accol}[1]{\left\{#1\right\}}
\newcommand{\croch}[1]{\left[ #1 \right]}
\newcommand{\Grdcroch}[1]{\Bigl[ #1 \Bigr]}
\newcommand{\grdcroch}[1]{\bigl[ #1 \bigr]}
\newcommand{\abs}[1]{\left\lvert #1 \right\rvert}
\newcommand{\limi}[3]{\displaystyle \lim_{#1\to #2}#3}
\newcommand{\pinf}{+\infty}
\newcommand{\minf}{-\infty}
%%%%%%%%%%%%%% ENSEMBLES %%%%%%%%%%%%%%%%%
\newcommand{\ensemblenombre}[1]{\mathbb{#1}}
\newcommand{\Nn}{\ensemblenombre{N}}
\newcommand{\Zz}{\ensemblenombre{Z}}
\newcommand{\Qq}{\ensemblenombre{Q}}
\newcommand{\Qqp}{\Qq^+}
\newcommand{\Rr}{\ensemblenombre{R}}
\newcommand{\Cc}{\ensemblenombre{C}}
\newcommand{\Nne}{\Nn^*}
\newcommand{\Zze}{\Zz^*}
\newcommand{\Zzn}{\Zz^-}
\newcommand{\Qqe}{\Qq^*}
\newcommand{\Rre}{\Rr^*}
\newcommand{\Rrp}{\Rr_+}
\newcommand{\Rrm}{\Rr_-}
\newcommand{\Rrep}{\Rr_+^*}
\newcommand{\Rrem}{\Rr_-^*}
\newcommand{\Cce}{\Cc^*}
%%%%%%%%%%%%%%  INTERVALLES %%%%%%%%%%%%%%%%%
\newcommand{\intff}[2]{\left[#1\;,\; #2\right]  }
\newcommand{\intof}[2]{\left]#1 \;, \;#2\right]  }
\newcommand{\intfo}[2]{\left[#1 \;,\; #2\right[  }
\newcommand{\intoo}[2]{\left]#1 \;,\; #2\right[  }



\providecommand{\newpar}{\\[\medskipamount]}

\newcommand{\annexessection}{%
  \newpage%
  \subsection*{Annexes}%
}

\providecommand{\lesson}[3]{%
  \title{#3}%
  \hypersetup{pdftitle={#2 : #3}}%
  \setcounter{section}{\numexpr #2 - 1}%
  \section{#3}%
  \fancyhead[R]{\truncate{0.73\textwidth}{#2 : #3}}%
}

\providecommand{\development}[3]{%
  \title{#3}%
  \hypersetup{pdftitle={#3}}%
  \section*{#3}%
  \fancyhead[R]{\truncate{0.73\textwidth}{#3}}%
}

\providecommand{\sheet}[3]{\development{#1}{#2}{#3}}

\providecommand{\ranking}[1]{%
  \title{Terminale #1}%
  \hypersetup{pdftitle={Terminale #1}}%
  \section*{Terminale #1}%
  \fancyhead[R]{\truncate{0.73\textwidth}{Terminale #1}}%
}

\providecommand{\summary}[1]{%
  \textit{#1}%
  \par%
  \medskip%
}

\tikzset{notestyleraw/.append style={inner sep=0pt, rounded corners=0pt, align=center}}

%\newcommand{\booklink}[1]{\website/bibliographie\##1}
\newcounter{reference}
\newcommand{\previousreference}{}
\providecommand{\reference}[2][]{%
  \needspace{20pt}%
  \notblank{#1}{
    \needspace{20pt}%
    \renewcommand{\previousreference}{#1}%
    \stepcounter{reference}%
    \label{reference-\previousreference-\thereference}%
  }{}%
  \todo[noline]{%
    \protect\vspace{20pt}%
    \protect\par%
    \protect\notblank{#1}{\cite{[\previousreference]}\\}{}%
    \protect\hyperref[reference-\previousreference-\thereference]{p. #2}%
  }%
}

\definecolor{devcolor}{HTML}{00695c}
\providecommand{\dev}[1]{%
  \reversemarginpar%
  \todo[noline]{
    \protect\vspace{20pt}%
    \protect\par%
    \bfseries\color{devcolor}\href{\website/developpements/#1}{[DEV]}
  }%
  \normalmarginpar%
}

% En-têtes :

\pagestyle{fancy}
\fancyhead[L]{\truncate{0.23\textwidth}{\thepage}}
\fancyfoot[C]{\scriptsize \href{\website}{\texttt{http://sencoursdemaths.com}}}

% Couleurs :

\definecolor{property}{HTML}{ffeb3b}
\definecolor{proposition}{HTML}{ffc107}
\definecolor{lemma}{HTML}{ff9800}
\definecolor{theorem}{HTML}{f44336}
\definecolor{corollary}{HTML}{e91e63}
\definecolor{definition}{HTML}{673ab7}
\definecolor{notation}{HTML}{9c27b0}
\definecolor{example}{HTML}{00bcd4}
\definecolor{cexample}{HTML}{795548}
\definecolor{application}{HTML}{009688}
\definecolor{remark}{HTML}{3f51b5}
\definecolor{algorithm}{HTML}{607d8b}
\definecolor{proof}{HTML}{e1f5fe}
\definecolor{exercice}{HTML}{e1f5fe}

% Théorèmes :

\theoremstyle{definition}
\newtheorem{theorem}{Théorème}

\newtheorem{property}[theorem]{Propriété}
\newtheorem{proposition}[theorem]{Proposition}
\newtheorem{lemma}[theorem]{Activité d'introduction}
\newtheorem{corollary}[theorem]{Conséquence}

\newtheorem{definition}[theorem]{Définition}
\newtheorem{notation}[theorem]{Notation}

\newtheorem{example}[theorem]{Exemple}
\newtheorem{cexample}[theorem]{Contre-exemple}
\newtheorem{application}[theorem]{Application}

\newtheorem{algorithm}[theorem]{Algorithme}
\newtheorem{exercice}[theorem]{Exercice}

\theoremstyle{remark}
\newtheorem{remark}[theorem]{Remarque}




\counterwithin*{theorem}{section}

\newcommand{\applystyletotheorem}[1]{
  \tcolorboxenvironment{#1}{
    enhanced,
    breakable,
    colback=#1!8!white,
    %right=0pt,
    %top=8pt,
    %bottom=8pt,
    boxrule=0pt,
    frame hidden,
    sharp corners,
    enhanced,borderline west={4pt}{0pt}{#1},
    %interior hidden,
    sharp corners,
    after=\par,
  }
}

\applystyletotheorem{property}
\applystyletotheorem{proposition}
\applystyletotheorem{lemma}
\applystyletotheorem{theorem}
\applystyletotheorem{corollary}
\applystyletotheorem{definition}
\applystyletotheorem{notation}
\applystyletotheorem{example}
\applystyletotheorem{cexample}
\applystyletotheorem{application}
\applystyletotheorem{remark}
%\applystyletotheorem{proof}
\applystyletotheorem{algorithm}
\applystyletotheorem{exercice}

% Environnements :

\NewEnviron{whitetabularx}[1]{%
  \renewcommand{\arraystretch}{2.5}
  \colorbox{white}{%
    \begin{tabularx}{\textwidth}{#1}%
      \BODY%
    \end{tabularx}%
  }%
}

% Maths :

\DeclareFontEncoding{FMS}{}{}
\DeclareFontSubstitution{FMS}{futm}{m}{n}
\DeclareFontEncoding{FMX}{}{}
\DeclareFontSubstitution{FMX}{futm}{m}{n}
\DeclareSymbolFont{fouriersymbols}{FMS}{futm}{m}{n}
\DeclareSymbolFont{fourierlargesymbols}{FMX}{futm}{m}{n}
\DeclareMathDelimiter{\VERT}{\mathord}{fouriersymbols}{152}{fourierlargesymbols}{147}

% Code :

\definecolor{greencode}{rgb}{0,0.6,0}
\definecolor{graycode}{rgb}{0.5,0.5,0.5}
\definecolor{mauvecode}{rgb}{0.58,0,0.82}
\definecolor{bluecode}{HTML}{1976d2}
\lstset{
  basicstyle=\footnotesize\ttfamily,
  breakatwhitespace=false,
  breaklines=true,
  %captionpos=b,
  commentstyle=\color{greencode},
  deletekeywords={...},
  escapeinside={\%*}{*)},
  extendedchars=true,
  frame=none,
  keepspaces=true,
  keywordstyle=\color{bluecode},
  language=Python,
  otherkeywords={*,...},
  numbers=left,
  numbersep=5pt,
  numberstyle=\tiny\color{graycode},
  rulecolor=\color{black},
  showspaces=false,
  showstringspaces=false,
  showtabs=false,
  stepnumber=2,
  stringstyle=\color{mauvecode},
  tabsize=2,
  %texcl=true,
  xleftmargin=10pt,
  %title=\lstname
}

\newcommand{\codedirectory}{}
\newcommand{\inputalgorithm}[1]{%
  \begin{algorithm}%
    \strut%
    \lstinputlisting{\codedirectory#1}%
  \end{algorithm}%
}





\begin{document}
  %<*content>
  \lesson{algebra}{1}{Fonctions numériques. Rappels et compléments}


\subsection{Limites}
Nous conseillons de lire le cours de première sur les limites avant de lire ce paragraphe.

\textsl{Lorsque nous écrivons $ \infty $ cela signifie que c'est valable pour $ +\infty$ comme pour $-\infty $ }

 Il existe quatre cas d'indétermination dans les opérations sur les limites:
\[ \text{<<} \pinf \minf \text{>>};\quad\text{<<} \frac{\infty}{\infty} \text{>>};\quad\text{<<} \frac{0}{0}\text{>>};\quad \text{<<} 0\times \infty \text{>>} \]
\subsubsection{Limites usuelles}
 $ n\in\Nne $
\begin{itemize}
\item[$ \bullet $] $\displaystyle \lim_{x \to \pinf}x^{n}= \pinf$ 
\item[$ \bullet $] $\displaystyle\lim_{x \to \minf}x^{n}= \begin{cases}
\pinf & \text{si n pair} \\
\minf & \text{si  n impair}
\end{cases}$ 
\item[$ \bullet $] $\displaystyle \lim_{x \to \pinf}\sqrt{x}=\pinf$ 
\item[$ \bullet $] $\displaystyle \lim_{x \to \infty} \frac{1}{x^{n}}=0$ 
\item[$ \bullet $] $\displaystyle \lim_{x \to 0}\frac{\sin x}{x}=1$ 
\item[$ \bullet $] $ \displaystyle\lim_{x \to 0}\frac{1-\cos x}{x^{2}}= \frac{1}{2}$ 

\end{itemize}
\begin{remark}
Les fonctions cosinus et sinus n'ont pas de limites à l'infini.
\end{remark}
\subsubsection{Limite de la composée de deux fonctions}
 Soient $f $  et $g$ deux fonctions, $a $, $b$  et  $c$ trois réels pouvant éventuellement être $ +\infty $ ou $ -\infty $. 
\[\text{Si}\displaystyle \lim_{x \to a} f(x) = b \quad \text{et} \quad \displaystyle \lim_{x \to b} f(x) = c \quad \text{alors} \quad \displaystyle \lim_{x \to a} g[f(x)] = c \]
  
\begin{example}
 Calculons $\displaystyle \lim_{x \to +\infty} \cos\paren{\frac{x+1}{x^{2}-2}}$ 
 
 $\displaystyle \lim_{x \to +\infty} \frac{x+1}{x^{2}-2} = 0 $ et $ \displaystyle \lim_{x \to 0} \cos x = \cos0=1 $ donc $  \displaystyle\lim_{x \to +\infty}\cos\paren{\frac{x+1}{x^{2}-2}}=1 $

\end{example}  

 \subsubsection{Comparaisons de limites}

Soient $f $, $g$ et $h$ trois fonctions et $ l\in\Rr$ ou $ l= +\infty $ ou $ l=-\infty $.


%\renewcommand{\arraystretch}{2.3}
\begin{tabularx}{\textwidth}{|X|X|X|}
\hline
\textbf{ Hypothèse 1}  & \textbf{Hypothèse 2} & \textbf{ Conclusion }\\
\hline
$ f\leq g $  & $ \displaystyle \lim_{x \to a}f(x)=+\infty $ & $\displaystyle \lim_{x \to a}g(x)=+\infty $ \\
 \hline
 $ f\leq g $  & $\displaystyle \lim_{x \to a}g(x)=-\infty $ & $ \displaystyle \lim_{x \to a}f(x)=-\infty $    \\
\hline
$ \abs{f(x)-l }\leq g(x) $ & $\displaystyle \lim_{x \to a}g(x)=0$ & $ \displaystyle \lim_{x \to a}f(x)=l  $ \\
\hline
$ g \leq f \leq h $ & $ \displaystyle \lim_{x \to a}g(x)=l $ et $ \displaystyle \lim_{x \to a}h(x)=l $ & $ \displaystyle \lim_{x \to a}f(x)=l $    \\
\hline
\end{tabularx}


\begin{remark}
 
La dernière propriété est parfois appelée << le théorème des gendarmes >>.
 \end{remark}

\begin{example}
$\triangleright $ Soit $ f(x)= x+ 3\cos x $.\\
Pour tout $x\in\Rr$: \quad $ x-3\leq f(x)\leq x+3 $ \\
$ \centerdot $ $ x-3\leq f(x)$ et $\displaystyle \lim_{x \to +\infty}x-3=+\infty  $ donc $ \displaystyle \lim_{x \to +\infty}f(x)=+\infty $\\
$ \centerdot $ $  f(x)\leq x+3$ et $\displaystyle \lim_{x \to -\infty}x-3=-\infty  $ donc $ \displaystyle \lim_{x \to -\infty}f(x)=-\infty $ \\

$ \triangleright $ Calculons $\displaystyle\lim_{x \to +\infty}\frac{\sin x}{x} $.\\

Pour tout $ x\geq 1 $: $ -1\leq\sin x \leq 1 $ \\
En multipliant par $ \frac{1}{x} $: on a $ -\frac{1}{x}\leq \frac{\sin x}{x} \leq\frac{1}{x} $\\
Or $\displaystyle\lim_{x \to +\infty}(-\frac{1}{x})= \displaystyle\lim_{x \to +\infty}\frac{1}{x}=0$ donc $\displaystyle\lim_{x \to +\infty}\frac{\sin x}{x}=0 $
\end{example}


\subsubsection{Limites et nombre dérivé}

\begin{theorem}
Soit $ f $ une fonction dérivable .
Si $\displaystyle \lim_{x \to a}f^{'}(x)=l$,\;  ( $l$ réel fini ou pas) alors $\displaystyle\lim_{x \to a} \frac{f(x)-f(a)}{x-a}=l$
\end{theorem}

\begin{example}
Calculons $\displaystyle\lim_{x \to 0}\frac{\sin x}{x} $

Posons $ f(x)=\sin x $\\
On a $ f(0)=0 $ et  $ f^{'}(x)=\cos x$\\
 $ \displaystyle\lim_{x \to 0}f^{'}(x)=\displaystyle\lim_{x \to 0}\cos x=1 $ donc $\displaystyle \lim_{x \to 0} \frac{f(x)-f(0)}{x-0}=\displaystyle \lim_{x \to 0}\frac{\sin x}{x}=1$
\end{example}

\subsection{Branches infinies d'une courbe}
Soit une fonction numérique $ f $ et $ \mathcal{C} $ sa courbe représentative dans un repère orthogonal du plan.
\subsubsection{Asymptotes}
\subsubsection*{Asymptote verticale}
Elle traduit, graphiquement, le fait que la fonction $f$ admet une limite infinie en un réel $a$.


\begin{definition}
$\displaystyle\lim_{x \to a^{+}}f(x)=\infty$  ou 
$\displaystyle\lim_{x \to a^{-}}f(x)=\infty$ si et seulement si la droite  $x=a$ est une asymptote verticale à    ç la courbe $\mathcal{C} $.
\end{definition}



\subsubsection*{Asymptote horizontale}
Elle traduit, graphiquement, le fait que la fonction $f$ admet une limite finie en l'infini.



\begin{definition}
$\displaystyle \lim_{x \to \infty}f(x)=b \quad \text{ssi} \quad y = b \;  \text{est une  asymptote horizontale à  la courbe de }\; f \text{ en}  \infty$.
\end{definition}

\begin{example}
Pour la fonction $f : x\mapsto 3+ \frac{5}{x-1} $ 
la droite d'équation   $y = 3 $ est asymptote horizontale  la droite d'équation $x = 1$ est asymptote verticale à la courbe de $f.$

\end{example}

\subsubsection*{Asymptote oblique}
\begin{definition}
Soit $f$ une fonction et $ \Delta $ la droite d'équation  $y=mx+p$.

$\displaystyle \lim_{x \to \infty}\left(f(x)-(mx+p)\right)=0$   si et seulement si    $y=mx+p$  est une  asymptote oblique  à la courbe de $f$ en  $\infty$.
\end{definition}

\begin{remark} 
Si $ f $ s'écrit sous la forme $ f(x)= ax+b + g(x) $ et si $ \displaystyle \lim_{x \to \infty}g(x)=c $  alors la droite $ y=ax+b+c $ est une asymptote à  $ \mathcal{C} $ en $ \infty. $
\end{remark}
\begin{example}
Pour la fonction $f : x\mapsto 2x-5- \frac{2}{x-1} $ 
la droite d'équation   $y = 2x-5 $ est asymptote oblique à sa courbe de $f$ car  $\displaystyle \lim_{x \to \infty}\frac{2}{x-1}=0$.
\end{example}
\subsubsection*{Position relative d'une courbe et son asymptote}
Pour étudier la position relative de la  courbe $\mathcal{C} $ d'une fonction $ f $ et  de  son asymptote $ \Delta : y=mx+p $, on étudie le signe de la différence $ f(x)-mx-p$.
\begin{itemize}
\item  Si $ f(x)-mx-p > 0$ alors  $ \mathcal{C}  $ est  située  au-dessus de la courbe de $ \Delta $ 
\item  Si $ f(x)-mx-p< 0$ alors la courbe de $\mathcal{C}$ est  située  en-dessous de la courbe de $\Delta$ 
\end{itemize}
On tiendra compte de l'ensemble sur lequel on doit étudier la position relative des deux courbes: ainsi, la position relative d'une courbe et de son asymptote peut être étudiée sur l'ensemble de définition complet de la fonction ou simplement au voisinage de l'infini. 
\subsubsection{Recherche de branches infinies}
Lorsque  $\lim_{x \to \infty}f(x)=\infty  $, la courbe présente une branche infinie qu'il faut étudier.
\begin{itemize}
\item    Si $\displaystyle \lim_{x \to \infty}\frac{f(x)}{x}=0  $  alors la courbe $ \mathcal{C} $ présente une branche parabolique dans la direction  de  l'axe des abscisses.
\item    Si $\displaystyle \lim_{x \to \infty}\frac{f(x)}{x}=\infty$  alors la courbe $ \mathcal{C} $ présente une branche parabolique dans la direction  de  l'axe des ordonnées.
\item  Si  $\lim_{x \to \infty}\frac{f(x)}{x}=a  $  réel  non nul alors on calcule $\lim_{x \to \infty}(f(x)-ax )  $ 

 $ \bullet $ Si  $\displaystyle \lim_{x \to \infty}(f(x)-ax) = b $  réel alors la droite $(D)$ d'équation : $y = a x  + b $ est asymptote à la courbe $ \mathcal{C} $ .

 $ \bullet $ Si $\displaystyle \lim_{x \to \infty}(f(x)-ax) = \infty $     alors la courbe admet une branche parabolique de direction asymptotique la droite d'équation $ y =  a x $.
\end{itemize}
\begin{exercice}
On considère la fonction $ f $ définie par :

$ f (x)=\left\{\begin{array}{l} \sqrt{x +4}\quad \text{si} \quad x\geq 2 \\ x+3-\frac{2}{x-1}\quad \text{si}\quad x< 2  \end{array} \right.$

\begin{enumerate}
\item Déterminer  les limites aux bornes de $ D_{f} $.
\item Déterminer toutes les branches infinies de $ \mathcal{C} $.
\item Etudier la position relative de la courbe par rapport à son asymptote oblique.
\end{enumerate}
\end{exercice}
  %</content>
\end{document}
