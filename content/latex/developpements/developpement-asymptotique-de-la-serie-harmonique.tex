\documentclass[12pt, a4paper]{report}

% LuaLaTeX :

\RequirePackage{iftex}
\RequireLuaTeX

% Packages :

\usepackage[french]{babel}
%\usepackage[utf8]{inputenc}
%\usepackage[T1]{fontenc}
\usepackage[pdfencoding=auto, pdfauthor={Hugo Delaunay}, pdfsubject={Mathématiques}, pdfcreator={agreg.skyost.eu}]{hyperref}
\usepackage{amsmath}
\usepackage{amsthm}
%\usepackage{amssymb}
\renewcommand{\proofname}{Solution}
\usepackage{stmaryrd}
\usepackage{tikz}
\usepackage{tkz-euclide}
\usepackage{fontspec}
\defaultfontfeatures[Erewhon]{FontFace = {bx}{n}{Erewhon-Bold.otf}}
\usepackage{fourier-otf}
\usepackage[nobottomtitles*]{titlesec}
\usepackage{fancyhdr}
\usepackage{listings}
\usepackage{catchfilebetweentags}
\usepackage[french, capitalise, noabbrev]{cleveref}
\usepackage[fit, breakall]{truncate}
\usepackage[top=2.5cm, right=2cm, bottom=2.5cm, left=2cm]{geometry}
\usepackage{enumitem}
\usepackage{tablists} %Pour faire 1)  2) 3)
\usepackage{tocloft}
\usepackage{microtype}
%\usepackage{mdframed}
%\usepackage{thmtools}
\usepackage{xcolor}
\usepackage{tabularx}
\usepackage{xltabular}
\usepackage{aligned-overset}
\usepackage[subpreambles=true]{standalone}
\usepackage{environ}
\usepackage[normalem]{ulem}
\usepackage{multicol}
 \usepackage{variations}
\usepackage{array}% Pour faire des tableaux
\usepackage{etoolbox}
\usepackage{setspace}
\usepackage[bibstyle=reading, citestyle=draft]{biblatex}
\usepackage{xpatch}
\usepackage[many, breakable]{tcolorbox}
\usepackage[backgroundcolor=white, bordercolor=white, textsize=scriptsize]{todonotes}
\usepackage{luacode}
\usepackage{float}
\usepackage{needspace}


% Police :

\setmathfont{Erewhon Math}

% Tikz :

\usetikzlibrary{calc}
\usetikzlibrary{3d}

% Longueurs :

\setlength{\parindent}{0pt}
\setlength{\headheight}{15pt}
\setlength{\fboxsep}{0pt}
\titlespacing*{\chapter}{0pt}{-20pt}{10pt}
\setlength{\marginparwidth}{1.5cm}
\setstretch{1.1}

% Métadonnées :

\author{agreg.skyost.eu}
\date{\today}

% Titres :

\setcounter{secnumdepth}{3}

\renewcommand{\thechapter}{\Roman{chapter}}
\renewcommand{\thesubsection}{\Roman{subsection}}
\renewcommand{\thesubsubsection}{\arabic{subsubsection}}
\renewcommand{\theparagraph}{\alph{paragraph}}

\titleformat{\chapter}{\huge\bfseries}{\thechapter}{20pt}{\huge\bfseries}
\titleformat*{\section}{\LARGE\bfseries}
\titleformat{\subsection}{\Large\bfseries}{\thesubsection \, - \,}{0pt}{\Large\bfseries}
\titleformat{\subsubsection}{\large\bfseries}{\thesubsubsection. \,}{0pt}{\large\bfseries}
\titleformat{\paragraph}{\bfseries}{\theparagraph. \,}{0pt}{\bfseries}

\setcounter{secnumdepth}{4}

% Table des matières :

\renewcommand{\cftsecleader}{\cftdotfill{\cftdotsep}}
\addtolength{\cftsecnumwidth}{10pt}

% Redéfinition des commandes :

\renewcommand*\thesection{\arabic{section}}
\renewcommand{\ker}{\mathrm{Ker}}

% Nouvelles commandes :

\newcommand{\website}{http://sencoursdemaths.com}

\newcommand{\tr}[1]{\mathstrut ^t #1}
\newcommand{\im}{\mathrm{Im}}
\newcommand{\rang}{\operatorname{rang}}
\newcommand{\trace}{\operatorname{trace}}
\newcommand{\id}{\operatorname{id}}
\newcommand{\stab}{\operatorname{Stab}}
\newcommand{\paren}[1]{\left(#1\right)}
\newcommand{\accol}[1]{\left\{#1\right\}}
\newcommand{\croch}[1]{\left[ #1 \right]}
\newcommand{\Grdcroch}[1]{\Bigl[ #1 \Bigr]}
\newcommand{\grdcroch}[1]{\bigl[ #1 \bigr]}
\newcommand{\abs}[1]{\left\lvert #1 \right\rvert}
\newcommand{\limi}[3]{\displaystyle \lim_{#1\to #2}#3}
\newcommand{\pinf}{+\infty}
\newcommand{\minf}{-\infty}
%%%%%%%%%%%%%% ENSEMBLES %%%%%%%%%%%%%%%%%
\newcommand{\ensemblenombre}[1]{\mathbb{#1}}
\newcommand{\Nn}{\ensemblenombre{N}}
\newcommand{\Zz}{\ensemblenombre{Z}}
\newcommand{\Qq}{\ensemblenombre{Q}}
\newcommand{\Qqp}{\Qq^+}
\newcommand{\Rr}{\ensemblenombre{R}}
\newcommand{\Cc}{\ensemblenombre{C}}
\newcommand{\Nne}{\Nn^*}
\newcommand{\Zze}{\Zz^*}
\newcommand{\Zzn}{\Zz^-}
\newcommand{\Qqe}{\Qq^*}
\newcommand{\Rre}{\Rr^*}
\newcommand{\Rrp}{\Rr_+}
\newcommand{\Rrm}{\Rr_-}
\newcommand{\Rrep}{\Rr_+^*}
\newcommand{\Rrem}{\Rr_-^*}
\newcommand{\Cce}{\Cc^*}
%%%%%%%%%%%%%%  INTERVALLES %%%%%%%%%%%%%%%%%
\newcommand{\intff}[2]{\left[#1\;,\; #2\right]  }
\newcommand{\intof}[2]{\left]#1 \;, \;#2\right]  }
\newcommand{\intfo}[2]{\left[#1 \;,\; #2\right[  }
\newcommand{\intoo}[2]{\left]#1 \;,\; #2\right[  }



\providecommand{\newpar}{\\[\medskipamount]}

\newcommand{\annexessection}{%
  \newpage%
  \subsection*{Annexes}%
}

\providecommand{\lesson}[3]{%
  \title{#3}%
  \hypersetup{pdftitle={#2 : #3}}%
  \setcounter{section}{\numexpr #2 - 1}%
  \section{#3}%
  \fancyhead[R]{\truncate{0.73\textwidth}{#2 : #3}}%
}

\providecommand{\development}[3]{%
  \title{#3}%
  \hypersetup{pdftitle={#3}}%
  \section*{#3}%
  \fancyhead[R]{\truncate{0.73\textwidth}{#3}}%
}

\providecommand{\sheet}[3]{\development{#1}{#2}{#3}}

\providecommand{\ranking}[1]{%
  \title{Terminale #1}%
  \hypersetup{pdftitle={Terminale #1}}%
  \section*{Terminale #1}%
  \fancyhead[R]{\truncate{0.73\textwidth}{Terminale #1}}%
}

\providecommand{\summary}[1]{%
  \textit{#1}%
  \par%
  \medskip%
}

\tikzset{notestyleraw/.append style={inner sep=0pt, rounded corners=0pt, align=center}}

%\newcommand{\booklink}[1]{\website/bibliographie\##1}
\newcounter{reference}
\newcommand{\previousreference}{}
\providecommand{\reference}[2][]{%
  \needspace{20pt}%
  \notblank{#1}{
    \needspace{20pt}%
    \renewcommand{\previousreference}{#1}%
    \stepcounter{reference}%
    \label{reference-\previousreference-\thereference}%
  }{}%
  \todo[noline]{%
    \protect\vspace{20pt}%
    \protect\par%
    \protect\notblank{#1}{\cite{[\previousreference]}\\}{}%
    \protect\hyperref[reference-\previousreference-\thereference]{p. #2}%
  }%
}

\definecolor{devcolor}{HTML}{00695c}
\providecommand{\dev}[1]{%
  \reversemarginpar%
  \todo[noline]{
    \protect\vspace{20pt}%
    \protect\par%
    \bfseries\color{devcolor}\href{\website/developpements/#1}{[DEV]}
  }%
  \normalmarginpar%
}

% En-têtes :

\pagestyle{fancy}
\fancyhead[L]{\truncate{0.23\textwidth}{\thepage}}
\fancyfoot[C]{\scriptsize \href{\website}{\texttt{http://sencoursdemaths.com}}}

% Couleurs :

\definecolor{property}{HTML}{ffeb3b}
\definecolor{proposition}{HTML}{ffc107}
\definecolor{lemma}{HTML}{ff9800}
\definecolor{theorem}{HTML}{f44336}
\definecolor{corollary}{HTML}{e91e63}
\definecolor{definition}{HTML}{673ab7}
\definecolor{notation}{HTML}{9c27b0}
\definecolor{example}{HTML}{00bcd4}
\definecolor{cexample}{HTML}{795548}
\definecolor{application}{HTML}{009688}
\definecolor{remark}{HTML}{3f51b5}
\definecolor{algorithm}{HTML}{607d8b}
\definecolor{proof}{HTML}{e1f5fe}
\definecolor{exercice}{HTML}{e1f5fe}

% Théorèmes :

\theoremstyle{definition}
\newtheorem{theorem}{Théorème}

\newtheorem{property}[theorem]{Propriété}
\newtheorem{proposition}[theorem]{Proposition}
\newtheorem{lemma}[theorem]{Activité d'introduction}
\newtheorem{corollary}[theorem]{Conséquence}

\newtheorem{definition}[theorem]{Définition}
\newtheorem{notation}[theorem]{Notation}

\newtheorem{example}[theorem]{Exemple}
\newtheorem{cexample}[theorem]{Contre-exemple}
\newtheorem{application}[theorem]{Application}

\newtheorem{algorithm}[theorem]{Algorithme}
\newtheorem{exercice}[theorem]{Exercice}

\theoremstyle{remark}
\newtheorem{remark}[theorem]{Remarque}




\counterwithin*{theorem}{section}

\newcommand{\applystyletotheorem}[1]{
  \tcolorboxenvironment{#1}{
    enhanced,
    breakable,
    colback=#1!8!white,
    %right=0pt,
    %top=8pt,
    %bottom=8pt,
    boxrule=0pt,
    frame hidden,
    sharp corners,
    enhanced,borderline west={4pt}{0pt}{#1},
    %interior hidden,
    sharp corners,
    after=\par,
  }
}

\applystyletotheorem{property}
\applystyletotheorem{proposition}
\applystyletotheorem{lemma}
\applystyletotheorem{theorem}
\applystyletotheorem{corollary}
\applystyletotheorem{definition}
\applystyletotheorem{notation}
\applystyletotheorem{example}
\applystyletotheorem{cexample}
\applystyletotheorem{application}
\applystyletotheorem{remark}
%\applystyletotheorem{proof}
\applystyletotheorem{algorithm}
\applystyletotheorem{exercice}

% Environnements :

\NewEnviron{whitetabularx}[1]{%
  \renewcommand{\arraystretch}{2.5}
  \colorbox{white}{%
    \begin{tabularx}{\textwidth}{#1}%
      \BODY%
    \end{tabularx}%
  }%
}

% Maths :

\DeclareFontEncoding{FMS}{}{}
\DeclareFontSubstitution{FMS}{futm}{m}{n}
\DeclareFontEncoding{FMX}{}{}
\DeclareFontSubstitution{FMX}{futm}{m}{n}
\DeclareSymbolFont{fouriersymbols}{FMS}{futm}{m}{n}
\DeclareSymbolFont{fourierlargesymbols}{FMX}{futm}{m}{n}
\DeclareMathDelimiter{\VERT}{\mathord}{fouriersymbols}{152}{fourierlargesymbols}{147}

% Code :

\definecolor{greencode}{rgb}{0,0.6,0}
\definecolor{graycode}{rgb}{0.5,0.5,0.5}
\definecolor{mauvecode}{rgb}{0.58,0,0.82}
\definecolor{bluecode}{HTML}{1976d2}
\lstset{
  basicstyle=\footnotesize\ttfamily,
  breakatwhitespace=false,
  breaklines=true,
  %captionpos=b,
  commentstyle=\color{greencode},
  deletekeywords={...},
  escapeinside={\%*}{*)},
  extendedchars=true,
  frame=none,
  keepspaces=true,
  keywordstyle=\color{bluecode},
  language=Python,
  otherkeywords={*,...},
  numbers=left,
  numbersep=5pt,
  numberstyle=\tiny\color{graycode},
  rulecolor=\color{black},
  showspaces=false,
  showstringspaces=false,
  showtabs=false,
  stepnumber=2,
  stringstyle=\color{mauvecode},
  tabsize=2,
  %texcl=true,
  xleftmargin=10pt,
  %title=\lstname
}

\newcommand{\codedirectory}{}
\newcommand{\inputalgorithm}[1]{%
  \begin{algorithm}%
    \strut%
    \lstinputlisting{\codedirectory#1}%
  \end{algorithm}%
}





\begin{document}
  %<*content>
  \development{algebra}{developpement-asymptotique-de-la-serie-harmonique}{Développement asymptotique de la série harmonique}

  \summary{On effectue un développement asymptotique à l'ordre $2$ de la série harmonique $\sum \frac{1}{n}$.}

  \reference[I-P]{380}

  \begin{lemma}
    \label{developpement-asymptotique-de-la-serie-harmonique-1}
    Soit $\alpha > 1$. Lorsque $n$ tend vers $+\infty$, on a
    \[ \sum_{k=n+1}^{+\infty} \frac{1}{n^\alpha} \sim \frac{1}{\alpha - 1} \frac{1}{n^{\alpha - 1}} \]
  \end{lemma}

  \begin{proof}
    La fonction $x \mapsto \frac{1}{x}^\alpha$ est décroissante sur $\mathbb{R}^+_*$, nous allons faire une comparaison série / intégrale.
    \begin{center}
      \begin{tikzpicture}[scale=1.5]
        %\clip(-0.5,-0.5) rectangle (2.5,2.5);
        \coordinate (A) at (1,0);
        \coordinate (B) at (2,0);
        \coordinate (C) at (1,1);
        \coordinate (D) at (2,0.25);
        \coordinate (E) at (0,1);
        \coordinate (F) at (0,0.25);
        \filldraw[cyan!50] (B) -- (A) -- (C) -- (2,1) -- cycle;
        \filldraw[cyan!80] (B) -- (A) -- (1,0.25) -- (D) -- cycle;
        \draw[domain=0.625:2.5, smooth, variable=\x, cyan] plot ({\x}, {1/(\x*\x)});
        \node (G) at (4,2) {\color{cyan!50} Aire du rectangle égale à $\frac{1}{k^\alpha}$};
        \node (H) at (1.5,-1) {\color{cyan!80} Aire du rectangle égale à $\frac{1}{(k+1)^\alpha}$};
        \draw[cyan, ->] ($(G)-(0,0.2)$) to [out=-120,in=80] (1.75,0.75);
        \draw[cyan, ->] ($(H)+(0,0.2)$) to [out=120,in=-90] (1.5,0.125);
        \node at (A) {$\bullet$};
        \node at (A) [below] {$k$};
        \node at (B) {$\bullet$};
        \node at (B) [below] {$k+1$};
        \node at (C) {$\bullet$};
        \node at (D) {$\bullet$};
        \node at (E) {$\bullet$};
        \node at (E) [left] {$\frac{1}{k^\alpha}$};
        \node at (F) {$\bullet$};
        \node at (F) [left] {$\frac{1}{(k+1)^\alpha}$};
        \node at (0.7,2.5) [right] {\color{cyan} $y = \frac{1}{x^\alpha}$};
        \draw[dashed] (B) -- (D) -- (F);
        \draw[dashed] (A) -- (C) -- (E);
        \draw[->] (-0.5,0) -- (2.5,0);
        \draw[->] (0,-0.5) -- (0,2.5);
      \end{tikzpicture}
    \end{center}
    On a
    \[ \forall k \geq 1, \, \frac{1}{(k+1)^\alpha} \leq \int_k^{k+1} \frac{1}{x^\alpha} \, \mathrm{d}x \leq \frac{1}{k^\alpha} \]
    D'où :
    \[ \forall k \geq 2, \, \int_k^{k+1} \frac{1}{x^\alpha} \, \mathrm{d}x \leq \frac{1}{k^\alpha} \leq \int_{k-1}^k \frac{1}{x^\alpha} \, \mathrm{d}x \]
    Soit $N \geq 2$. Pour tout $n \in \llbracket 2, N \rrbracket$,
    \begin{align*}
      &\int_n^{N+1} \frac{1}{x^\alpha} \, \mathrm{d}x \leq \sum_{k=n}^N \frac{1}{k^\alpha} \leq \int_{n-1}^N \frac{1}{x^\alpha} \, \mathrm{d}x \\
      \iff& \left[ \frac{-1}{\alpha - 1} \frac{1}{x^{\alpha - 1}} \right]^{N+1}_n \leq \sum_{k=n}^N \frac{1}{k^\alpha} \leq \left[ \frac{-1}{\alpha - 1} \frac{1}{x^{\alpha - 1}} \right]^N_{n-1} \\
      \iff& \frac{1}{\alpha - 1} \left( \frac{1}{n^{\alpha - 1}} - \frac{1}{(N+1)^{\alpha - 1}} \right) \leq \sum_{k=n}^N \frac{1}{k^\alpha} \leq \frac{1}{\alpha - 1} \left( \frac{1}{(n-1)^{\alpha - 1}} - \frac{1}{N^{\alpha - 1}} \right)
    \end{align*}
    La suite $\left(\sum_{k=n}^N \frac{1}{k^\alpha} \right)$ est donc convergente, car elle est croissante et majorée par $\frac{1}{\alpha - 1} \left( \frac{1}{(n-1)^{\alpha - 1}} \right)$. Lorsque $N$ tend vers $+\infty$, on a donc
    \[ \frac{1}{\alpha - 1} \left( \frac{1}{n^{\alpha - 1}} \right) \leq \sum_{k=n}^{+\infty} \frac{1}{k^\alpha} \leq \frac{1}{\alpha - 1} \left( \frac{1}{(n-1)^{\alpha - 1}} \right) \]
    Or, comme $n^{\alpha - 1} \sim (n-1)^{\alpha - 1}$ quand $n$ tend vers $+\infty$, on en conclut l'équivalent annoncé.
  \end{proof}

  \begin{theorem}[Développement asymptotique de la série harmonique]
    On note $\forall n \in \mathbb{N}^*, \, H_n = \sum_{k=1}^{n} \frac{1}{k}$. Alors, quand $n$ tend vers $+\infty$,
    \[ H_n = \ln(n) + \gamma + \frac{1}{2n} - \frac{1}{12n^2} + o\left( \frac{1}{n^2} \right) \]
  \end{theorem}

  \begin{proof}
    La fonction $x \mapsto \frac{1}{x}$ est décroissante sur $\mathbb{R}^+_*$, cela invite à faire une comparaison série / intégrale.
    \begin{center}
      \begin{tikzpicture}[scale=1.5]
        %\clip(-0.5,-0.5) rectangle (2.5,2.5);
        \coordinate (A) at (1,0);
        \coordinate (B) at (2,0);
        \coordinate (C) at (1,1);
        \coordinate (D) at (2,0.5);
        \coordinate (E) at (0,1);
        \coordinate (F) at (0,0.5);
        \filldraw[teal!50] (B) -- (A) -- (C) -- (2,1) -- cycle;
        \filldraw[teal!80] (B) -- (A) -- (1,0.5) -- (D) -- cycle;
        \draw[domain=0.4:2.5, smooth, variable=\x, teal] plot ({\x}, {1/\x});
        \node (G) at (4,2) {\color{teal!50} Aire du rectangle égale à $\frac{1}{k}$};
        \node (H) at (1.5,-1) {\color{teal!80} Aire du rectangle égale à $\frac{1}{k+1}$};
        \draw[teal, ->] ($(G)-(0,0.2)$) to [out=-120,in=80] (1.75,0.75);
        \draw[teal, ->] ($(H)+(0,0.2)$) to [out=120,in=-90] (1.5,0.25);
        \node at (A) {$\bullet$};
        \node at (A) [below] {$k$};
        \node at (B) {$\bullet$};
        \node at (B) [below] {$k+1$};
        \node at (C) {$\bullet$};
        \node at (D) {$\bullet$};
        \node at (E) {$\bullet$};
        \node at (E) [left] {$\frac{1}{k}$};
        \node at (F) {$\bullet$};
        \node at (F) [left] {$\frac{1}{k+1}$};
        \node at (0.4,2.5) [right] {\color{teal} $y = \frac{1}{x}$};
        \draw[dashed] (B) -- (D) -- (F);
        \draw[dashed] (A) -- (C) -- (E);
        \draw[->] (-0.5,0) -- (2.5,0);
        \draw[->] (0,-0.5) -- (0,2.5);
      \end{tikzpicture}
    \end{center}
    On a
    \[ \forall k \geq 1, \, \frac{1}{k+1} \leq \int_k^{k+1} \frac{1}{x} \, \mathrm{d}x \leq \frac{1}{k} \]
    Traitons les deux morceaux séparément.
    \begin{itemize}
      \item $\forall k \geq 1, \, \int_k^{k+1} \frac{1}{x} \, \mathrm{d}x \leq \frac{1}{k}$ par l'inégalité de droite. Donc, en sommant entre $1$ et $n \in \mathbb{N}^*$ :
      \[ \ln(n+1) = \int_{1}^{n+1} \frac{1}{x} \, \mathrm{d}x \leq H_n \]
      \item $\forall k \geq 2, \, \frac{1}{k} \leq \int_{k-1}^k \frac{1}{x} \, \mathrm{d}x$ par l'inégalité de gauche avec un changement de variable. Donc, en sommant entre $2$ et $n \in \mathbb{N}^*$ :
      \[ \sum_{k=2}^n \frac{1}{k} \leq \int_1^n \frac{1}{x} \, \mathrm{d}x = \ln(n) \]
      et en ajoutant $1$ :
      \[ H_n \leq \ln(n) + 1 \]
    \end{itemize}
    On peut tout regrouper pour obtenir les inégalités suivantes :
    \[ \ln(n+1) \leq H_n \leq \ln(n) + 1 \]
    et donc, quand $n$ tend vers $+\infty$,
    \[ H_n \sim \ln(n) \]
    Pour la suite, on pose pour tout $n \geq 1$, $u_n = H_n - \ln(n)$ et pour tout $n \geq 2$, $v_n = H_{n-1} - \ln(n)$. On a :
    \begin{itemize}
      \item $\forall n \geq 2$, $u_n - v_n = \frac{1}{n} \geq 0$ et converge vers $0$ quand $n$ tend vers $+\infty$.
      \item $\forall n \geq 1$,
      \begin{align*}
        u_n - u_{n+1} &= -\frac{1}{n+1} - \ln(n) + \ln(n+1) \\
        &= -\frac{1}{n+1} - \ln \left( 1 - \frac{1}{n+1} \right) \\
        &\geq 0
      \end{align*}
      car $\ln(1+x) \leq x$ pour $x \in ]-1, +\infty[$.
      \item $\forall n \geq 2$,
      \begin{align*}
        v_{n+1} - v_n &= \frac{1}{n} + \ln(n) - \ln(n+1) \\
        &= \frac{1}{n} - \ln \left( 1 + \frac{1}{n} \right) \\
        &\geq 0
      \end{align*}
    \end{itemize}
    les suites $(u_n)$ et $(v_n)$ sont adjacentes, elles convergent donc vers un réel $\gamma \in \mathbb{R}$. Posons maintenant
    \[ \forall n \geq 1, \, t_n = u_n - \gamma = H_n - \ln(n) - \gamma \]
    Nous allons utiliser le lien entre séries et suites : cherchons un équivalente de la suite $(t_n - t_{n-1})$ pour obtenir un équivalent de la somme partielle de la série de terme général $(t_n - t_{n-1})$ qui n'est autre que la suite $(t_n)$. À l'aide du développement limité de $\ln(1+x)$ en $0$ on obtient
    \begin{align*}
      t_n - t_{n-1} &= \ln(n-1) - \ln(n) + \frac{1}{n} \\
      &= \ln \left( 1 - \frac{1}{n} \right) + \frac{1}{n} \\
      &\sim -\frac{1}{2n^2}
    \end{align*}
    D'après le critère de Riemann, la série de terme général $t_k - t_{k-1}$ converge. Le théorème de sommation des équivalents donne l'équivalence des restes. Or, un équivalent du reste de la série de Riemann $\sum \frac{1}{n^2}$ est donné par le \cref{developpement-asymptotique-de-la-serie-harmonique-1} et vaut $\frac{1}{n}$ :
    \[ \sum_{k=n+1}^{+\infty} t_k - t_{k-1} = -t_n \sim \sum_{k=n+1}^{+\infty} -\frac{1}{2k^2} \sim -\frac{1}{2n} \]
    D'où $t_n \sim \frac{1}{2n}$ et $H_n = \ln(n) + \gamma + \frac{1}{2n} + o\left( \frac{1}{n} \right)$. On pose alors $\forall n \geq 1$, $w_n = t_n - \frac{1}{2n}$ et on procède de manière similaire pour obtenir, pour tout $n \geq 2$ :
    \begin{align*}
      w_n - w_{n-1} &= \frac{1}{n} + \ln \left( 1 - \frac{1}{n} \right) + \frac{1}{2n-2} - \frac{1}{2n} \\
      &= \frac{1}{n} - \frac{1}{n} - \frac{1}{2n^2} - \frac{1}{3n^3} + \frac{1}{2n} \frac{1}{1 - \frac{1}{n}} - \frac{1}{2n} + o\left( \frac{1}{n^3} \right) \\
      &= - \frac{1}{2n^2} + \frac{1}{2n} \left( 1 + \frac{1}{n} + \frac{1}{n^2} \right) - \frac{1}{2n} + o\left( \frac{1}{n^3} \right) \\
      &= \frac{1}{6n^3} + o\left( \frac{1}{n^3} \right)
    \end{align*}
    On a donc
    \[ \sum_{k=n+1}^{+\infty} w_k - w_{k-1} = -w_n \sim \frac{1}{2} \frac{1}{6n^2} = \frac{1}{12n^2} \]
    d'où le résultat.
  \end{proof}
  %</content>
\end{document}
