\documentclass[12pt, a4paper]{report}

% LuaLaTeX :

\RequirePackage{iftex}
\RequireLuaTeX

% Packages :

\usepackage[french]{babel}
%\usepackage[utf8]{inputenc}
%\usepackage[T1]{fontenc}
\usepackage[pdfencoding=auto, pdfauthor={Hugo Delaunay}, pdfsubject={Mathématiques}, pdfcreator={agreg.skyost.eu}]{hyperref}
\usepackage{amsmath}
\usepackage{amsthm}
%\usepackage{amssymb}
\renewcommand{\proofname}{Solution}
\usepackage{stmaryrd}
\usepackage{tikz}
\usepackage{tkz-euclide}
\usepackage{fontspec}
\defaultfontfeatures[Erewhon]{FontFace = {bx}{n}{Erewhon-Bold.otf}}
\usepackage{fourier-otf}
\usepackage[nobottomtitles*]{titlesec}
\usepackage{fancyhdr}
\usepackage{listings}
\usepackage{catchfilebetweentags}
\usepackage[french, capitalise, noabbrev]{cleveref}
\usepackage[fit, breakall]{truncate}
\usepackage[top=2.5cm, right=2cm, bottom=2.5cm, left=2cm]{geometry}
\usepackage{enumitem}
\usepackage{tablists} %Pour faire 1)  2) 3)
\usepackage{tocloft}
\usepackage{microtype}
%\usepackage{mdframed}
%\usepackage{thmtools}
\usepackage{xcolor}
\usepackage{tabularx}
\usepackage{xltabular}
\usepackage{aligned-overset}
\usepackage[subpreambles=true]{standalone}
\usepackage{environ}
\usepackage[normalem]{ulem}
\usepackage{multicol}
 \usepackage{variations}
\usepackage{array}% Pour faire des tableaux
\usepackage{etoolbox}
\usepackage{setspace}
\usepackage[bibstyle=reading, citestyle=draft]{biblatex}
\usepackage{xpatch}
\usepackage[many, breakable]{tcolorbox}
\usepackage[backgroundcolor=white, bordercolor=white, textsize=scriptsize]{todonotes}
\usepackage{luacode}
\usepackage{float}
\usepackage{needspace}


% Police :

\setmathfont{Erewhon Math}

% Tikz :

\usetikzlibrary{calc}
\usetikzlibrary{3d}

% Longueurs :

\setlength{\parindent}{0pt}
\setlength{\headheight}{15pt}
\setlength{\fboxsep}{0pt}
\titlespacing*{\chapter}{0pt}{-20pt}{10pt}
\setlength{\marginparwidth}{1.5cm}
\setstretch{1.1}

% Métadonnées :

\author{agreg.skyost.eu}
\date{\today}

% Titres :

\setcounter{secnumdepth}{3}

\renewcommand{\thechapter}{\Roman{chapter}}
\renewcommand{\thesubsection}{\Roman{subsection}}
\renewcommand{\thesubsubsection}{\arabic{subsubsection}}
\renewcommand{\theparagraph}{\alph{paragraph}}

\titleformat{\chapter}{\huge\bfseries}{\thechapter}{20pt}{\huge\bfseries}
\titleformat*{\section}{\LARGE\bfseries}
\titleformat{\subsection}{\Large\bfseries}{\thesubsection \, - \,}{0pt}{\Large\bfseries}
\titleformat{\subsubsection}{\large\bfseries}{\thesubsubsection. \,}{0pt}{\large\bfseries}
\titleformat{\paragraph}{\bfseries}{\theparagraph. \,}{0pt}{\bfseries}

\setcounter{secnumdepth}{4}

% Table des matières :

\renewcommand{\cftsecleader}{\cftdotfill{\cftdotsep}}
\addtolength{\cftsecnumwidth}{10pt}

% Redéfinition des commandes :

\renewcommand*\thesection{\arabic{section}}
\renewcommand{\ker}{\mathrm{Ker}}

% Nouvelles commandes :

\newcommand{\website}{http://sencoursdemaths.com}

\newcommand{\tr}[1]{\mathstrut ^t #1}
\newcommand{\im}{\mathrm{Im}}
\newcommand{\rang}{\operatorname{rang}}
\newcommand{\trace}{\operatorname{trace}}
\newcommand{\id}{\operatorname{id}}
\newcommand{\stab}{\operatorname{Stab}}
\newcommand{\paren}[1]{\left(#1\right)}
\newcommand{\accol}[1]{\left\{#1\right\}}
\newcommand{\croch}[1]{\left[ #1 \right]}
\newcommand{\Grdcroch}[1]{\Bigl[ #1 \Bigr]}
\newcommand{\grdcroch}[1]{\bigl[ #1 \bigr]}
\newcommand{\abs}[1]{\left\lvert #1 \right\rvert}
\newcommand{\limi}[3]{\displaystyle \lim_{#1\to #2}#3}
\newcommand{\pinf}{+\infty}
\newcommand{\minf}{-\infty}
%%%%%%%%%%%%%% ENSEMBLES %%%%%%%%%%%%%%%%%
\newcommand{\ensemblenombre}[1]{\mathbb{#1}}
\newcommand{\Nn}{\ensemblenombre{N}}
\newcommand{\Zz}{\ensemblenombre{Z}}
\newcommand{\Qq}{\ensemblenombre{Q}}
\newcommand{\Qqp}{\Qq^+}
\newcommand{\Rr}{\ensemblenombre{R}}
\newcommand{\Cc}{\ensemblenombre{C}}
\newcommand{\Nne}{\Nn^*}
\newcommand{\Zze}{\Zz^*}
\newcommand{\Zzn}{\Zz^-}
\newcommand{\Qqe}{\Qq^*}
\newcommand{\Rre}{\Rr^*}
\newcommand{\Rrp}{\Rr_+}
\newcommand{\Rrm}{\Rr_-}
\newcommand{\Rrep}{\Rr_+^*}
\newcommand{\Rrem}{\Rr_-^*}
\newcommand{\Cce}{\Cc^*}
%%%%%%%%%%%%%%  INTERVALLES %%%%%%%%%%%%%%%%%
\newcommand{\intff}[2]{\left[#1\;,\; #2\right]  }
\newcommand{\intof}[2]{\left]#1 \;, \;#2\right]  }
\newcommand{\intfo}[2]{\left[#1 \;,\; #2\right[  }
\newcommand{\intoo}[2]{\left]#1 \;,\; #2\right[  }



\providecommand{\newpar}{\\[\medskipamount]}

\newcommand{\annexessection}{%
  \newpage%
  \subsection*{Annexes}%
}

\providecommand{\lesson}[3]{%
  \title{#3}%
  \hypersetup{pdftitle={#2 : #3}}%
  \setcounter{section}{\numexpr #2 - 1}%
  \section{#3}%
  \fancyhead[R]{\truncate{0.73\textwidth}{#2 : #3}}%
}

\providecommand{\development}[3]{%
  \title{#3}%
  \hypersetup{pdftitle={#3}}%
  \section*{#3}%
  \fancyhead[R]{\truncate{0.73\textwidth}{#3}}%
}

\providecommand{\sheet}[3]{\development{#1}{#2}{#3}}

\providecommand{\ranking}[1]{%
  \title{Terminale #1}%
  \hypersetup{pdftitle={Terminale #1}}%
  \section*{Terminale #1}%
  \fancyhead[R]{\truncate{0.73\textwidth}{Terminale #1}}%
}

\providecommand{\summary}[1]{%
  \textit{#1}%
  \par%
  \medskip%
}

\tikzset{notestyleraw/.append style={inner sep=0pt, rounded corners=0pt, align=center}}

%\newcommand{\booklink}[1]{\website/bibliographie\##1}
\newcounter{reference}
\newcommand{\previousreference}{}
\providecommand{\reference}[2][]{%
  \needspace{20pt}%
  \notblank{#1}{
    \needspace{20pt}%
    \renewcommand{\previousreference}{#1}%
    \stepcounter{reference}%
    \label{reference-\previousreference-\thereference}%
  }{}%
  \todo[noline]{%
    \protect\vspace{20pt}%
    \protect\par%
    \protect\notblank{#1}{\cite{[\previousreference]}\\}{}%
    \protect\hyperref[reference-\previousreference-\thereference]{p. #2}%
  }%
}

\definecolor{devcolor}{HTML}{00695c}
\providecommand{\dev}[1]{%
  \reversemarginpar%
  \todo[noline]{
    \protect\vspace{20pt}%
    \protect\par%
    \bfseries\color{devcolor}\href{\website/developpements/#1}{[DEV]}
  }%
  \normalmarginpar%
}

% En-têtes :

\pagestyle{fancy}
\fancyhead[L]{\truncate{0.23\textwidth}{\thepage}}
\fancyfoot[C]{\scriptsize \href{\website}{\texttt{http://sencoursdemaths.com}}}

% Couleurs :

\definecolor{property}{HTML}{ffeb3b}
\definecolor{proposition}{HTML}{ffc107}
\definecolor{lemma}{HTML}{ff9800}
\definecolor{theorem}{HTML}{f44336}
\definecolor{corollary}{HTML}{e91e63}
\definecolor{definition}{HTML}{673ab7}
\definecolor{notation}{HTML}{9c27b0}
\definecolor{example}{HTML}{00bcd4}
\definecolor{cexample}{HTML}{795548}
\definecolor{application}{HTML}{009688}
\definecolor{remark}{HTML}{3f51b5}
\definecolor{algorithm}{HTML}{607d8b}
\definecolor{proof}{HTML}{e1f5fe}
\definecolor{exercice}{HTML}{e1f5fe}

% Théorèmes :

\theoremstyle{definition}
\newtheorem{theorem}{Théorème}

\newtheorem{property}[theorem]{Propriété}
\newtheorem{proposition}[theorem]{Proposition}
\newtheorem{lemma}[theorem]{Activité d'introduction}
\newtheorem{corollary}[theorem]{Conséquence}

\newtheorem{definition}[theorem]{Définition}
\newtheorem{notation}[theorem]{Notation}

\newtheorem{example}[theorem]{Exemple}
\newtheorem{cexample}[theorem]{Contre-exemple}
\newtheorem{application}[theorem]{Application}

\newtheorem{algorithm}[theorem]{Algorithme}
\newtheorem{exercice}[theorem]{Exercice}

\theoremstyle{remark}
\newtheorem{remark}[theorem]{Remarque}




\counterwithin*{theorem}{section}

\newcommand{\applystyletotheorem}[1]{
  \tcolorboxenvironment{#1}{
    enhanced,
    breakable,
    colback=#1!8!white,
    %right=0pt,
    %top=8pt,
    %bottom=8pt,
    boxrule=0pt,
    frame hidden,
    sharp corners,
    enhanced,borderline west={4pt}{0pt}{#1},
    %interior hidden,
    sharp corners,
    after=\par,
  }
}

\applystyletotheorem{property}
\applystyletotheorem{proposition}
\applystyletotheorem{lemma}
\applystyletotheorem{theorem}
\applystyletotheorem{corollary}
\applystyletotheorem{definition}
\applystyletotheorem{notation}
\applystyletotheorem{example}
\applystyletotheorem{cexample}
\applystyletotheorem{application}
\applystyletotheorem{remark}
%\applystyletotheorem{proof}
\applystyletotheorem{algorithm}
\applystyletotheorem{exercice}

% Environnements :

\NewEnviron{whitetabularx}[1]{%
  \renewcommand{\arraystretch}{2.5}
  \colorbox{white}{%
    \begin{tabularx}{\textwidth}{#1}%
      \BODY%
    \end{tabularx}%
  }%
}

% Maths :

\DeclareFontEncoding{FMS}{}{}
\DeclareFontSubstitution{FMS}{futm}{m}{n}
\DeclareFontEncoding{FMX}{}{}
\DeclareFontSubstitution{FMX}{futm}{m}{n}
\DeclareSymbolFont{fouriersymbols}{FMS}{futm}{m}{n}
\DeclareSymbolFont{fourierlargesymbols}{FMX}{futm}{m}{n}
\DeclareMathDelimiter{\VERT}{\mathord}{fouriersymbols}{152}{fourierlargesymbols}{147}

% Code :

\definecolor{greencode}{rgb}{0,0.6,0}
\definecolor{graycode}{rgb}{0.5,0.5,0.5}
\definecolor{mauvecode}{rgb}{0.58,0,0.82}
\definecolor{bluecode}{HTML}{1976d2}
\lstset{
  basicstyle=\footnotesize\ttfamily,
  breakatwhitespace=false,
  breaklines=true,
  %captionpos=b,
  commentstyle=\color{greencode},
  deletekeywords={...},
  escapeinside={\%*}{*)},
  extendedchars=true,
  frame=none,
  keepspaces=true,
  keywordstyle=\color{bluecode},
  language=Python,
  otherkeywords={*,...},
  numbers=left,
  numbersep=5pt,
  numberstyle=\tiny\color{graycode},
  rulecolor=\color{black},
  showspaces=false,
  showstringspaces=false,
  showtabs=false,
  stepnumber=2,
  stringstyle=\color{mauvecode},
  tabsize=2,
  %texcl=true,
  xleftmargin=10pt,
  %title=\lstname
}

\newcommand{\codedirectory}{}
\newcommand{\inputalgorithm}[1]{%
  \begin{algorithm}%
    \strut%
    \lstinputlisting{\codedirectory#1}%
  \end{algorithm}%
}




% Bibliographie :

%\addbibresource{\bibliographypath}%
\defbibheading{bibliography}[\bibname]{\section*{#1}}
\renewbibmacro*{entryhead:full}{\printfield{labeltitle}}%
\DeclareFieldFormat{url}{\newline\footnotesize\url{#1}}%

\AtEndDocument{%
  \newpage%
  \pagestyle{empty}%
  \printbibliography%
}


\begin{document}
  %<*content>
  \development{algebra}{theoreme-de-sylow}{Premier théorème de Sylow}

  \summary{En procédant par récurrence sur le cardinal du groupe, on montre l'existence d'un sous-groupe de Sylow.}

  \reference[GOU21]{44}

  \begin{theorem}[Cauchy ``faible'']
    \label{theoreme-de-sylow-1}
    Soit $G$ un groupe abélien fini et soit $p$ un diviseur premier de l'ordre de $G$. Alors, il existe un sous-groupe de $G$ d'ordre $p$.
  \end{theorem}

  \begin{proof}
    $G$ est fini, on peut donc l'écrire
    \[ G = \langle x_1, \dots, x_n \rangle \]
    où $(x_1, \dots, x_n)$ est un système de générateurs de $G$. On définit
    \[
      \varphi :
      \begin{array}{ccc}
        \langle x_1 \rangle \times \dots \times \langle x_n \rangle &\rightarrow& G \\
        (y_1, \dots, y_n) &\mapsto& y_1 \dots y_n
      \end{array}
    \]
    Comme $G$ est abélien, $\varphi$ est clairement un morphisme de groupes. Et comme $(x_1, \dots, x_n)$  est un système de générateurs de $G$, $\varphi$ est surjectif.
    On peut appliquer le premier théorème d'isomorphisme pour obtenir
    \[ G \cong (\langle x_1 \rangle \times \dots \times \langle x_n \rangle) / \ker(\varphi) \]
    En particulier, $\vert G \vert \times \vert \ker(\varphi) \vert = \vert \langle x_1 \rangle \vert \times \dots \times \vert \langle x_n \rangle \vert$. On note, pour tout $i \in \llbracket 1, n \rrbracket$, $r_i = \vert \langle x_i \rangle \vert$. On a ainsi,
    \[ G \mid r_1 \dots r_n \implies p \mid r_1 \dots r_n \]
    par transitivité de $\mid$. Par le lemme d'Euclide, il existe $i \in \llbracket 1, n \rrbracket$ tel que $p \mid r_i$. On écrit $r_i = pq$ avec $q \in \mathbb{N}^*$, et on pose $x = x_i^q$. Alors, $x$ est d'ordre $p$ et $H = \langle x \rangle$ est un sous-groupe de $G$ d'ordre $p$.
  \end{proof}

  \begin{theorem}[Premier théorème de Sylow]
    Soit $G$ un groupe fini d'ordre $n p^\alpha$ avec $n, \alpha \in \mathbb{N}$ et $p$ premier tel que $p \nmid n$. Alors, il existe un sous-groupe de $G$ d’ordre $p^\alpha$.
  \end{theorem}

  \begin{proof}
    Posons $h = \vert G \vert$. On va procéder par récurrence forte sur $h$.
    \begin{itemize}
      \item \uline{Si $h$ = 1 :} Alors, $n = 1$ et $\alpha = 0$. La propriété est donc triviale.
      \item \uline{On suppose la propriété vraie pour les groupes d'ordre strictement inférieur à $h$.} Si $\alpha = 0$, c'est encore une fois trivial, pour les mêmes raisons qu'à l'initialisation de la propriété. Supposons donc $\alpha \geq 1$. On fait agir $G$ sur lui-même par conjugaison, via l'action :
      \[ (g,h) \mapsto ghg^{-1} \]
      Soit $\Omega$ un système de représentants associé à la relation ``être dans la même orbite''. La formule des classes donne
      \[ |G| = \sum_{\omega \in \Omega} |G \cdot \omega| = \sum_{\omega \in \Omega} (G : \stab_G(\omega)) = \sum_{\omega \in \Omega} \frac{|G|}{|\stab_G(\omega)|} \tag{$*$} \]
      Mais,
      \[ \stab_G(\omega) = G \iff \forall g \in G, \, g \omega g^{-1} = \omega \iff \omega \in Z(G) \]
      donc, en regroupant, on peut réécrire $(*)$ :
      \begin{align*}
        |G| &= \sum_{\omega \in \Omega} \frac{|G|}{|\stab_G(\omega)|} \\
        & = \sum_{\omega \in Z(G)} \frac{|G|}{|\stab_G(\omega)|} + \sum_{\omega \notin Z(G)} \frac{|G|}{|\stab_G(\omega)|} \\
        &= \vert Z(G) \vert + \sum_{\omega \notin Z(G)} \frac{|G|}{|\stab_G(\omega)|} \tag{$**$}
      \end{align*}
      On a maintenant deux cas :
      \begin{itemize}
        \item \uline{Il existe $\omega$ tel que $p^\alpha \mid |\stab_G(\omega)|$ :} Alors, comme $\stab_G(\omega)$ est un sous-groupe de $G$ d'ordre divisant strictement celui de $G$, on peut y appliquer l'hypothèse de récurrence pour obtenir un sous-groupe d'ordre $p^\alpha$. Ce sous-groupe est donc également un sous-groupe de $G$.
        \item \uline{Pour tout $\omega$, $p^\alpha \nmid |\stab_G(\omega)|$ :} Alors, en factorisant par $p$ dans les termes de la somme de $(**)$, on constate que $p \mid \frac{|G|}{|\stab_G(\omega)|}$ pour tout $\omega \notin Z(G)$. Comme $p \mid h$, toujours d'après $(**)$, on a
        \[ p \mid \vert Z(G) \vert \]
        $Z(G)$ étant commutatif, on peut appliquer le \cref{theoreme-de-sylow-1}. On obtient l'existence d'un sous-groupe $H$ de $Z(G)$ d'ordre $p$, qui est de plus distingué dans $G$ car inclus dans $Z(G)$. Alors,
        \[ \vert G/H \vert = \frac{\vert G \vert}{\vert H \vert} = np^{\alpha - 1} \]
        Il suffit maintenant d'appliquer l'hypothèse de récurrence à $G/H$, qui donne l'existence d'un sous-groupe $K$ de $G/H$ d'ordre $p^{\alpha - 1}$. On considère la surjection canonique
        \[ \pi_H : G \rightarrow G/H \]
        Alors, $\pi_H^{-1}(K) = \{ g \in G \mid gH \in K \}$ est un sous-groupe de $G$ d'ordre $\vert K \vert \times \vert H \vert = p^\alpha$ :
        \begin{center}
          \begin{tikzpicture}[scale=0.75]
            \draw[very thin, fill=teal!10, rotate=10] (-0.25,-1) ellipse (3.75 and 2.25);
            \draw (0.25,-2.5) circle (4.5);
            \begin{scope}[shift={(7,0)}]
              \draw[very thin, fill=teal!10, rotate=50] (0,0.4) ellipse (1.3 and 0.85);
              \foreach \a in {1,...,5} {
                \node at ({\a*72}:0.95) {\tiny $g_{\a}H$};
                \node[coordinate] (\a) at ($({\a*72}:0.95)+(0,0.2)$) {};
              }
              \draw(0,0) circle (1.5);
              \node at (0,0) {$K$};
            \end{scope}
            \foreach \center [count=\i] in {(-0.25,0),(2.25,-1),(-2,-1.75)} {
              \filldraw[teal!20] \center circle (1.1);
              \draw [very thin, ->] \center to [out=70,in=110] (\i);
            }
            \foreach \center [count=\i] in {(-0.25,0),(2.25,-1),(-2,-1.75),(2,-4.25),(-1,-5)} {
              \draw[dashed] \center circle (1.1);
              \begin{scope}[shift={(\center)}]
                \foreach \a [count=\j] in {1,...,5} {
                  \node at ({\a*72}:0.75) {\tiny $g_{\i}h_{\j}$};
                }
              \end{scope}
            }
            \node at (4.5,2) {$\pi_H$};
            \node at (0.5,-2.5) {$\pi_H^{-1}(K)$};
            \node[align=left] at (7.75,-6) {Partition de $G$ selon $\sim$ \\ où $g \sim h \iff g^{-1}h \in H$};
            \node at (9.5,-1) {$G/H$};
          \end{tikzpicture}
          % Dessin : https://agreg-maths.fr/uploads/versions/3229/Dev_Sylow_Reccurrence.pdf.
        \end{center}
        ce qu'on voulait.
      \end{itemize}
    \end{itemize}
  \end{proof}
  %</content>
\end{document}
