\documentclass[12pt, a4paper]{report}

% LuaLaTeX :

\RequirePackage{iftex}
\RequireLuaTeX

% Packages :

\usepackage[french]{babel}
%\usepackage[utf8]{inputenc}
%\usepackage[T1]{fontenc}
\usepackage[pdfencoding=auto, pdfauthor={Hugo Delaunay}, pdfsubject={Mathématiques}, pdfcreator={agreg.skyost.eu}]{hyperref}
\usepackage{amsmath}
\usepackage{amsthm}
%\usepackage{amssymb}
\renewcommand{\proofname}{Solution}
\usepackage{stmaryrd}
\usepackage{tikz}
\usepackage{tkz-euclide}
\usepackage{fontspec}
\defaultfontfeatures[Erewhon]{FontFace = {bx}{n}{Erewhon-Bold.otf}}
\usepackage{fourier-otf}
\usepackage[nobottomtitles*]{titlesec}
\usepackage{fancyhdr}
\usepackage{listings}
\usepackage{catchfilebetweentags}
\usepackage[french, capitalise, noabbrev]{cleveref}
\usepackage[fit, breakall]{truncate}
\usepackage[top=2.5cm, right=2cm, bottom=2.5cm, left=2cm]{geometry}
\usepackage{enumitem}
\usepackage{tablists} %Pour faire 1)  2) 3)
\usepackage{tocloft}
\usepackage{microtype}
%\usepackage{mdframed}
%\usepackage{thmtools}
\usepackage{xcolor}
\usepackage{tabularx}
\usepackage{xltabular}
\usepackage{aligned-overset}
\usepackage[subpreambles=true]{standalone}
\usepackage{environ}
\usepackage[normalem]{ulem}
\usepackage{multicol}
 \usepackage{variations}
\usepackage{array}% Pour faire des tableaux
\usepackage{etoolbox}
\usepackage{setspace}
\usepackage[bibstyle=reading, citestyle=draft]{biblatex}
\usepackage{xpatch}
\usepackage[many, breakable]{tcolorbox}
\usepackage[backgroundcolor=white, bordercolor=white, textsize=scriptsize]{todonotes}
\usepackage{luacode}
\usepackage{float}
\usepackage{needspace}


% Police :

\setmathfont{Erewhon Math}

% Tikz :

\usetikzlibrary{calc}
\usetikzlibrary{3d}

% Longueurs :

\setlength{\parindent}{0pt}
\setlength{\headheight}{15pt}
\setlength{\fboxsep}{0pt}
\titlespacing*{\chapter}{0pt}{-20pt}{10pt}
\setlength{\marginparwidth}{1.5cm}
\setstretch{1.1}

% Métadonnées :

\author{agreg.skyost.eu}
\date{\today}

% Titres :

\setcounter{secnumdepth}{3}

\renewcommand{\thechapter}{\Roman{chapter}}
\renewcommand{\thesubsection}{\Roman{subsection}}
\renewcommand{\thesubsubsection}{\arabic{subsubsection}}
\renewcommand{\theparagraph}{\alph{paragraph}}

\titleformat{\chapter}{\huge\bfseries}{\thechapter}{20pt}{\huge\bfseries}
\titleformat*{\section}{\LARGE\bfseries}
\titleformat{\subsection}{\Large\bfseries}{\thesubsection \, - \,}{0pt}{\Large\bfseries}
\titleformat{\subsubsection}{\large\bfseries}{\thesubsubsection. \,}{0pt}{\large\bfseries}
\titleformat{\paragraph}{\bfseries}{\theparagraph. \,}{0pt}{\bfseries}

\setcounter{secnumdepth}{4}

% Table des matières :

\renewcommand{\cftsecleader}{\cftdotfill{\cftdotsep}}
\addtolength{\cftsecnumwidth}{10pt}

% Redéfinition des commandes :

\renewcommand*\thesection{\arabic{section}}
\renewcommand{\ker}{\mathrm{Ker}}

% Nouvelles commandes :

\newcommand{\website}{http://sencoursdemaths.com}

\newcommand{\tr}[1]{\mathstrut ^t #1}
\newcommand{\im}{\mathrm{Im}}
\newcommand{\rang}{\operatorname{rang}}
\newcommand{\trace}{\operatorname{trace}}
\newcommand{\id}{\operatorname{id}}
\newcommand{\stab}{\operatorname{Stab}}
\newcommand{\paren}[1]{\left(#1\right)}
\newcommand{\accol}[1]{\left\{#1\right\}}
\newcommand{\croch}[1]{\left[ #1 \right]}
\newcommand{\Grdcroch}[1]{\Bigl[ #1 \Bigr]}
\newcommand{\grdcroch}[1]{\bigl[ #1 \bigr]}
\newcommand{\abs}[1]{\left\lvert #1 \right\rvert}
\newcommand{\limi}[3]{\displaystyle \lim_{#1\to #2}#3}
\newcommand{\pinf}{+\infty}
\newcommand{\minf}{-\infty}
%%%%%%%%%%%%%% ENSEMBLES %%%%%%%%%%%%%%%%%
\newcommand{\ensemblenombre}[1]{\mathbb{#1}}
\newcommand{\Nn}{\ensemblenombre{N}}
\newcommand{\Zz}{\ensemblenombre{Z}}
\newcommand{\Qq}{\ensemblenombre{Q}}
\newcommand{\Qqp}{\Qq^+}
\newcommand{\Rr}{\ensemblenombre{R}}
\newcommand{\Cc}{\ensemblenombre{C}}
\newcommand{\Nne}{\Nn^*}
\newcommand{\Zze}{\Zz^*}
\newcommand{\Zzn}{\Zz^-}
\newcommand{\Qqe}{\Qq^*}
\newcommand{\Rre}{\Rr^*}
\newcommand{\Rrp}{\Rr_+}
\newcommand{\Rrm}{\Rr_-}
\newcommand{\Rrep}{\Rr_+^*}
\newcommand{\Rrem}{\Rr_-^*}
\newcommand{\Cce}{\Cc^*}
%%%%%%%%%%%%%%  INTERVALLES %%%%%%%%%%%%%%%%%
\newcommand{\intff}[2]{\left[#1\;,\; #2\right]  }
\newcommand{\intof}[2]{\left]#1 \;, \;#2\right]  }
\newcommand{\intfo}[2]{\left[#1 \;,\; #2\right[  }
\newcommand{\intoo}[2]{\left]#1 \;,\; #2\right[  }



\providecommand{\newpar}{\\[\medskipamount]}

\newcommand{\annexessection}{%
  \newpage%
  \subsection*{Annexes}%
}

\providecommand{\lesson}[3]{%
  \title{#3}%
  \hypersetup{pdftitle={#2 : #3}}%
  \setcounter{section}{\numexpr #2 - 1}%
  \section{#3}%
  \fancyhead[R]{\truncate{0.73\textwidth}{#2 : #3}}%
}

\providecommand{\development}[3]{%
  \title{#3}%
  \hypersetup{pdftitle={#3}}%
  \section*{#3}%
  \fancyhead[R]{\truncate{0.73\textwidth}{#3}}%
}

\providecommand{\sheet}[3]{\development{#1}{#2}{#3}}

\providecommand{\ranking}[1]{%
  \title{Terminale #1}%
  \hypersetup{pdftitle={Terminale #1}}%
  \section*{Terminale #1}%
  \fancyhead[R]{\truncate{0.73\textwidth}{Terminale #1}}%
}

\providecommand{\summary}[1]{%
  \textit{#1}%
  \par%
  \medskip%
}

\tikzset{notestyleraw/.append style={inner sep=0pt, rounded corners=0pt, align=center}}

%\newcommand{\booklink}[1]{\website/bibliographie\##1}
\newcounter{reference}
\newcommand{\previousreference}{}
\providecommand{\reference}[2][]{%
  \needspace{20pt}%
  \notblank{#1}{
    \needspace{20pt}%
    \renewcommand{\previousreference}{#1}%
    \stepcounter{reference}%
    \label{reference-\previousreference-\thereference}%
  }{}%
  \todo[noline]{%
    \protect\vspace{20pt}%
    \protect\par%
    \protect\notblank{#1}{\cite{[\previousreference]}\\}{}%
    \protect\hyperref[reference-\previousreference-\thereference]{p. #2}%
  }%
}

\definecolor{devcolor}{HTML}{00695c}
\providecommand{\dev}[1]{%
  \reversemarginpar%
  \todo[noline]{
    \protect\vspace{20pt}%
    \protect\par%
    \bfseries\color{devcolor}\href{\website/developpements/#1}{[DEV]}
  }%
  \normalmarginpar%
}

% En-têtes :

\pagestyle{fancy}
\fancyhead[L]{\truncate{0.23\textwidth}{\thepage}}
\fancyfoot[C]{\scriptsize \href{\website}{\texttt{http://sencoursdemaths.com}}}

% Couleurs :

\definecolor{property}{HTML}{ffeb3b}
\definecolor{proposition}{HTML}{ffc107}
\definecolor{lemma}{HTML}{ff9800}
\definecolor{theorem}{HTML}{f44336}
\definecolor{corollary}{HTML}{e91e63}
\definecolor{definition}{HTML}{673ab7}
\definecolor{notation}{HTML}{9c27b0}
\definecolor{example}{HTML}{00bcd4}
\definecolor{cexample}{HTML}{795548}
\definecolor{application}{HTML}{009688}
\definecolor{remark}{HTML}{3f51b5}
\definecolor{algorithm}{HTML}{607d8b}
\definecolor{proof}{HTML}{e1f5fe}
\definecolor{exercice}{HTML}{e1f5fe}

% Théorèmes :

\theoremstyle{definition}
\newtheorem{theorem}{Théorème}

\newtheorem{property}[theorem]{Propriété}
\newtheorem{proposition}[theorem]{Proposition}
\newtheorem{lemma}[theorem]{Activité d'introduction}
\newtheorem{corollary}[theorem]{Conséquence}

\newtheorem{definition}[theorem]{Définition}
\newtheorem{notation}[theorem]{Notation}

\newtheorem{example}[theorem]{Exemple}
\newtheorem{cexample}[theorem]{Contre-exemple}
\newtheorem{application}[theorem]{Application}

\newtheorem{algorithm}[theorem]{Algorithme}
\newtheorem{exercice}[theorem]{Exercice}

\theoremstyle{remark}
\newtheorem{remark}[theorem]{Remarque}




\counterwithin*{theorem}{section}

\newcommand{\applystyletotheorem}[1]{
  \tcolorboxenvironment{#1}{
    enhanced,
    breakable,
    colback=#1!8!white,
    %right=0pt,
    %top=8pt,
    %bottom=8pt,
    boxrule=0pt,
    frame hidden,
    sharp corners,
    enhanced,borderline west={4pt}{0pt}{#1},
    %interior hidden,
    sharp corners,
    after=\par,
  }
}

\applystyletotheorem{property}
\applystyletotheorem{proposition}
\applystyletotheorem{lemma}
\applystyletotheorem{theorem}
\applystyletotheorem{corollary}
\applystyletotheorem{definition}
\applystyletotheorem{notation}
\applystyletotheorem{example}
\applystyletotheorem{cexample}
\applystyletotheorem{application}
\applystyletotheorem{remark}
%\applystyletotheorem{proof}
\applystyletotheorem{algorithm}
\applystyletotheorem{exercice}

% Environnements :

\NewEnviron{whitetabularx}[1]{%
  \renewcommand{\arraystretch}{2.5}
  \colorbox{white}{%
    \begin{tabularx}{\textwidth}{#1}%
      \BODY%
    \end{tabularx}%
  }%
}

% Maths :

\DeclareFontEncoding{FMS}{}{}
\DeclareFontSubstitution{FMS}{futm}{m}{n}
\DeclareFontEncoding{FMX}{}{}
\DeclareFontSubstitution{FMX}{futm}{m}{n}
\DeclareSymbolFont{fouriersymbols}{FMS}{futm}{m}{n}
\DeclareSymbolFont{fourierlargesymbols}{FMX}{futm}{m}{n}
\DeclareMathDelimiter{\VERT}{\mathord}{fouriersymbols}{152}{fourierlargesymbols}{147}

% Code :

\definecolor{greencode}{rgb}{0,0.6,0}
\definecolor{graycode}{rgb}{0.5,0.5,0.5}
\definecolor{mauvecode}{rgb}{0.58,0,0.82}
\definecolor{bluecode}{HTML}{1976d2}
\lstset{
  basicstyle=\footnotesize\ttfamily,
  breakatwhitespace=false,
  breaklines=true,
  %captionpos=b,
  commentstyle=\color{greencode},
  deletekeywords={...},
  escapeinside={\%*}{*)},
  extendedchars=true,
  frame=none,
  keepspaces=true,
  keywordstyle=\color{bluecode},
  language=Python,
  otherkeywords={*,...},
  numbers=left,
  numbersep=5pt,
  numberstyle=\tiny\color{graycode},
  rulecolor=\color{black},
  showspaces=false,
  showstringspaces=false,
  showtabs=false,
  stepnumber=2,
  stringstyle=\color{mauvecode},
  tabsize=2,
  %texcl=true,
  xleftmargin=10pt,
  %title=\lstname
}

\newcommand{\codedirectory}{}
\newcommand{\inputalgorithm}[1]{%
  \begin{algorithm}%
    \strut%
    \lstinputlisting{\codedirectory#1}%
  \end{algorithm}%
}





\begin{document}
  %<*content>
  \development{analysis}{connexite-des-valeurs-d-adherence-d-une-suite-dans-un-compact}{Connexité des valeurs d'adhérence d'une suite dans un compact}

  \summary{On montre que l'ensemble des valeurs d'adhérence d'une suite d'un espace métrique compact est connexe en raisonnant par l'absurde, puis on utilise ce résultat pour démontrer le lemme des grenouilles.}

  \reference[I-P]{116}

  Soit $(E, d)$ un espace métrique.

  \begin{theorem}
    \label{connexite-des-valeurs-d-adherence-d-une-suite-dans-un-compact-1}
    On suppose $E$ compact. Soit $(u_n)$ une suite de $E$ telle que $d(u_n,u_{n-1}) \longrightarrow 0$. Alors l'ensemble $\Gamma$ des valeurs d'adhérence de $(u_n)$ est connexe.
  \end{theorem}

  \begin{proof}
    Pour tout $p \in \mathbb{N}$, on note $A_p = \{u_n \mid n\geq p\}$. On a $\Gamma = \bigcap_{p \in \mathbb{N}} \overline{A_p}$. $\Gamma$ est fermé (en tant qu'intersection de fermés) dans $E$ qui est compact, donc $\Gamma$ est compact.
    Supposons que $\Gamma$ soit non connexe ; on peut alors écrire $\Gamma = A \, \sqcup \, B$, où $A$ et $B$ sont deux fermés disjoints de $\Gamma$. Comme $\Gamma$ est compact, $A$ et $B$ le sont aussi. Notons $\alpha = d(A, B) > 0$ (car $A \, \cap \, B = \emptyset$). Posons :
    \[
    A'= \left \{ x \in E \mid d(x, A) < \frac{\alpha}{3} \right \} \text{ et } B'= \left \{ x \in E \mid d(x, B) < \frac{\alpha}{3} \right \}
    \]
    $A'$ et $B'$ sont ouverts (en tant qu'images réciproques d'ouverts par des application continues), donc $K = E \setminus (A'\cup B')$ est fermé dans $E$, donc compact.
    \newpar
    Montrons que $(u_n)$ admet une valeur d'adhérence dans $K$, ce qui serait absurde car $\Gamma \, \cap \, K = \emptyset$. Comme $\lim_{n \rightarrow +\infty} d(u_n, u_{n-1})=0$,
    \[ \exists N_0 \in \mathbb{N} \text{ tel que } \forall n \geq N_0, \, d(u_n, u_{n-1}) < \frac{\alpha}{3} \tag{$*$} \]
    Soit $N \geq N_0$.
    \begin{itemize}
      \item Soit $x_0 \in A$. Comme $x_0$ est valeur d'adhérence de $(u_n)$, $\exists n_1 > N$ tel que $d(x_0, u_{n_1}) < \frac{\alpha}{3}$. Donc $u_{n_1} \in A'$.
      \item Soit $y_0 \in B$. De même, $\exists n_2 > n_1$ tel que $d(y_0, u_{n_2}) < \frac{\alpha}{3}$. Donc $u_{n_2} \in B'$.
    \end{itemize}
    Soit maintenant $n_0$ le premier entier supérieur à $n_1$ tel que $u_{n_0} \notin A'$ (un tel entier existe car $u_{n_2} \notin A'$). On a alors $u_{n_0 - 1} \in A'$.
    \begin{center}
      \begin{tikzpicture}[scale=1, pics/a/.style n args={3}{code={\draw [thick, fill=#1, fill opacity=0.3, scale=#2, shift={#3}]  plot[smooth, tension=.7] coordinates {(-3,2.5) (-3.5,2.2) (-4.2,1) (-2,0.8) (-1.2,2) (-3,2.5)};}}, pics/b/.style n args={3}{code={\draw [thick, fill=#1, fill opacity=0.3, scale=#2, shift={#3}]  plot[smooth, tension=.7] coordinates {(-1.95,-0.35) (-2.5,-0.7) (-3.4,-1.9) (-1,-2) (-0.1,-0.05) (-1.95,-0.35)};}}]
        \draw [thick, fill=black, fill opacity=0.05]  plot[smooth, tension=.7] coordinates {(-4,2.5) (-3,3) (-2,2.8) (-0.8,2.5) (-0.5,1.5) (0.5,0) (0,-2) (-1.5,-2.5) (-4,-2) (-3.5,-0.5) (-5,1) (-4,2.5)};
        \pic {a={black!20!green!60!white}{1}{(0,0)}};
        \pic {a={black!20!green}{0.75}{(-0.9,0.5)}};
        \pic {b={black!20!green!60!white}{1}{(0,0)}};
        \pic {b={black!20!green}{0.75}{(-0.6,-0.45)}};
        \draw(-3,1.3) node {$\bullet$} node[left]{\tiny $x_0$};
        \draw(-3.1,0.82) node {$\bullet$} node[left]{\tiny $u_{n_1}$};
        \draw(-1.2,-0.7) node {$\bullet$} node[right]{\tiny $y_0$};
        \draw(-1.3,-0.3) node {$\bullet$} node[left]{\tiny $u_{n_2}$};
        \draw(-2.5,0.75) node {$\bullet$} node[right]{\tiny $u_{n_0 - 1}$};
        \draw(-2.3,0.4) node {$\bullet$} node[right]{\tiny $u_{n_0}$};
        \node at (-2.5,1.5) {\small $A$};
        \node at (-1.35,1.85) {\small $A'$};
        \node at (-1.6,-1.2) {\small $B$};
        \node at (-1.6,-2.07) {\small $B'$};
        \node at (-0.7,0.7) {$K$};
        \node at (-2.5,3.3) {$E$};
      \end{tikzpicture}
    \end{center}
    D'après $(*)$, en appliquant l'inégalité triangulaire,
    \begin{align*}
      d(u_{n_0}, B) & \geq d(u_{n_0 - 1}, B) - d(u_{n_0 - 1}, u_{n_0})           \\
      & \geq d(A, B) - d(u_{n_0 - 1}, A) - d(u_{n_0 - 1}, u_{n_0}) \\
      & > \frac{\alpha}{3}
    \end{align*}
    ce qui prouve que $u_{n_0} \notin B'$. Comme $u_{n_0} \notin A'$, on a $u_{n_0} \in K$. On vient de montrer que,
    \[ \forall N \geq N_0, \, \exists n_0 \geq N \text{ tel que } u_{n_0} \in K \]
    On peut créer comme cela une sous-suite de $(u_n)$ dans $K$. Or $K$ est compact, donc $(u_n)$ admet une valeur d'adhérence dans $K$.
  \end{proof}

  \begin{application}[Lemme de la grenouille]
    Soient $f : [0, 1] \rightarrow [0, 1]$ continue et $(x_n)$ une suite de $[0, 1]$ telle que
    \[ \begin{cases} x_0 \in [0, 1] \\ x_{n+1} = f(x_n) \end{cases} \]
    Alors $(x_n)$ converge si et seulement si $\lim_{n \rightarrow +\infty } x_{n+1} - x_n = 0$.
  \end{application}

  \begin{proof}
    Le sens direct est évident. Montrons la réciproque. On suppose donc que $\lim_{n \rightarrow +\infty } x_{n+1} - x_n = 0$ et on note $\Gamma$ l'ensemble des valeurs d'adhérence de $(x_n)$. $\Gamma$ est non vide (car $(x_n)$ est bornée, donc admet une valeur d'adhérence par le théorème de Bolzano-Weierstrass) et est un connexe de $\mathbb{R}$ (par le \cref{connexite-des-valeurs-d-adherence-d-une-suite-dans-un-compact-1}), donc $\Gamma$ est un intervalle non vide.
    \newpar
    Soit $a \in \Gamma$. Il existe $\varphi : \mathbb{N} \rightarrow \mathbb{N}$ strictement croissante (on dit que $\varphi$ est une extractrice) telle que $x_{\varphi(n)} \longrightarrow a$. Mais alors,
    \[ x_{\varphi(n) + 1} - x_{\varphi(n)} = f(x_{\varphi(n)}) - x_{\varphi(n)} \longrightarrow f(a) - a \]
    et par hypothèse, le membre de gauche converge vers $0$. Donc $f(a) - a = 0$ ie. $a$ est un point fixe de $f$.
    \newpar
    Supposons par l'absurde que $(x_n)$ diverge. Alors $\Gamma$ n'est pas un singleton, donc est un intervalle d'intérieur non vide : on peut trouver $c \in \Gamma$ et $h > 0$ tel que $[c - h, c + h] \subseteq \Gamma$.
    \newpar
    Or, $c \in \Gamma$, donc
    \[ \exists N \geq 0 \text{ tel que } |x_N - c| \leq \frac{h}{2} \implies x_N \in \Gamma \]
    et en particulier, $x_N$ est un point fixe de $f$. Ainsi, $x_{n+1} = f(x_n) = x_n$ pour tout $n \geq N$ : absurde.
  \end{proof}
  %</content>
\end{document}
