\documentclass[12pt, a4paper]{report}

% LuaLaTeX :

\RequirePackage{iftex}
\RequireLuaTeX

% Packages :

\usepackage[french]{babel}
%\usepackage[utf8]{inputenc}
%\usepackage[T1]{fontenc}
\usepackage[pdfencoding=auto, pdfauthor={Hugo Delaunay}, pdfsubject={Mathématiques}, pdfcreator={agreg.skyost.eu}]{hyperref}
\usepackage{amsmath}
\usepackage{amsthm}
%\usepackage{amssymb}
\renewcommand{\proofname}{Solution}
\usepackage{stmaryrd}
\usepackage{tikz}
\usepackage{tkz-euclide}
\usepackage{fontspec}
\defaultfontfeatures[Erewhon]{FontFace = {bx}{n}{Erewhon-Bold.otf}}
\usepackage{fourier-otf}
\usepackage[nobottomtitles*]{titlesec}
\usepackage{fancyhdr}
\usepackage{listings}
\usepackage{catchfilebetweentags}
\usepackage[french, capitalise, noabbrev]{cleveref}
\usepackage[fit, breakall]{truncate}
\usepackage[top=2.5cm, right=2cm, bottom=2.5cm, left=2cm]{geometry}
\usepackage{enumitem}
\usepackage{tablists} %Pour faire 1)  2) 3)
\usepackage{tocloft}
\usepackage{microtype}
%\usepackage{mdframed}
%\usepackage{thmtools}
\usepackage{xcolor}
\usepackage{tabularx}
\usepackage{xltabular}
\usepackage{aligned-overset}
\usepackage[subpreambles=true]{standalone}
\usepackage{environ}
\usepackage[normalem]{ulem}
\usepackage{multicol}
 \usepackage{variations}
\usepackage{array}% Pour faire des tableaux
\usepackage{etoolbox}
\usepackage{setspace}
\usepackage[bibstyle=reading, citestyle=draft]{biblatex}
\usepackage{xpatch}
\usepackage[many, breakable]{tcolorbox}
\usepackage[backgroundcolor=white, bordercolor=white, textsize=scriptsize]{todonotes}
\usepackage{luacode}
\usepackage{float}
\usepackage{needspace}


% Police :

\setmathfont{Erewhon Math}

% Tikz :

\usetikzlibrary{calc}
\usetikzlibrary{3d}

% Longueurs :

\setlength{\parindent}{0pt}
\setlength{\headheight}{15pt}
\setlength{\fboxsep}{0pt}
\titlespacing*{\chapter}{0pt}{-20pt}{10pt}
\setlength{\marginparwidth}{1.5cm}
\setstretch{1.1}

% Métadonnées :

\author{agreg.skyost.eu}
\date{\today}

% Titres :

\setcounter{secnumdepth}{3}

\renewcommand{\thechapter}{\Roman{chapter}}
\renewcommand{\thesubsection}{\Roman{subsection}}
\renewcommand{\thesubsubsection}{\arabic{subsubsection}}
\renewcommand{\theparagraph}{\alph{paragraph}}

\titleformat{\chapter}{\huge\bfseries}{\thechapter}{20pt}{\huge\bfseries}
\titleformat*{\section}{\LARGE\bfseries}
\titleformat{\subsection}{\Large\bfseries}{\thesubsection \, - \,}{0pt}{\Large\bfseries}
\titleformat{\subsubsection}{\large\bfseries}{\thesubsubsection. \,}{0pt}{\large\bfseries}
\titleformat{\paragraph}{\bfseries}{\theparagraph. \,}{0pt}{\bfseries}

\setcounter{secnumdepth}{4}

% Table des matières :

\renewcommand{\cftsecleader}{\cftdotfill{\cftdotsep}}
\addtolength{\cftsecnumwidth}{10pt}

% Redéfinition des commandes :

\renewcommand*\thesection{\arabic{section}}
\renewcommand{\ker}{\mathrm{Ker}}

% Nouvelles commandes :

\newcommand{\website}{http://sencoursdemaths.com}

\newcommand{\tr}[1]{\mathstrut ^t #1}
\newcommand{\im}{\mathrm{Im}}
\newcommand{\rang}{\operatorname{rang}}
\newcommand{\trace}{\operatorname{trace}}
\newcommand{\id}{\operatorname{id}}
\newcommand{\stab}{\operatorname{Stab}}
\newcommand{\paren}[1]{\left(#1\right)}
\newcommand{\accol}[1]{\left\{#1\right\}}
\newcommand{\croch}[1]{\left[ #1 \right]}
\newcommand{\Grdcroch}[1]{\Bigl[ #1 \Bigr]}
\newcommand{\grdcroch}[1]{\bigl[ #1 \bigr]}
\newcommand{\abs}[1]{\left\lvert #1 \right\rvert}
\newcommand{\limi}[3]{\displaystyle \lim_{#1\to #2}#3}
\newcommand{\pinf}{+\infty}
\newcommand{\minf}{-\infty}
%%%%%%%%%%%%%% ENSEMBLES %%%%%%%%%%%%%%%%%
\newcommand{\ensemblenombre}[1]{\mathbb{#1}}
\newcommand{\Nn}{\ensemblenombre{N}}
\newcommand{\Zz}{\ensemblenombre{Z}}
\newcommand{\Qq}{\ensemblenombre{Q}}
\newcommand{\Qqp}{\Qq^+}
\newcommand{\Rr}{\ensemblenombre{R}}
\newcommand{\Cc}{\ensemblenombre{C}}
\newcommand{\Nne}{\Nn^*}
\newcommand{\Zze}{\Zz^*}
\newcommand{\Zzn}{\Zz^-}
\newcommand{\Qqe}{\Qq^*}
\newcommand{\Rre}{\Rr^*}
\newcommand{\Rrp}{\Rr_+}
\newcommand{\Rrm}{\Rr_-}
\newcommand{\Rrep}{\Rr_+^*}
\newcommand{\Rrem}{\Rr_-^*}
\newcommand{\Cce}{\Cc^*}
%%%%%%%%%%%%%%  INTERVALLES %%%%%%%%%%%%%%%%%
\newcommand{\intff}[2]{\left[#1\;,\; #2\right]  }
\newcommand{\intof}[2]{\left]#1 \;, \;#2\right]  }
\newcommand{\intfo}[2]{\left[#1 \;,\; #2\right[  }
\newcommand{\intoo}[2]{\left]#1 \;,\; #2\right[  }



\providecommand{\newpar}{\\[\medskipamount]}

\newcommand{\annexessection}{%
  \newpage%
  \subsection*{Annexes}%
}

\providecommand{\lesson}[3]{%
  \title{#3}%
  \hypersetup{pdftitle={#2 : #3}}%
  \setcounter{section}{\numexpr #2 - 1}%
  \section{#3}%
  \fancyhead[R]{\truncate{0.73\textwidth}{#2 : #3}}%
}

\providecommand{\development}[3]{%
  \title{#3}%
  \hypersetup{pdftitle={#3}}%
  \section*{#3}%
  \fancyhead[R]{\truncate{0.73\textwidth}{#3}}%
}

\providecommand{\sheet}[3]{\development{#1}{#2}{#3}}

\providecommand{\ranking}[1]{%
  \title{Terminale #1}%
  \hypersetup{pdftitle={Terminale #1}}%
  \section*{Terminale #1}%
  \fancyhead[R]{\truncate{0.73\textwidth}{Terminale #1}}%
}

\providecommand{\summary}[1]{%
  \textit{#1}%
  \par%
  \medskip%
}

\tikzset{notestyleraw/.append style={inner sep=0pt, rounded corners=0pt, align=center}}

%\newcommand{\booklink}[1]{\website/bibliographie\##1}
\newcounter{reference}
\newcommand{\previousreference}{}
\providecommand{\reference}[2][]{%
  \needspace{20pt}%
  \notblank{#1}{
    \needspace{20pt}%
    \renewcommand{\previousreference}{#1}%
    \stepcounter{reference}%
    \label{reference-\previousreference-\thereference}%
  }{}%
  \todo[noline]{%
    \protect\vspace{20pt}%
    \protect\par%
    \protect\notblank{#1}{\cite{[\previousreference]}\\}{}%
    \protect\hyperref[reference-\previousreference-\thereference]{p. #2}%
  }%
}

\definecolor{devcolor}{HTML}{00695c}
\providecommand{\dev}[1]{%
  \reversemarginpar%
  \todo[noline]{
    \protect\vspace{20pt}%
    \protect\par%
    \bfseries\color{devcolor}\href{\website/developpements/#1}{[DEV]}
  }%
  \normalmarginpar%
}

% En-têtes :

\pagestyle{fancy}
\fancyhead[L]{\truncate{0.23\textwidth}{\thepage}}
\fancyfoot[C]{\scriptsize \href{\website}{\texttt{http://sencoursdemaths.com}}}

% Couleurs :

\definecolor{property}{HTML}{ffeb3b}
\definecolor{proposition}{HTML}{ffc107}
\definecolor{lemma}{HTML}{ff9800}
\definecolor{theorem}{HTML}{f44336}
\definecolor{corollary}{HTML}{e91e63}
\definecolor{definition}{HTML}{673ab7}
\definecolor{notation}{HTML}{9c27b0}
\definecolor{example}{HTML}{00bcd4}
\definecolor{cexample}{HTML}{795548}
\definecolor{application}{HTML}{009688}
\definecolor{remark}{HTML}{3f51b5}
\definecolor{algorithm}{HTML}{607d8b}
\definecolor{proof}{HTML}{e1f5fe}
\definecolor{exercice}{HTML}{e1f5fe}

% Théorèmes :

\theoremstyle{definition}
\newtheorem{theorem}{Théorème}

\newtheorem{property}[theorem]{Propriété}
\newtheorem{proposition}[theorem]{Proposition}
\newtheorem{lemma}[theorem]{Activité d'introduction}
\newtheorem{corollary}[theorem]{Conséquence}

\newtheorem{definition}[theorem]{Définition}
\newtheorem{notation}[theorem]{Notation}

\newtheorem{example}[theorem]{Exemple}
\newtheorem{cexample}[theorem]{Contre-exemple}
\newtheorem{application}[theorem]{Application}

\newtheorem{algorithm}[theorem]{Algorithme}
\newtheorem{exercice}[theorem]{Exercice}

\theoremstyle{remark}
\newtheorem{remark}[theorem]{Remarque}




\counterwithin*{theorem}{section}

\newcommand{\applystyletotheorem}[1]{
  \tcolorboxenvironment{#1}{
    enhanced,
    breakable,
    colback=#1!8!white,
    %right=0pt,
    %top=8pt,
    %bottom=8pt,
    boxrule=0pt,
    frame hidden,
    sharp corners,
    enhanced,borderline west={4pt}{0pt}{#1},
    %interior hidden,
    sharp corners,
    after=\par,
  }
}

\applystyletotheorem{property}
\applystyletotheorem{proposition}
\applystyletotheorem{lemma}
\applystyletotheorem{theorem}
\applystyletotheorem{corollary}
\applystyletotheorem{definition}
\applystyletotheorem{notation}
\applystyletotheorem{example}
\applystyletotheorem{cexample}
\applystyletotheorem{application}
\applystyletotheorem{remark}
%\applystyletotheorem{proof}
\applystyletotheorem{algorithm}
\applystyletotheorem{exercice}

% Environnements :

\NewEnviron{whitetabularx}[1]{%
  \renewcommand{\arraystretch}{2.5}
  \colorbox{white}{%
    \begin{tabularx}{\textwidth}{#1}%
      \BODY%
    \end{tabularx}%
  }%
}

% Maths :

\DeclareFontEncoding{FMS}{}{}
\DeclareFontSubstitution{FMS}{futm}{m}{n}
\DeclareFontEncoding{FMX}{}{}
\DeclareFontSubstitution{FMX}{futm}{m}{n}
\DeclareSymbolFont{fouriersymbols}{FMS}{futm}{m}{n}
\DeclareSymbolFont{fourierlargesymbols}{FMX}{futm}{m}{n}
\DeclareMathDelimiter{\VERT}{\mathord}{fouriersymbols}{152}{fourierlargesymbols}{147}

% Code :

\definecolor{greencode}{rgb}{0,0.6,0}
\definecolor{graycode}{rgb}{0.5,0.5,0.5}
\definecolor{mauvecode}{rgb}{0.58,0,0.82}
\definecolor{bluecode}{HTML}{1976d2}
\lstset{
  basicstyle=\footnotesize\ttfamily,
  breakatwhitespace=false,
  breaklines=true,
  %captionpos=b,
  commentstyle=\color{greencode},
  deletekeywords={...},
  escapeinside={\%*}{*)},
  extendedchars=true,
  frame=none,
  keepspaces=true,
  keywordstyle=\color{bluecode},
  language=Python,
  otherkeywords={*,...},
  numbers=left,
  numbersep=5pt,
  numberstyle=\tiny\color{graycode},
  rulecolor=\color{black},
  showspaces=false,
  showstringspaces=false,
  showtabs=false,
  stepnumber=2,
  stringstyle=\color{mauvecode},
  tabsize=2,
  %texcl=true,
  xleftmargin=10pt,
  %title=\lstname
}

\newcommand{\codedirectory}{}
\newcommand{\inputalgorithm}[1]{%
  \begin{algorithm}%
    \strut%
    \lstinputlisting{\codedirectory#1}%
  \end{algorithm}%
}




% Bibliographie :

%\addbibresource{\bibliographypath}%
\defbibheading{bibliography}[\bibname]{\section*{#1}}
\renewbibmacro*{entryhead:full}{\printfield{labeltitle}}%
\DeclareFieldFormat{url}{\newline\footnotesize\url{#1}}%

\AtEndDocument{%
  \newpage%
  \pagestyle{empty}%
  \printbibliography%
}


\begin{document}
  %<*content>
  \development{algebra}{theoreme-chinois}{Théorème chinois}

  \summary{On montre le théorème chinois et on propose une application à la résolution d'un système de congruences.}

  \reference[ROM21]{250}

  Soit $A$ un anneau principal. Soient $r \geq 2$ un entier et $a_1, \dots, a_r \in A$ des éléments premiers entre eux deux à deux.

  \begin{notation}
    Pour tout $i \in \llbracket 1, r \rrbracket$, on note
    \[ \pi_i = \pi_{(a_i)} : A \rightarrow A/(a_i) \]
    la surjection canonique de $A$ sur $A/(a_i)$. On note également $\pi = \pi_{(a_1 \dots a_r)} : A \rightarrow A/(a_1 \dots A_r)$.
  \end{notation}

  \begin{theorem}[Chinois]
    \label{theoreme-chinois-1}
    Alors :
    \begin{enumerate}[label=(\roman*)]
      \item L'application :
      \[
        \varphi :
        \begin{array}{ccc}
          A &\rightarrow& A/(a_1) \times \dots \times A/(a_r) \\
          x &\mapsto& (\pi_1(x), \dots, \pi_r(x))
        \end{array}
      \]
      est un morphisme d'anneaux de noyau $\ker(\varphi) = (a_1 \dots a_r)$.
      \item Il existe $u_1, \dots, u_r \in A$ tels que
      \[ \sum_{i=1}^{r} u_i b_i = 1 \]
      où $\forall i \in \llbracket 1, r \rrbracket$, $b_i = \frac{a}{a_i}$ et $a = a_1 \dots a_r$.
      \item $\varphi$ est surjectif et induit un isomorphisme $\overline{\varphi} : A/(a_1 \dots a_r) \rightarrow A/(a_1) \times \dots \times A/(a_r)$. On a,
      \[
        \overline{\varphi}^{-1} :
        \begin{array}{ccc}
          A/(a_1) \times \dots \times A/(a_r) &\rightarrow& A/(a_1 \dots a_r) \\
          (\pi_1(x_1), \dots, \pi_r(x_r)) &\mapsto& \pi \left( \sum_{i=1}^{r} x_i u_i b_i \right)
        \end{array}
      \]
      où $\pi$ est la surjection canonique de $A$ sur le quotient $A/(a_1 \dots a_r)$.
    \end{enumerate}
  \end{theorem}

  \begin{proof}
    \begin{enumerate}[label=(\roman*)]
      \item On vérifie sans difficulté que $\varphi$ est un morphisme d'anneaux (du fait que les projections canoniques sur les quotients en sont). De là,
      \begin{align*}
        \ker(\varphi) &= \{ x \in A \mid \forall i \in \llbracket 1, r \rrbracket, \, \pi_i(x) = 0 \} \\
        &= \{ x \in A \mid \forall i \in \llbracket 1, r \rrbracket, \, a_i \mid x \} \\
        &= \{ x \in A \mid \operatorname{ppcm}(a_1, \dots, a_r) \mid x \}
      \end{align*}
      Mais, $a_1, \dots, a_r$ sont premiers entre eux deux à deux. Donc,
      \[ \operatorname{ppcm}(a_1, \dots, a_r) = a_1 \dots a_r \]
      et on conclut que $\ker(\varphi) = (a_1 \dots a_r)$.
      \item Supposons par l'absurde que $b_1, \dots, b_r$ ne sont pas premiers entre eux dans leur ensemble. Comme $A$ est principal, donc factoriel, il existe un premier $p \in A$ tel que
      \[ \forall i \in \llbracket 1, r \rrbracket, \, p \mid b_i \]
      Comme $p$ divise $b_1 = a_2 \dots a_r$, il existe $i \in \llbracket 2, r \rrbracket$ tel que $p \mid a_i$. Mais, divisant $b_i$, il divise $a_j$ où $j \in \llbracket 1, r \rrbracket \setminus \{ i \}$. Contradiction car $a_1$ et $a_j$ sont premiers entre eux. La fin du raisonnement est une conséquence directe du théorème de Bézout valable dans les anneaux principaux.
      \item Pour $i, j \in \llbracket 1, r \rrbracket$ tels que $i \neq j$, on a
      \[ \pi_j(b_i) = \pi_j(0) \]
      puisque $b_i$ est multiple de $a_j$. Ceci permet d'écrire
      \[ \pi_j(1) = \pi_j\left( \sum_{i=1}^{r} u_i b_i \right) = \pi_j(u_j) \pi_j(b_j) \]
      Donc, $\pi_j(b_j)$ est inversible dans $A/(a_j)$, d'inverse $\pi_j(u_j)$. Ainsi, soient $\pi_1(x_1), \dots, \pi_r(x_r) \in A/(a_1) \times \dots \times A/(a_r)$. En posant
      \[ x = \sum_{i=1}^r x_i u_i b_i \]
      on a
      \[ \pi_j(x) = \pi_j(x_j) \pi_j(u_j) \pi_j(b_j) = \pi_j(x_j) \]
      donc $\varphi(x) = (\pi_1(x_1), \dots, \pi_r(x_r))$. Le morphisme $\varphi$ est surjectif. Par le théorème de factorisation des morphismes, il induit un isomorphisme
      \[
        \overline{\varphi} :
        \begin{array}{ccc}
          A/(a_1 \dots a_r) &\rightarrow& A/(a_1) \times \dots \times A/(a_r) \\
          \pi(x) &\mapsto& (\pi_1(x), \dots, \pi_r(x))
        \end{array}
      \]
      et on a même prouvé que l'inverse $\overline{\varphi}^{-1}$ est défini par
      \[
      \overline{\varphi}^{-1} :
      \begin{array}{ccc}
        A/(a_1) \times \dots \times A/(a_r) &\rightarrow& A/(a_1 \dots a_r) \\
        (\pi_1(x_1), \dots, \pi_r(x_r)) &\mapsto& \pi \left( \sum_{i=1}^{r} x_i u_i b_i \right)
      \end{array}
      \]
    \end{enumerate}
  \end{proof}

  \reference[ULM18]{58}

  \begin{example}
    Le système
    \[
    \begin{cases}
      u \equiv 1 \mod 3 \\
      u \equiv 3 \mod 5 \\
      u \equiv 0 \mod 7
    \end{cases}
    \]
    admet une unique solution dans $\mathbb{Z}/105\mathbb{Z}$ : $\overline{28}$. Les solutions dans $\mathbb{Z}$ sont donc de la forme $28 + 105k$ avec $k \in \mathbb{Z}$.
  \end{example}

  \begin{proof}
    On se place dans l'anneau principal $A = \mathbb{Z}$. Les entiers $3$, $5$ et $7$ sont premiers entre eux : le triplet $(1 + (3), 3 + (5), 0 + (7)) = (x_1 + (3), x_2 + (5), x_3 + (3))$ admet un unique antécédent par $\overline{\varphi}^{-1}$ du \cref{theoreme-chinois-1}. On a ainsi existence et unicité d'une solution modulo $3 \times 5 \times 7 = 105$. On explicite une relation de Bézout pour $15, 21, 35$ :
    \[ \underbrace{-1}_{=u_1} \times \underbrace{35}_{=b_1} + \underbrace{6}_{=u_2} \times \underbrace{21}_{=b_2} + \underbrace{(-6)}_{=u_3} \times \underbrace{15}_{=b_3} = 1 \]
    Reste à calculer
    \begin{align*}
      \overline{\varphi}^{-1}(1 + (3), 3 + (5), 0 + (7)) &= \sum_{i=1}^3 x_i u_i b_i + (105) \\
      &= 1 \times (-1) \times 35 + 3 \times 6 \times 21 + 0 \times (-6) \times 15  + (105) \\
      &= 343 + (105) \\
      &= 28 + (105)
    \end{align*}
    Les solutions sont bien de la forme escomptée.
  \end{proof}

  \cite{[ULM18]} utilise un autre algorithme pour trouver la solution. Le fait de chercher un antécédent permet de faire un lien ``direct'' avec le \cref{theoreme-chinois-1}. Attention, il faut réussir à trouver les coefficients de Bézout\dots
  %</content>
\end{document}
