\documentclass[12pt, a4paper]{report}

% LuaLaTeX :

\RequirePackage{iftex}
\RequireLuaTeX

% Packages :

\usepackage[french]{babel}
%\usepackage[utf8]{inputenc}
%\usepackage[T1]{fontenc}
\usepackage[pdfencoding=auto, pdfauthor={Hugo Delaunay}, pdfsubject={Mathématiques}, pdfcreator={agreg.skyost.eu}]{hyperref}
\usepackage{amsmath}
\usepackage{amsthm}
%\usepackage{amssymb}
\renewcommand{\proofname}{Solution}
\usepackage{stmaryrd}
\usepackage{tikz}
\usepackage{tkz-euclide}
\usepackage{fontspec}
\defaultfontfeatures[Erewhon]{FontFace = {bx}{n}{Erewhon-Bold.otf}}
\usepackage{fourier-otf}
\usepackage[nobottomtitles*]{titlesec}
\usepackage{fancyhdr}
\usepackage{listings}
\usepackage{catchfilebetweentags}
\usepackage[french, capitalise, noabbrev]{cleveref}
\usepackage[fit, breakall]{truncate}
\usepackage[top=2.5cm, right=2cm, bottom=2.5cm, left=2cm]{geometry}
\usepackage{enumitem}
\usepackage{tablists} %Pour faire 1)  2) 3)
\usepackage{tocloft}
\usepackage{microtype}
%\usepackage{mdframed}
%\usepackage{thmtools}
\usepackage{xcolor}
\usepackage{tabularx}
\usepackage{xltabular}
\usepackage{aligned-overset}
\usepackage[subpreambles=true]{standalone}
\usepackage{environ}
\usepackage[normalem]{ulem}
\usepackage{multicol}
 \usepackage{variations}
\usepackage{array}% Pour faire des tableaux
\usepackage{etoolbox}
\usepackage{setspace}
\usepackage[bibstyle=reading, citestyle=draft]{biblatex}
\usepackage{xpatch}
\usepackage[many, breakable]{tcolorbox}
\usepackage[backgroundcolor=white, bordercolor=white, textsize=scriptsize]{todonotes}
\usepackage{luacode}
\usepackage{float}
\usepackage{needspace}


% Police :

\setmathfont{Erewhon Math}

% Tikz :

\usetikzlibrary{calc}
\usetikzlibrary{3d}

% Longueurs :

\setlength{\parindent}{0pt}
\setlength{\headheight}{15pt}
\setlength{\fboxsep}{0pt}
\titlespacing*{\chapter}{0pt}{-20pt}{10pt}
\setlength{\marginparwidth}{1.5cm}
\setstretch{1.1}

% Métadonnées :

\author{agreg.skyost.eu}
\date{\today}

% Titres :

\setcounter{secnumdepth}{3}

\renewcommand{\thechapter}{\Roman{chapter}}
\renewcommand{\thesubsection}{\Roman{subsection}}
\renewcommand{\thesubsubsection}{\arabic{subsubsection}}
\renewcommand{\theparagraph}{\alph{paragraph}}

\titleformat{\chapter}{\huge\bfseries}{\thechapter}{20pt}{\huge\bfseries}
\titleformat*{\section}{\LARGE\bfseries}
\titleformat{\subsection}{\Large\bfseries}{\thesubsection \, - \,}{0pt}{\Large\bfseries}
\titleformat{\subsubsection}{\large\bfseries}{\thesubsubsection. \,}{0pt}{\large\bfseries}
\titleformat{\paragraph}{\bfseries}{\theparagraph. \,}{0pt}{\bfseries}

\setcounter{secnumdepth}{4}

% Table des matières :

\renewcommand{\cftsecleader}{\cftdotfill{\cftdotsep}}
\addtolength{\cftsecnumwidth}{10pt}

% Redéfinition des commandes :

\renewcommand*\thesection{\arabic{section}}
\renewcommand{\ker}{\mathrm{Ker}}

% Nouvelles commandes :

\newcommand{\website}{http://sencoursdemaths.com}

\newcommand{\tr}[1]{\mathstrut ^t #1}
\newcommand{\im}{\mathrm{Im}}
\newcommand{\rang}{\operatorname{rang}}
\newcommand{\trace}{\operatorname{trace}}
\newcommand{\id}{\operatorname{id}}
\newcommand{\stab}{\operatorname{Stab}}
\newcommand{\paren}[1]{\left(#1\right)}
\newcommand{\accol}[1]{\left\{#1\right\}}
\newcommand{\croch}[1]{\left[ #1 \right]}
\newcommand{\Grdcroch}[1]{\Bigl[ #1 \Bigr]}
\newcommand{\grdcroch}[1]{\bigl[ #1 \bigr]}
\newcommand{\abs}[1]{\left\lvert #1 \right\rvert}
\newcommand{\limi}[3]{\displaystyle \lim_{#1\to #2}#3}
\newcommand{\pinf}{+\infty}
\newcommand{\minf}{-\infty}
%%%%%%%%%%%%%% ENSEMBLES %%%%%%%%%%%%%%%%%
\newcommand{\ensemblenombre}[1]{\mathbb{#1}}
\newcommand{\Nn}{\ensemblenombre{N}}
\newcommand{\Zz}{\ensemblenombre{Z}}
\newcommand{\Qq}{\ensemblenombre{Q}}
\newcommand{\Qqp}{\Qq^+}
\newcommand{\Rr}{\ensemblenombre{R}}
\newcommand{\Cc}{\ensemblenombre{C}}
\newcommand{\Nne}{\Nn^*}
\newcommand{\Zze}{\Zz^*}
\newcommand{\Zzn}{\Zz^-}
\newcommand{\Qqe}{\Qq^*}
\newcommand{\Rre}{\Rr^*}
\newcommand{\Rrp}{\Rr_+}
\newcommand{\Rrm}{\Rr_-}
\newcommand{\Rrep}{\Rr_+^*}
\newcommand{\Rrem}{\Rr_-^*}
\newcommand{\Cce}{\Cc^*}
%%%%%%%%%%%%%%  INTERVALLES %%%%%%%%%%%%%%%%%
\newcommand{\intff}[2]{\left[#1\;,\; #2\right]  }
\newcommand{\intof}[2]{\left]#1 \;, \;#2\right]  }
\newcommand{\intfo}[2]{\left[#1 \;,\; #2\right[  }
\newcommand{\intoo}[2]{\left]#1 \;,\; #2\right[  }



\providecommand{\newpar}{\\[\medskipamount]}

\newcommand{\annexessection}{%
  \newpage%
  \subsection*{Annexes}%
}

\providecommand{\lesson}[3]{%
  \title{#3}%
  \hypersetup{pdftitle={#2 : #3}}%
  \setcounter{section}{\numexpr #2 - 1}%
  \section{#3}%
  \fancyhead[R]{\truncate{0.73\textwidth}{#2 : #3}}%
}

\providecommand{\development}[3]{%
  \title{#3}%
  \hypersetup{pdftitle={#3}}%
  \section*{#3}%
  \fancyhead[R]{\truncate{0.73\textwidth}{#3}}%
}

\providecommand{\sheet}[3]{\development{#1}{#2}{#3}}

\providecommand{\ranking}[1]{%
  \title{Terminale #1}%
  \hypersetup{pdftitle={Terminale #1}}%
  \section*{Terminale #1}%
  \fancyhead[R]{\truncate{0.73\textwidth}{Terminale #1}}%
}

\providecommand{\summary}[1]{%
  \textit{#1}%
  \par%
  \medskip%
}

\tikzset{notestyleraw/.append style={inner sep=0pt, rounded corners=0pt, align=center}}

%\newcommand{\booklink}[1]{\website/bibliographie\##1}
\newcounter{reference}
\newcommand{\previousreference}{}
\providecommand{\reference}[2][]{%
  \needspace{20pt}%
  \notblank{#1}{
    \needspace{20pt}%
    \renewcommand{\previousreference}{#1}%
    \stepcounter{reference}%
    \label{reference-\previousreference-\thereference}%
  }{}%
  \todo[noline]{%
    \protect\vspace{20pt}%
    \protect\par%
    \protect\notblank{#1}{\cite{[\previousreference]}\\}{}%
    \protect\hyperref[reference-\previousreference-\thereference]{p. #2}%
  }%
}

\definecolor{devcolor}{HTML}{00695c}
\providecommand{\dev}[1]{%
  \reversemarginpar%
  \todo[noline]{
    \protect\vspace{20pt}%
    \protect\par%
    \bfseries\color{devcolor}\href{\website/developpements/#1}{[DEV]}
  }%
  \normalmarginpar%
}

% En-têtes :

\pagestyle{fancy}
\fancyhead[L]{\truncate{0.23\textwidth}{\thepage}}
\fancyfoot[C]{\scriptsize \href{\website}{\texttt{http://sencoursdemaths.com}}}

% Couleurs :

\definecolor{property}{HTML}{ffeb3b}
\definecolor{proposition}{HTML}{ffc107}
\definecolor{lemma}{HTML}{ff9800}
\definecolor{theorem}{HTML}{f44336}
\definecolor{corollary}{HTML}{e91e63}
\definecolor{definition}{HTML}{673ab7}
\definecolor{notation}{HTML}{9c27b0}
\definecolor{example}{HTML}{00bcd4}
\definecolor{cexample}{HTML}{795548}
\definecolor{application}{HTML}{009688}
\definecolor{remark}{HTML}{3f51b5}
\definecolor{algorithm}{HTML}{607d8b}
\definecolor{proof}{HTML}{e1f5fe}
\definecolor{exercice}{HTML}{e1f5fe}

% Théorèmes :

\theoremstyle{definition}
\newtheorem{theorem}{Théorème}

\newtheorem{property}[theorem]{Propriété}
\newtheorem{proposition}[theorem]{Proposition}
\newtheorem{lemma}[theorem]{Activité d'introduction}
\newtheorem{corollary}[theorem]{Conséquence}

\newtheorem{definition}[theorem]{Définition}
\newtheorem{notation}[theorem]{Notation}

\newtheorem{example}[theorem]{Exemple}
\newtheorem{cexample}[theorem]{Contre-exemple}
\newtheorem{application}[theorem]{Application}

\newtheorem{algorithm}[theorem]{Algorithme}
\newtheorem{exercice}[theorem]{Exercice}

\theoremstyle{remark}
\newtheorem{remark}[theorem]{Remarque}




\counterwithin*{theorem}{section}

\newcommand{\applystyletotheorem}[1]{
  \tcolorboxenvironment{#1}{
    enhanced,
    breakable,
    colback=#1!8!white,
    %right=0pt,
    %top=8pt,
    %bottom=8pt,
    boxrule=0pt,
    frame hidden,
    sharp corners,
    enhanced,borderline west={4pt}{0pt}{#1},
    %interior hidden,
    sharp corners,
    after=\par,
  }
}

\applystyletotheorem{property}
\applystyletotheorem{proposition}
\applystyletotheorem{lemma}
\applystyletotheorem{theorem}
\applystyletotheorem{corollary}
\applystyletotheorem{definition}
\applystyletotheorem{notation}
\applystyletotheorem{example}
\applystyletotheorem{cexample}
\applystyletotheorem{application}
\applystyletotheorem{remark}
%\applystyletotheorem{proof}
\applystyletotheorem{algorithm}
\applystyletotheorem{exercice}

% Environnements :

\NewEnviron{whitetabularx}[1]{%
  \renewcommand{\arraystretch}{2.5}
  \colorbox{white}{%
    \begin{tabularx}{\textwidth}{#1}%
      \BODY%
    \end{tabularx}%
  }%
}

% Maths :

\DeclareFontEncoding{FMS}{}{}
\DeclareFontSubstitution{FMS}{futm}{m}{n}
\DeclareFontEncoding{FMX}{}{}
\DeclareFontSubstitution{FMX}{futm}{m}{n}
\DeclareSymbolFont{fouriersymbols}{FMS}{futm}{m}{n}
\DeclareSymbolFont{fourierlargesymbols}{FMX}{futm}{m}{n}
\DeclareMathDelimiter{\VERT}{\mathord}{fouriersymbols}{152}{fourierlargesymbols}{147}

% Code :

\definecolor{greencode}{rgb}{0,0.6,0}
\definecolor{graycode}{rgb}{0.5,0.5,0.5}
\definecolor{mauvecode}{rgb}{0.58,0,0.82}
\definecolor{bluecode}{HTML}{1976d2}
\lstset{
  basicstyle=\footnotesize\ttfamily,
  breakatwhitespace=false,
  breaklines=true,
  %captionpos=b,
  commentstyle=\color{greencode},
  deletekeywords={...},
  escapeinside={\%*}{*)},
  extendedchars=true,
  frame=none,
  keepspaces=true,
  keywordstyle=\color{bluecode},
  language=Python,
  otherkeywords={*,...},
  numbers=left,
  numbersep=5pt,
  numberstyle=\tiny\color{graycode},
  rulecolor=\color{black},
  showspaces=false,
  showstringspaces=false,
  showtabs=false,
  stepnumber=2,
  stringstyle=\color{mauvecode},
  tabsize=2,
  %texcl=true,
  xleftmargin=10pt,
  %title=\lstname
}

\newcommand{\codedirectory}{}
\newcommand{\inputalgorithm}[1]{%
  \begin{algorithm}%
    \strut%
    \lstinputlisting{\codedirectory#1}%
  \end{algorithm}%
}




% Bibliographie :

%\addbibresource{\bibliographypath}%
\defbibheading{bibliography}[\bibname]{\section*{#1}}
\renewbibmacro*{entryhead:full}{\printfield{labeltitle}}%
\DeclareFieldFormat{url}{\newline\footnotesize\url{#1}}%

\AtEndDocument{%
  \newpage%
  \pagestyle{empty}%
  \printbibliography%
}


\begin{document}
  %<*content>
  \lesson{algebra}{191}{Exemples d'utilisation de techniques d'algèbre en géométrie.}

  \subsection{Utilisation des nombres complexes}

  On se place dans un plan affine euclidien $\mathcal{P}$ muni d'un repère orthonormé $\mathcal{R} = (O, \overrightarrow{i}, \overrightarrow{j})$.

  \subsubsection{Module, argument}

  \reference[ROM21]{97}

  \begin{theorem}
    L'application
    \[
      \begin{array}{ccc}
        \mathcal{R} &\rightarrow& \mathbb{C} \\
        (x,y) &\mapsto& x+iy
      \end{array}
    \]
    est une bijection.
  \end{theorem}

  En utilisant cette identification entre $\mathcal{P}$ et $\mathbb{C}$, on peut identifier tout point du plan à un nombre complexe.

  \begin{theorem}
    Soient $A$ et $B$ deux points dont on note $a$ et $b$ les complexes associés.
    \begin{enumerate}[label=(\roman*)]
      \item $\vert a \vert = OA$.
      \item $\vert b - a \vert = AB$.
      \item Soit $r \in \mathbb{R}_*^+$. L'ensemble des nombres complexes $z$ tels que $\vert z - a \vert = r$ (resp. $\vert z - a \vert < r$ / $\vert z - a \vert \leq r$) est le cercle (resp. le disque ouvert / fermé) de centre $A$ et de rayon $r$.
      \item Un point $M$ d'affixe $z$ est sur la médiatrice de $[AB]$ si et seulement si $\vert z - a \vert = \vert z - b \vert$.
    \end{enumerate}
  \end{theorem}

  \begin{proposition}[Inégalité triangulaire]
    Soient $z_1, \dots, z_n \in \mathbb{C}^*$ avec $n \geq 2$, on a
    \[ \left| \sum_{k=1}^n z_k \right| \leq \sum_{k=1}^n \vert z_k \vert \]
    l'égalité étant réalisée si et seulement si $z_1, \dots, z_n$ sont linéairement liés.
  \end{proposition}

  \begin{remark}
    En reprenant les notations précédentes, et en désignant par $M_1, \dots, M_n$ les points associés aux complexes $z_1, \dots, z_n$, l'égalité
    \[ \left\Vert \sum_{k=1}^n \overrightarrow{OM_k} \right\Vert = \sum_{k=1}^n \left\Vert \overrightarrow{OM_k} \right\Vert \]
    est équivalente à dire que les points $O, M_1, \dots, M_n$ sont alignés.
  \end{remark}

  \begin{theorem}
    Si $z$ est un nombre complexe de module $1$, il existe un unique réel $\theta \in [-\pi, \pi[$ tel que
    \[ z = \cos(\theta) + i\sin(\theta) \]
  \end{theorem}

  \begin{definition}
    On dit qu'un réel $\theta$ est un \textbf{argument} du nombre complexe $z$ non nul si
    \[ \frac{z}{\vert z \vert} = \cos(\theta) + i\sin(\theta) \]
  \end{definition}

  \begin{theorem}
    Soient $\overrightarrow{v_1}$ et $\overrightarrow{v_2}$ deux vecteurs non nuls du plan. On note $z_1$ et $z_2$ les complexes associés.
    \begin{enumerate}[label=(\roman*)]
      \item Si $\theta_1$ est un argument de $z_1$, alors c'est également une mesure de l'angle orienté $\widehat{(\overrightarrow{i}, \overrightarrow{v_1})}$.
      \item Un argument de $\frac{z_2}{z_1}$ est une mesure de l'angle orienté $\theta = \widehat{(\overrightarrow{v_1}, \overrightarrow{v_2})}$ et on a :
      \[ \cos(\theta) = \frac{\langle \overrightarrow{v_1}, \overrightarrow{v_2} \rangle}{\Vert \overrightarrow{v_1} \Vert \Vert \overrightarrow{v_2} \Vert} \text{ et } \sin(\theta) = \frac{\det \left( \overrightarrow{v_1}, \overrightarrow{v_2} \right)}{\Vert \overrightarrow{v_1} \Vert \Vert \overrightarrow{v_2} \Vert} \]
      où $\langle ., . \rangle$ désigne le produit scalaire canonique.
    \end{enumerate}
  \end{theorem}

  \subsubsection{Le triangle dans le plan complexe}

  \reference{105}

  \begin{definition}
    Un \textbf{vrai triangle} dans le plan $\mathcal{P}$ est la donnée de trois points non alignés $A$, $B$ et $C$. Un tel triangle est noté $\mathcal{T} = ABC$.
  \end{definition}

  Soit $\mathcal{T} = ABC$ un vrai triangle. On note $a$, $b$ et $c$ les complexes associés respectivement à $A$, $B$ et $C$.

  \begin{theorem}
    L'aire de $ABC$ est
    \[ \frac{1}{2} \left| \det\left( \overrightarrow{AB}, \overrightarrow{AC} \right) \right| \]
  \end{theorem}

  \begin{proposition}
    Le trois médianes de $\mathcal{T}$ concourent au point dont le complexe associé est
    \[ \frac{a+b+c}{3} \]
  \end{proposition}

  \begin{definition}
    Le point précédent est appelé \textbf{centre de gravité} de $\mathcal{T}$. C'est aussi l'\textbf{isobarycentre} des points $A$, $B$ et $C$.
  \end{definition}

  \begin{proposition}
    Le trois hauteurs de $\mathcal{T}$ concourent au point dont le complexe associé est
    \[ a_\Omega + b_\Omega + c_\Omega \]
    où $a_\Omega$, $b_\Omega$, $c_\Omega$ sont les complexes associés aux points $A$, $B$ et $C$ considérés dans le repère $(\Omega, \overrightarrow{i}, \overrightarrow{j})$ avec $\Omega$ centre du cercle circonscrit au triangle $\mathcal{T}$.
  \end{proposition}

  \begin{definition}
    Le point précédent est appelé \textbf{orthocentre} de $\mathcal{T}$.
  \end{definition}

  \begin{proposition}
    Dans un vrai triangle, orthocentre, centre du cercle circonscrit et centre de gravité sont alignés.
  \end{proposition}

  \subsubsection{Droites et cercles dans le plan complexe}

  \begin{theorem}
    Toute équation de la forme
    \[ \alpha z \overline{z} + \overline{\beta} z + \beta \overline{z} + \gamma = 0, \, \alpha, \gamma \in \mathbb{R}, \, \beta \in \mathbb{C} \]
    représente dans $\mathcal{P}$ :
    \begin{enumerate}[label=(\roman*)]
      \item $\mathcal{P}$ tout entier si $\alpha = \beta = \gamma = 0$.
      \item $\emptyset$ si :
      \begin{itemize}
        \item $\alpha = \beta = 0$ et $\gamma \neq 0$ ;
        \item ou $\alpha \neq 0$ et $\vert \beta \vert^2 - \alpha \gamma < 0$.
      \end{itemize}
      \item Une droite dirigée par le vecteur $\overrightarrow{v}$ représentant le complexe $i\beta$ si $\alpha = 0$ et $\beta \neq 0$.
      \item Le cercle dont le centre est associé au complexe $-\frac{\beta}{\alpha}$ et de rayon $\frac{\sqrt{\vert \beta \vert^2 - \alpha \gamma}}{\vert \alpha \vert}$ si $\alpha \neq 0$ et $\vert \beta \vert^2 - \alpha \gamma \geq 0$.
    \end{enumerate}
  \end{theorem}

  \begin{corollary}[Théorème d'Appolonius]
    Soient $a$ et $b$ deux nombres complexes distincts et $\lambda \in \mathbb{R}^+$. L'ensemble
    \[ E_\lambda = \{ z \in \mathbb{C} \mid \vert z - b \vert = \lambda \vert z - a \vert \} \]
    est identifié dans $\mathcal{P}$ ;
    \begin{itemize}
      \item À la médiatrice du segment $[AB]$ pour $\lambda = 1$.
      \item Au cercle de centre le complexe associé à $\frac{b-\lambda^2 a}{1 - \lambda^2}$ et de rayon $\frac{\lambda \vert a - b \vert}{\vert 1 - \lambda^2 \vert}$ pour $\lambda \neq 1$.
    \end{itemize}
  \end{corollary}

  \begin{theorem}
    Soient $A$, $B$, $C$ et $D$ des points deux à deux distincts associés respectivement aux complexes $a$, $b$, $c$ et $d$. Ces points sont alignés si et seulement si
    \[ \frac{c-b}{c-a} \frac{d-a}{d-b} \in \mathbb{R}^+ \]
  \end{theorem}

  \begin{corollary}[Théorème de Ptolémée]
    Soient $A$, $B$, $C$ et $D$ des points deux à deux distincts. Le quadrilatère convexe $ABCD$ est inscriptible dans un cercle si et seulement si
    \[ AC \times BD = AB \times CD + AD \times BC \]
  \end{corollary}

  \subsection{Utilisation de la théorie des groupes}

  \subsubsection{Actions de groupe}

  \paragraph{Cadre général}

  \reference[ULM21]{71}

  Soit $X$ un ensemble fini. On considère une action $\cdot$ de $G$ sur $X$.

  \begin{proposition}
    Soit $x \in X$. Alors :
    \begin{itemize}
      \item $|G \cdot x| = (G : \stab_G(x))$.
      \item $|G| = |\stab_G(x)| |G \cdot x|$.
      \item $|G \cdot x| = \frac{|G|}{|\stab_G(x)|}$
    \end{itemize}
  \end{proposition}

  \begin{theorem}[Formule des classes]
    Soit $\Omega$ un système de représentants des orbites de l'action de $G$ sur $X$. Alors,
    \[ |X| = \sum_{\omega \in \Omega} |G \cdot \omega| = \sum_{\omega \in \Omega} (G : \stab_G(\omega)) = \sum_{\omega \in \Omega} \frac{|G|}{|\stab_G(\omega)|} \]
  \end{theorem}

  \begin{definition}
    On définit :
    \begin{itemize}
      \item $X^G = \{ x \in X \mid \forall g \in G, \, g \cdot x = x \}$ l'ensemble des points de $X$ laissés fixes par tous les éléments de $G$.
      \item $X^g = \{ x \in X \mid g \cdot x = x \}$ l'ensemble des points de $X$ laissés fixes par $g \in G$.
    \end{itemize}
  \end{definition}

  \begin{theorem}[Formule de Burnside]
    Le nombre $r$ d'orbites de $X$ sous l'action de $G$ est donné par
    \[ r = \frac{1}{|G|} \sum_{g \in G} |X^g| \]
  \end{theorem}

  \reference[I-P]{121}

  \begin{application}
    Deux colorations des faces d'un cube sont les mêmes si on peut passer de l'une à l'autre par une isométrie du dodécaèdre. Alors, le nombre de colorations distinctes d'un cube avec $c$ couleurs est
    \[ \frac{c^2}{24} (c^4 + 3^2 + 12c + 8) \]
  \end{application}

  \paragraph{Espaces affines}

  \reference[ROM21]{73}

  On peut réécrire le définition d'un espace affine en termes d'actions de groupes.

  \begin{definition}
    Soit $E$ un espace vectoriel sur $\mathbb{R}$. Un \textbf{espace affine} $\mathcal{E}$ est un ensemble non vide qui agit (à droite) sur $E$ de manière simplement transitive. On note $\cdot$ l'action correspondante. Les éléments de $\mathcal{E}$ sont appelés \textbf{points} et les éléments de $E$ sont appelés \textbf{vecteurs}.
  \end{definition}

  \begin{remark}
    Ainsi, pour tout couple $(x,y) \in \mathcal{E}$, il existe un unique $u \in E$ tel que $y = x \cdot u$. On note alors $u = \overrightarrow{xy}$.
  \end{remark}

  Le reste de la théorie découle de cette remarque.

  \subsubsection{Groupe diédral}

  \reference[ULM21]{8}

  \begin{definition}
    Pour un entier $n \geq 1$, le \textbf{groupe diédral} $D_n$ est le sous-groupe, de $\mathrm{GL}_2(\mathbb{R})$ engendré par la symétrie axiale $s$ et la rotation d'angle $\theta = \frac{2\pi}{n}$ définies respectivement par les matrices
    \[
    S =
    \begin{pmatrix}
      1 & 0 \\
      0 & -1
    \end{pmatrix}
    \text{ et }
    R =
    \begin{pmatrix}
      \cos(\theta) & -\sin(\theta) \\
      \sin(\theta) & \cos(\theta)
    \end{pmatrix}
    \]
  \end{definition}

  \begin{example}
    $D_1 = \{ \operatorname{id}, s \}$.
  \end{example}

  \begin{proposition}
    \begin{enumerate}[label=(\roman*)]
      \item $D_n$ est un groupe d'ordre $2n$.
      \item $r^n = s^2 = \operatorname{id}$ et $sr = r^{-1}s$.
    \end{enumerate}
  \end{proposition}

  \reference{28}

  \begin{proposition}
    Un groupe non cyclique d'ordre $4$ est isomorphe à $D_2$.
  \end{proposition}

  \reference{65}

  \begin{example}
    $S_2$ est isomorphe à $D_2$.
  \end{example}

  \reference{28}

  \begin{proposition}
    Un groupe fini d'ordre $2p$ avec $p$ premier est soit cyclique, soit isomorphe à $D_p$.
  \end{proposition}

  \begin{example}
    $S_3$ est isomorphe à $D_3$.
  \end{example}

  \reference{47}

  \begin{proposition}
    Les sous-groupes de $D_n$ sont soit cyclique, soit isomorphes à un $D_m$ où $m \mid n$.
  \end{proposition}

  \reference[ROM21]{84}

  \begin{theorem}
    On désigne par $\Gamma_n$ l'ensemble des sommets d'un polygone à $n$ côtés et par $\mathrm{Is}(\Gamma_n)$ l'ensemble des isométries qui conservent $\Gamma_n$. Alors,
    \[ \mathrm{Is}(\Gamma_n) = D_n \]
  \end{theorem}

  \begin{example}
    Les isométries conservant un triangle équilatéral sont les éléments de $D_3$.
  \end{example}

  \subsection{Utilisation de la théorie des corps}

  \reference[GOZ]{47}

  On note $\mathcal{P}$ un plan affine euclidien muni d'un repère orthonormé direct $\mathcal{R} = (O, \overrightarrow{i}, \overrightarrow{j})$. On s'autorise à identifier chaque point $M \in \mathcal{P}$ avec ses coordonnées $(x,y) \in \mathbb{R}^2$ dans $\mathcal{R}$.

  \begin{definition}
    On dit qu'un point $M \in \mathcal{P}$ est \textbf{constructible} (sous-entendu \textit{à la règle et au compas}) si on peut le construire en utilisant uniquement la règle et le compas, en supposant $O$ et $I=(1,0)$ déjà construits.
  \end{definition}

  \begin{proposition}
    Soient $A$, $B$ deux points constructibles distincts.
    \begin{enumerate}[label=(\roman*)]
      \item Si $A$ est constructible, son symétrique par rapport à $O$ l'est aussi.
      \item $J = (0,1)$ est constructible.
      \item Si $C$ est un point constructible, on peut construire à la règle et au compas la perpendiculaire à $(AB)$ passant par $C$.
      \item Si $C$ est un point constructible, on peut construire à la règle et au compas la parallèle à $(AB)$ passant par $C$.
    \end{enumerate}
  \end{proposition}

  \begin{proposition}
    Soit $x \in \mathbb{R}$.
    \[ (x,0) \text{ est constructible} \iff (0,x) \text{ est constructible} \]
  \end{proposition}

  \begin{definition}
    Un nombre vérifiant la proposition précédente est dit \textbf{nombre constructible}.
  \end{definition}

  \begin{proposition}
    \begin{enumerate}[label=(\roman*)]
      \item Tout élément de $\mathbb{Q}$ est constructible.
      \item $(x,y)$ est constructible si et seulement si $x$ et $y$ le sont.
    \end{enumerate}
  \end{proposition}

  \begin{theorem}
    L'ensemble $\mathbb{E}$ des nombres constructibles est un sous-corps de $\mathbb{R}$ stable par racine carrée.
  \end{theorem}

  \dev{theoreme-de-wantzel}

  \begin{theorem}[Wantzel]
    Soit $t \in \mathbb{R}$. $t$ est constructible si et seulement s'il existe une suite fini $(L_0, \dots, L_p)$ de sous-corps de $\mathbb{R}$ vérifiant :
    \begin{enumerate}[label=(\roman*)]
      \item $L_0 = \mathbb{Q}$.
      \item $\forall i \in \llbracket 1, p-1 \rrbracket$, $L_i$ est une extension quadratique de $L_{i-1}$.
      \item $t \in L_p$.
    \end{enumerate}
  \end{theorem}

  \begin{corollary}
    \begin{enumerate}[label=(\roman*)]
      \item Si $x$ est constructible, le degré de l'extension $\mathbb{Q}[x]$ sur $\mathbb{Q}$ est de la forme $2^s$ pour $s \in \mathbb{N}$.
      \item Tout nombre constructible est algébrique.
    \end{enumerate}
  \end{corollary}

  \begin{cexample}
    \begin{itemize}
      \item $\sqrt[3]{2}$ est algébrique, non constructible.
      \item $\sqrt{\pi}$ est transcendant et n'est donc pas constructible.
    \end{itemize}
  \end{cexample}

  \begin{application}[Quadrature du cercle]
    Il est impossible de construire, à la règle et au compas, un carré ayant même aire qu'un disque donné.
  \end{application}

  \begin{application}[Duplication du cube]
    Il est impossible de construire, à la règle et au compas, l'arête d'un cube ayant un volume double de celui d'un cube donné.
  \end{application}

  \subsection{Utilisation de l'algèbre linéaire}

  \subsubsection{Déterminant et volume}

  \reference[GRI]{130}

  \begin{theorem}
    L'aire $\mathcal{A}(v,w)$ du parallélogramme engendré par deux vecteurs $v, w \in \mathbb{R}^n$ est égale à
    \[ \mathcal{A}(v,w) = \vert \det(v,w) \vert \]
  \end{theorem}

  \begin{corollary}
    Soient $v_1, \dots, v_n \in \mathbb{R}^n$. On note $\mathcal{V}(v_1, \dots, v_n)$ le volume du parallélépipède rectangle engendré par $v_1, \dots, v_n$ (ie. l'ensemble $\{ z \in \mathbb{R}^n \mid z = \sum_{i=1}^n \lambda_i v_i, \, \lambda_i \in [0,1] \}$). On a alors :
    \[ \mathcal{V}(v_1, \dots, v_n) = \vert \det(v_1, \dots, v_n) \vert \]
  \end{corollary}

  \subsubsection{Étude d'une suite de polygones}

  \reference[GOU21]{153}

  \begin{proposition}[Déterminant circulant]
    Soient $n \in \mathbb{N}^*$ et $a_1, \dots, a_n \in \mathbb{C}$. On pose $\omega = e^{\frac{2i\pi}{n}}$. Alors
    \[ \begin{vmatrix} a_0 & a_1 & \dots & a_{n-1} \\ a_{n-1} & a_0 & \dots & a_{n-2}\\ \vdots & \vdots & \ddots & \vdots \\ a_1 & a_2 & \dots & a_0 \end{vmatrix} = \prod_{j=0}^{n-1} P(\omega^j) \]
    où $P = \sum_{k=0}^{n-1} a_k X^k$.
  \end{proposition}

  \reference[I-P]{389}
  \dev{suite-de-polygones}

  \begin{application}[Suite de polygones]
    Soit $P_0$ un polygone dont les sommets sont $\{ z_{0,1}, \dots, z_{0,n} \}$. On définit la suite de polygones $(P_k)$ par récurrence en disant que, pour tout $k \in \mathbb{N}^*$, les sommets de $P_{k+1}$ sont les milieux des arêtes de $P_k$.
    \newpar
    Alors la suite $(P_k)$ converge vers l'isobarycentre de $P_0$.
  \end{application}

  \subsubsection{Groupe spécial orthogonal en dimension \texorpdfstring{$2$}{2} et \texorpdfstring{$3$}{3}}

  \reference[GRI]{241}

  \begin{definition}
    On définit $\mathrm{SO}(E) = \{ u \in \mathcal{O}(E) \mid \det(u) = 1 \}$ et $\mathrm{SO}_n(\mathbb{R}) = \{ A \in \mathcal{O}_n(\mathbb{R}) \mid \det(A) = 1 \}$
  \end{definition}

  \reference[ROM21]{724}

  \begin{proposition}
    $\mathrm{SO}(E)$ est un sous-groupe distingué de $\mathcal{O}(E)$ d'indice $2$ (de même que $\mathrm{SO}_n(\mathbb{R})$ dans $\mathcal{O}_n(\mathbb{R})$).
  \end{proposition}

  \reference[GRI]{241}

  \begin{example}
    \[ \frac{1}{3} \begin{pmatrix} 2 & -1 & 2 \\ 2 & 2 & -1 \\ -1 & 2 & 2 \end{pmatrix} \in \mathrm{SO}_3(\mathbb{R}) \]
  \end{example}

  \begin{theorem}
    Soit $A \in \mathcal{O}_2(\mathbb{R})$. Alors :
    \begin{itemize}
      \item \uline{Si $A \in \mathrm{SO}_2(\mathbb{R})$ :}
      \[ \exists \theta \in \mathbb{R} \text{ tel que } A = \begin{pmatrix} \cos(\theta) & -\sin(\theta) \\ \sin(\theta) & \cos(\theta) \end{pmatrix} \]
      (rotation d'angle $\theta$).
      \item \uline{Si $A \notin \mathrm{SO}_2(\mathbb{R})$ :}
      \[ \exists \theta \in \mathbb{R} \text{ tel que } A = \begin{pmatrix} \cos(\theta) & \sin(\theta) \\ \sin(\theta) & -\cos(\theta) \end{pmatrix} \]
      (symétrie orthogonale par rapport à la droite d'angle polaire $\frac{\theta}{2}$).
    \end{itemize}
  \end{theorem}

  \begin{theorem}
    Soit $A \in \mathcal{O}_3(\mathbb{R})$ et $u$ l'endomorphisme de $E$ dont la matrice dans la base canonique est $A$. Alors, il existe $\mathcal{B}$ une base orthonormée de $E$ telle que la matrice de $u$ dans $\mathcal{B}$ est
    \[ \begin{pmatrix} \cos(\theta) & -\sin(\theta) & 0 \\ \sin(\theta) & \cos(\theta) & 0 \\ 0 & 0 & \epsilon \end{pmatrix} \]
    avec $\epsilon = \pm 1$. On note $E_\epsilon$ le sous-espace vectoriel associé à la valeur propre $\epsilon$.
    \begin{itemize}
      \item \uline{Si $\epsilon = 1$ :} $f \in \mathrm{SO}(E)$ est la rotation d'angle $2\cos(\theta) + 1$ autour de l'axe $E_1$.
      \item \uline{Si $\epsilon = -1$ :} $f \notin \mathrm{SO}(E)$ est la composée de la rotation d'angle $2\cos(\theta) - 1$ autour de l'axe $E_{-1}$ avec la symétrie orthogonale par rapport à $E_{-1}^{\perp}$.
    \end{itemize}
  \end{theorem}

  \reference[ULM21]{138}

  \begin{theorem}
    Soit $G$ un sous-groupe fini de $\mathrm{SO}_3(\mathbb{R})$. Alors, $G$ est isomorphe à $\mathbb{Z}/n\mathbb{Z}$, $D_n$, $A_4$, $S_4$ ou $A_5$ (où $n \geq 2$).
  \end{theorem}

  \begin{application}[Solides de Platon]
    Il y a cinq polyèdres réguliers : le tétraèdre, le cube, l'octaèdre, le dodécaèdre et l'icosaèdre.
  \end{application}

  \newpage
  \section*{Annexes}

  \reference[I-P]{389}

  \begin{figure}[h]
    \begin{center}
      \begin{tikzpicture}
        \coordinate (A) at (0:3);
        \coordinate (B) at (72:3);
        \coordinate (C) at (2*72:3);
        \coordinate (D) at (3*72:3);
        \coordinate (E) at (4*72:3);
        \coordinate (F) at (A);
        \foreach \i in {0,...,10} {
          \draw(A) node {$\bullet$};
          \draw(B) node {$\bullet$};
          \draw(C) node {$\bullet$};
          \draw(D) node {$\bullet$};
          \draw(E) node {$\bullet$};
          \draw[fill=cyan!60, fill opacity=0.2](A) -- (B) -- (C) -- (D) -- (E) -- (A);
          \coordinate (A) at ($(A)!0.5!(B)$);
          \coordinate (B) at ($(B)!0.5!(C)$);
          \coordinate (C) at ($(C)!0.5!(D)$);
          \coordinate (D) at ($(D)!0.5!(E)$);
          \coordinate (E) at ($(E)!0.5!(F)$);
          \coordinate (F) at (A);
        }
      \end{tikzpicture}
    \end{center}
    \caption{La suite de polygones.}
  \end{figure}

  \reference[GRI]{242}

  \begin{figure}[H]
    \begin{center}
      \begin{tikzpicture}
        \draw[->] (-1, 0) -- (5, 0);
        \draw[->] (0, -1) -- (0, 5);
        \draw [teal,dashed,domain=-10:100] plot ({4*cos(\x)}, {4*sin(\x)});
        \draw [->,color=teal,domain=10:60] plot ({cos(\x)}, {sin(\x)});
        \node at (35:1.3) {\color{teal}$\theta$};
        \draw (0,0) -- (10:4) node {\color{orange}$\bullet$} node[right] {\color{orange}$x$};
        \draw (0,0) -- (60:4) node {\color{orange}$\bullet$} node[above right] {\color{orange}$Ax$};
        \node at (2,-1) [below] {$A \in \mathrm{SO}_2(\mathbb{R})$};
      \end{tikzpicture}
      \hfill
      \begin{tikzpicture}
        \coordinate (A) at (20:4);
        \coordinate (B) at (80:4);
        \draw[->] (-1, 0) -- (5, 0);
        \draw[->] (0, -1) -- (0, 5);
        \draw[dashed] (A) -- (B);
        \draw[teal] (0, 0) -- (50:5) node [above right] {\color{teal}$\frac{\theta}{2}$};
        \draw (0,0) -- (A) node {\color{orange}$\bullet$} node[right] {\color{orange}$x$};
        \draw (0,0) -- (B) node {\color{orange}$\bullet$} node[above right] {\color{orange}$Ax$};
        \node at (2,-1) [below] {$A \notin \mathrm{SO}_2(\mathbb{R})$};
      \end{tikzpicture}
    \end{center}
    \caption{Le groupe $\mathcal{O}_2(\mathbb{R})$.}
  \end{figure}

  \reference{244}

  \begin{figure}[H]
    \begin{center}
      \begin{tikzpicture}
        \draw[->] (-1, 0, 0) -- (5, 0, 0);
        \draw[->] (0, -1, 0) -- (0, 5, 0) node [above] {$E_1$};
        \draw[->] (0, 0, -1) -- (0, 0, 5);
        \begin{scope}[canvas is zx plane at y=4]
          \coordinate (O) at (0,0);
          \draw[teal,dashed] (O) circle (2);
          \coordinate (A) at (10:2);
          \coordinate (B) at (60:2);
          \draw [->,color=teal,domain=10:60] plot ({cos(\x)}, {sin(\x)});
          \node at (35:1.5) {\color{teal}$\theta$};
          \draw (O) -- (A);
          \draw (O) -- (B);
        \end{scope}
        \draw (0,0,0) -- (A) node {\color{orange}$\bullet$} node[above left] {\color{orange}$x$};
        \draw (0,0,0) -- (B) node {\color{orange}$\bullet$} node[above] {\color{orange}$Ax$};
        \node at (2,-1) [below] {$A \in \mathrm{SO}_3(\mathbb{R})$};
      \end{tikzpicture}
      \hfill
      \begin{tikzpicture}
        \draw[->] (-1, 0, 0) -- (5, 0, 0);
        \draw[->] (0, -1, 0) -- (0, 5, 0) node [above] {$E_{-1}$};
        \draw[->] (0, 0, -1) -- (0, 0, 5);
        \begin{scope}[canvas is zx plane at y=4]
          \coordinate (O) at (0,0);
          \draw[teal,dashed] (O) circle (2);
          \coordinate (A) at (10:2);
          \coordinate (B) at (60:2);
          \draw [->,color=teal,domain=10:60] plot ({cos(\x)}, {sin(\x)});
          \node at (35:1.5) {\color{teal}$\theta$};
          \draw (O) -- (A);
          \draw (O) -- (B);
        \end{scope}
        \begin{scope}[canvas is zx plane at y=0]
          \coordinate (C) at (60:2);
        \end{scope}
        \draw (0,0,0) -- (A) node {\color{orange}$\bullet$} node[above left] {\color{orange}$x$};
        \draw[dashed] (0,0,0) -- (B) node {\color{orange}$\bullet$};
        \draw[dashed] (B) -- (C);
        \draw (0,0,0) -- (C) node {\color{orange}$\bullet$} node[below right] {\color{orange}$Ax$};
        \node at (2,-1) [below] {$A \notin \mathrm{SO}_3(\mathbb{R})$};
      \end{tikzpicture}
    \end{center}
    \caption{Le groupe $\mathcal{O}_3(\mathbb{R})$.}
  \end{figure}
  %</content>
\end{document}
